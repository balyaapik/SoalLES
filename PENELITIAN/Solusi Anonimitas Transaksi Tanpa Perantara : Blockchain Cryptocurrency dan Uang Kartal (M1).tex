\documentclass[12pt,a4paper]{article}
\usepackage[latin1]{inputenc}
\usepackage[bahasa]{babel}
\usepackage{amsmath}
\usepackage{amsfonts}
\usepackage{amssymb}
\usepackage{graphicx}
\usepackage[left=4.00cm, right=3.00cm, top=3.00cm, bottom=3.00cm]{geometry}
\author{Balya Rochmadi}
\title{CRYPTO-RUPIAH: A THIRD PARTY RECOGNITION PEER TO PEER CRYPTO-CURRENCY}
\begin{document}
	\maketitle
	\begin{abstract}
		\textit{Crypto-Rupiah adalah sebuah solusi masa depan dari uang rupiah yang memudahkan pengawasan, kontrol, serta legalitas transaksi dan memungkinkan pihak ketiga untuk membuat kebijakan terhadap jumlah uang beredar. Otoritas Sentral berfokus pada pengaturan makroprudensial sedangkan pembuatan uang baru diserahkan ke publik dengan menggunakan prinsip \textit{mathematical competition}  }
	\end{abstract}
	
	
	
	
	
	
	
	
	
\end{document}
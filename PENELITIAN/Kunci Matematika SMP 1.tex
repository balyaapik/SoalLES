\documentclass[12pt,a4paper]{article}
\usepackage[latin1]{inputenc}
\usepackage{amsmath}
\usepackage{amsfonts}
\usepackage{amssymb}
\usepackage{graphicx}
\usepackage[left=4.00cm, right=3.00cm, top=3.00cm, bottom=3.00cm]{geometry}
\author{H.O.W.K.E}
\title{KUNCI UJIAN MATEMATIKA 1 SMP}
\begin{document}
	\maketitle
	\begin{enumerate}
		\item $\frac{PQ}{AB}=\frac{QS}{BC}$ 
		\item Segitiga istimewa $90-60-30$ maka $\overline{RP}=2x$ 
		\item Perbandingan kesebangunan, semisal jarak sisi bawah bingkai adalah $x$ maka,
		\\ $\dfrac{2p}{2p-2a}= \dfrac{p}{p-a-x} $ \\
		\\
		\\ $\dfrac{2p}{2(p-a)}=\dfrac{p}{p-a-x}$ \\
		\\
		\\ $\dfrac{p}{p-a}=\dfrac{p}{p-a-x}$; bagi dua dengan dua \\
		\\
		\\ $\dfrac{1}{p-a}=\dfrac{1}{p-a-x}$; bagi $p$ dengan $p$ \\
		\\ $p-a=p-a-x$ \\
		\\ $0=-x$ \\
		\\ $x=0$ 
		\item Kunci utama untuk memecahkan masalah ini adalah penggunaan sifat segitiga istimewa, asumsikan bahwa terdapat garis $\overline{TG}$ yang membelah $\overline{MN}$ menjadi 2 sama panjang sehingga tercipta $\triangle TGQ$ dan $\triangle TGP$ dengan
		$\triangle TGQ \cong \triangle TGP$, perhatikan illustrasi berikut!
		\begin{center}
		\includegraphics[scale=0.5]{../Pictures/HOWKE/g11207}
		\end{center}
		Maka diperoleh $\measuredangle Q=\measuredangle P = 60 \textdegree$, untuk segitiga $\triangle TGQ$ dan  $\triangle TGP$ sudut $\measuredangle Q$ dan $\measuredangle P$ harus dibagi dua sama rata 
		sehingga terbentuk sudut $30 \textdegree$, $\measuredangle T$ dibagi dua untuk dua segitiga adalah $60\textdegree$. Dari gambar diatas $\triangle TGQ$ dan $\triangle TGP$ memenuhi persyaratan segitiga istimewa 90-60-30 sehingga \\
		
		$\overline{QG}=6$; karena $\overline{QP}=12$ dibagi 2
		\\
		karena dalam segitiga istimewa $\overline{QG}=\overline{TG}\sqrt{3}$
		berarti \\
		\\
		$\overline{TG}=\sqrt{3}=QG$\\
		\\
		$\overline{TG}=\sqrt{3}=6$\\
		\\
		$\overline{TG}=\dfrac{6}{\sqrt{3}}$\\
		\\
		$\overline{TG}=\dfrac{6}{\sqrt{3}}\times \dfrac{\sqrt{3}}{\sqrt{3}}$ ; rasionalisasi akar\\
		\\
		$\overline{TG}=\dfrac{6\sqrt{3}}{3}$\\
		\\
		$\overline{TG}=2\sqrt{3}$
		\\
		
		Berarti $\overline{TQ}=2\overline{TG}$ atau $6\sqrt{3}$ \\
		\\
		perhatikan illustrasi berikut!
		\begin{center}
		\includegraphics[scale=0.7]{../Pictures/HOWKE/g11242}
		\end{center}
		
		Kemudian kita buat garis bantu $\overline{MV}$ dan $\overline{NZ}$ sebagi garis tinggi yang panjangnya sama dengan $\overline{TG}$,perhatikan illustrasi berikut!
		\begin{center}
			\includegraphics[scale=0.7]{../Pictures/HOWKE/g11541}
		\end{center}
		
		maka terbentuk $\triangle MQV \cong \triangle NZP $ dan  $\overline{QV}=\overline{ZP}$ dan $\overline{MN}= \overline{NZ}$. Karena $\overline{TG}=\overline{MN}=\overline{NZ}=2\sqrt{3}$
		dan $\triangle MQV$ beserta $\triangle NZP$ adalah segitiga istimewa 90-60-30 menjadikan perhitungan MQ adalah sebagai berikut.\\
		
		karena dalam segitiga istimewa,\\
		\\
		$2\overline{QV}=\overline{MQ}$\\
		\\
		maka,\\
		\\
		$\overline{QV}\sqrt{3}=\overline{MV}$\\
		\\
		$\overline{QV}=\dfrac{2\sqrt{3}}{\sqrt{3}}$\\
		\\
		$\overline{QV}=2$\\
		\\
		Berarti,\\
		\\
		$\overline{MQ}=2 \times \overline{QV}$\\
		\\
		$\overline{MQ}=2 \times 2=4$
		
		Karena $\overline{QV}=2$ dan $\overline{QP}=\overline{ZP}$ dan $\overline{VZ}=\overline{MN}$maka,\\
		\\
		$\overline{MN}=\overline{QP}- (\ \overline{QV}+\overline{ZP} )\ $\\
		\\
		$\overline{MN}= 12-(2+2)=8$
		
		Perhatikan illustrasi berikut,
		\begin{center}
		\includegraphics[scale=0.7]{../Pictures/HOWKE/g11146}
		\end{center}
		
		Sehingga keliling trapesium MNQP= 4+4+8+12=28.	
		
		\item Karena trapesium tersebut adalah trapesium sama kaki dan teratur maka,
		\begin{enumerate}
			\item $\overline{AH}=18-6=12$ cm\\
			\\
			Asumsikan  $\overline{AB}=\overline{FG}=\overline{DC}$ maka $\overline{ED}=21-14=7 cm$ jadi,
			\item $\dfrac{\overline{HF}}{\overline{ED}}=\dfrac{\overline{AH}}{\overline{DE}}$\\
			\\
			$\dfrac{\overline{HF}}{7}=\dfrac{12}{18}$\\
			\\
			$\overline{HF}=\dfrac{12 \times 7}{18}$ \\
			\\
			$\overline{HF}=2\dfrac{2}{3}$ cm
			\item $\overline{ED}=7$ cm
			\item $\overline{HG}=2\dfrac{2}{3}+14=16\dfrac{2}{3} cm$  
		\end{enumerate}
		\item Jawaban adalah sebagai berikut:
		\begin{enumerate}
			\item $\overline{AD}=\sqrt{12\times 8}=\sqrt{96}=4 \sqrt{6}$
			\item $\overline{AC}=\sqrt{{12^2}\dotplus (4\sqrt{6})^2}=\sqrt{240}=4 \sqrt{15}$cm
			\item $\overline{AB}=\sqrt{{8^2}\dotplus (4\sqrt{6})^2}=\sqrt{160}=4 \sqrt{10}$cm
			\item Luas $\triangle ABC=\dfrac{1}{2} \times 8 \times 4\sqrt{6}=16\sqrt{6}$cm
			\item Luas $\triangle ADB=\dfrac{1}{2} \times 12 \times 4\sqrt{6}=24\sqrt{6}$cm
			\item Luas $\triangle ACD=\dfrac{1}{2} \times 4\sqrt{15} \times 4\sqrt{10}=16\sqrt{150}= 5\sqrt{6}$
		\end{enumerate}
		\item Panjang garis $\overline{XV}=\sqrt{2^2+3^2}=\sqrt{12}=2\sqrt{3}$ dan garis $\overline{XZ}=\sqrt{4^2+3^2}=5$. Dari panjang garis-garis tersebut dapat ditarik kesimpulan bahwa $\overline{VW}^2=\overline{VZ}\times \overline{XZ}=2\times 5=10$ jadi $\overline{VW}=\sqrt{10}$.
		\item Pembuktian dilakukan dengan metode SAS.
		\item Pembuktan dilakukan dengan metode ASA.
		\item Pembuktian dilakukan dengan metode ASA. 
	\end{enumerate} 
\end{document}

\documentclass[12pt,a4paper]{exam}
\usepackage[latin1]{inputenc}
\usepackage{amsmath}
\usepackage{amsfonts}
\usepackage{amssymb}
\usepackage{graphicx}
\usepackage[left=1.00cm, right=1.00cm, top=1.00cm, bottom=1.30cm]{geometry}
\author{Balya Rochmadi}
\title{SOAL FISIKA SMP TAHAP 1}
\begin{document}
	\maketitle
	\begin{enumerate}
		\item Perhatikan gambar berikut!
			\begin{center}\includegraphics[scale=0.5]{../../../Desktop/Mikrometer-skrup-Pembahasan-soal-dan-jawaban-UN-Fisika-SMA-MA-2013-3-1}\end{center}
			Sebuah permata berbentuk bola diukur menggunakan mikrometer sekrup seperti pada gambar diatas. Berapakah volumenya? ($\pi=\pi$)
		\item Dalam sebuah percobaan diketahui sebuah robot menyelam pada kedalaman 200 m pada cairan A, mendapatkan tekanan sebesar $6000 N/m^2$. Pada percobaan selanjutnya pada cairan B dengan kedalaman yang sama robot tersebut mendapatkan tekanan sebesar $8000 N/m^2$. Jika volume kedua cairan tersebut adalah sama, berapakah perbandingan berat kedua cairan tersebut? (Asumsikan gravitasi : $10 ms^{-2}$)
		\item Perhatikan gambar berikut!
	\begin{center}
		\includegraphics[scale=0.5]{../../../Desktop/Gambar+Soal+IPA+91}
	\end{center}
		Berapakah tegangan jepitnya?
		\item Perhatikan Gambar berikut!
		
		\begin{center}
			\includegraphics[scale=1]{"../../../Desktop/Gambar Soal IPA 9"}
		\end{center}
		Berapakah resistan totalnya?
		\item Dua buah partikel bermuatan $Q_1=+8C$ dan $Q_2=-12C$ terpisah pada jarak 10cm. Sedangkan $Q_3=+4C$ berada di antara $Q_1$ dan $Q_2$ serta berjarak 4 cm dari $Q_1$ dan 6 cm $Q_2$. Berapakah gaya yang diterima oleh $Q_3$?
		. 
	\end{enumerate}



\end{document}
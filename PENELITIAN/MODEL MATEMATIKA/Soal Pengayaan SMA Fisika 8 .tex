\documentclass[12pt,a4paper,draft,final,oneside,twoside,openright,openany]{article}
\usepackage[latin1]{inputenc}
\usepackage{amsmath}
\usepackage{amsfonts}
\usepackage{multicol}
\usepackage{textcomp}
\usepackage{amssymb}
\usepackage{graphicx}
\usepackage[left=3.00cm, right=3.00cm, top=3.00cm, bottom=3.00cm]{geometry}
\author{Balya Rochmadi}		
\title{Soal Pengayaan Fisika SMA 9 HOTS\\Mapel : Energi 1 \\Closed Book}
\begin{document}
	\maketitle
	\Large
	\noindent\makebox[\linewidth]{\rule{\paperwidth}{0.4pt}}
	\begin{enumerate}
		\item Dua blok, dari massa $m_1$ dan $m_2$, didorong oleh gaya F seperti yang ditunjukkan pada gambar. Koefisien gesekan antara setiap blok dan meja adalah 0,40. (a) Berapa nilai F jika balok harus memiliki percepatan 200 $cm/s^2$? Seberapa besar kekuatan m1 yang diberikan pada $m_2$? Gunakan $m_1$ = 300g dan $m_2$ = 500g. Ingat untuk bekerja dalam satuan SI.
			\begin{center}
				\includegraphics[width=0.8\linewidth]{/home/dila/Desktop/99Pcture/Selection_045}
			\end{center}
		\item Pada Gambar, koefisien gesekan kinetik antara blok A dan tabel adalah 0,20. Juga, $m_a$ = 25 kg, $m_b$ = 15 kg. Seberapa jauh kotak B akan jatuh pada 3,0 detik pertama setelah sistem dilepaskan?
					\begin{center}
						\includegraphics[width=0.8\linewidth]{/home/dila/Desktop/99Pcture/Selection_046}
					\end{center}
		\item Pada sistem yang ditunjukkan, gaya F mempercepat blok m1 ke kanan. Temukan akselerasinya dalam termin F dan koefisien gesekan pada permukaan.
		\begin{center}
			\includegraphics[width=0.8\linewidth]{/home/dila/Desktop/99Pcture/Selection_047}
		\end{center}
		\item Pada Gambar 3-19, bobot objek adalah 200 N dan 300 N. Katrol dianggap tanpa gesekan dan tanpa massa. Pulley $P-1$ memiliki poros stasioner, tetapi pulley $P_2$ bebas untuk bergerak naik dan turun. Temukan tegangan $T_1$ dan $T_2$ dan percepatan sistemnya!
			\begin{center}
				\includegraphics[width=0.5\linewidth]{/home/dila/Desktop/99Pcture/Selection_048}
			\end{center}
		\item Tiga blok dengan massa 6,0 kg, 9,0 kg, dan 10 kg terhubung seperti yang ditunjukkan pada Gambar. Koefisien gesekan antara tabel dan blok 10 kg adalah 0,20. Temukan (a) percepatan sistem dan (b) tegangan di tali di sebelah kiri dan di tali di sebelah kanan.
		\begin{center}
			\includegraphics[width=0.8\linewidth]{/home/dila/Desktop/99Pcture/Selection_049}
		\end{center}
		
		
	\end{enumerate}
\end{document}
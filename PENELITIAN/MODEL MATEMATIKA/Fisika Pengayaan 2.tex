\documentclass[12pt,a4paper,onecolumn]{article}
\usepackage[latin1]{inputenc}
\usepackage{amsmath}
\usepackage{amsfonts}
\usepackage{amssymb}
\usepackage[left=3.00cm, right=3.00cm, top=3.00cm, bottom=3.00cm]{geometry}
\author{Balya Rochmadi}
\title{\Large Soal Pengayaan Fisika SMP 2 HOTS}
\begin{document}
	\Large
	\maketitle
	\begin{enumerate}
		\item Mobil yang mula-mula bergerak dengan
		kelajuan 40 $m/s$ direm dengan perlambatan 8$ms^{-2}$ . Pada separuh jarak sebelum berhenti,
		kelajuannya (dalam $m/s$) adalah
		\item Sebuah benda dilempar keatas dengan kecepatan awal  $150 m/s$, berapakah energi kinetik benda tersebut pada detik ke 2 setelah benda tersebut dilempar? (massa benda = 1kg)
		\item Sebuah benda pada ketinggian 35 meter memiliki energi potensial sebesar 280 J, berapakah energi kinetik benda tersebut pada saat 2,5 detik setelah benda tersebut jatuh?

		
	\end{enumerate}
\end{document}
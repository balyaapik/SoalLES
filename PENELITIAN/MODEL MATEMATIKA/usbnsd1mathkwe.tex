\documentclass[14pt,a4paper]{exam}
\usepackage[latin1]{inputenc}
\usepackage{amsmath}
\usepackage{amsfonts}
\usepackage{amssymb}
\usepackage[left=3.00cm, right=3.00cm, top=3.00cm, bottom=3.00cm]{geometry}
\author{Balya Rochmadi}
\title{SIMULASI USBN SD MATEMATIKA TAHAP 1}
\begin{document}
\Large
\maketitle

\begin{enumerate}
	\item  $7827+8198-2345+1244=...$
	\begin{choices}
		\choice 14924
		\choice 14934
		\choice 24924
		\choice 14224
		\choice Tidak terdapat jawaban yang benar
	\end{choices}
	\item  $(24-20)\times -2 + (10 \div 2)=...$
	\begin{choices}
		\choice $-13$
		\choice $13$
		\choice $-3$		
		\choice $ 3$
		\choice Tidak terdapat jawaban yang benar
	\end{choices}
	\item Ibnu berenang setiap 3 hari, Ishak berenang setiap 4 hari, dan Muis berenang setiap 5 hari. Jika pada tanggal 1 Maret 2017 mereka berenang bersama, maka mereka akan berenang bersama kembali pada tanggal?.
	\begin{choices}
		\choice 29 April 2017                              
		\choice 1 Mei 2017
		\choice 30 April 2017                              
		\choice 2 Mei 2017
	\end{choices}
	\item Bu Lastri mempunyai 36 buah mangga, 60 buah jeruk, dan 72 buah salak. Buah-buah tersebut dikemas untuk dibagikan kepada tetangga. Setiap kemasan berisi buah dengan jumlah dan jenis berbeda yang sama banyak. Bu Lastri paling banyak dapat membuat??..kemasan
	
\end{enumerate}	
	
	
\end{document}
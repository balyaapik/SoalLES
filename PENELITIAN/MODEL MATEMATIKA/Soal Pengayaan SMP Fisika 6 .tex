\documentclass[12pt,a4paper,draft,final,oneside,twoside,openright,openany]{article}
\usepackage[latin1]{inputenc}
\usepackage{amsmath}
\usepackage{amsfonts}
\usepackage{multicol}
\usepackage{amssymb}
\usepackage{graphicx}
\usepackage[left=3.00cm, right=3.00cm, top=3.00cm, bottom=3.00cm]{geometry}
\author{Balya Rochmadi}		
\title{Soal Pengayaan Fisika SMP 5 HOTS\\Mapel : Hukum Kirchoff I dan II \\Closed Book}
\begin{document}
	\maketitle
	\Large
	\noindent\makebox[\linewidth]{\rule{\paperwidth}{0.4pt}}

	\paragraph{Petunjuk: }
				\begin{enumerate}
					\item Hukum Ohm : $I=\dfrac{V}{R}$
					\item Power(Dissipasi Elektrik) : $P=iE$ atau $P=VI$ atau $P=I^2R$
					\item Power(Dissipasi Elektrik) setiap waktu : $P=iEt$ atau $P=VIt$ atau $P=I^2Rt$
					\item Resistor Pararel $\dfrac{1}{R}=\Sigma \dfrac{1}{R_n}$, Resistor seri= $R_s=\Sigma R_n$
					\item Hukum Kirchoff 1, Percabangan $I_{in}=I_{out}$
					\item Hukum Kirchoff 2, Loop $\Sigma E -\Sigma IR= 0$
					\item Energi Beda Potensial : 	$W=\dfrac{Q}{V}$
						\item Rumus Coloumb	: $F_c=k\dfrac{q_1q_2}{r^2}$, $k=9\times 10^9$
						\item Energi yang dibutuhkan untuk mengalirkan arus listrik $W=q\Delta V$ dan  adalah $\Delta V=\text{beda daya/voltase}$
						\item Gaya Medan Listrik : $F_{e}=qE$ ($q=\text{muatan listrik}, E=\text{kuat medan listrik}$)
						\item Muatan diapit diantara dua muatan listrik : $\Sigma F_c=0$
						\item Kuat medan Listrik : $E=k(\dfrac{Q}{r^2})$
						\item Kuat medan Listrik diapit $E=k(\dfrac{Q_1}{r_1^2}+\dfrac{Q_2}{r_2^2})$
					
				\end{enumerate}
					\noindent\makebox[\linewidth]{\rule{\paperwidth}{0.4pt}}
				\begin{enumerate}
					\item Perhatikan Gambar Berikut!
						\begin{center}
							\includegraphics[width=0.8\linewidth]{/home/dila/Desktop/99Pcture/Selection_038}
						\end{center}
						Berapakah arus yang mengalir melalui R3?
					\item Perhatikan Gambar berikut!
						\begin{center}
							\includegraphics[width=0.8\linewidth]{/home/dila/Desktop/99Pcture/Selection_039}
						\end{center}
						Jika diketahui :
						E1 = 16 V,
						E2 = 8 V,
						E3 = 10 V,
						R1 = 12 ohm,
						R2 = 6 ohm,
						R3 = 6 ohm. Jika hambatan dalam sumber tegangan diabaikan, berapa kuat arus yang melalui R2?
					\item Berapakah arus rangkaian ini?
						\begin{center}
							\includegraphics[width=0.8\linewidth]{/home/dila/Desktop/99Pcture/Selection_040}
						\end{center}
					\item Perhatikan gambar berikut
					
					\begin{center}
						\includegraphics[width=0.8\linewidth]{/home/dila/Desktop/99Pcture/Selection_041}
					\end{center}
					Jika perbandingan antara I1 : I2 : I3 = 1 : 2 : 3, berapakah arus yang mengalir dalam masing-masing cabang tersebut?
					\item Diketahui rangkaian sebagai berikut!
					\begin{center}g
						\includegraphics[width=0.8\linewidth]{/home/dila/Desktop/99Pcture/Selection_042}
					\end{center}
					\begin{enumerate}
						\item Berapakah Arus total?
						\item Berapakah Dissipasi daya total sirkuit tersebut?
						\item Sebutkan semua Dissipasi daya dari masing-masing komponen di sirkuit tersebut!
					\end{enumerate}
						
						
				\end{enumerate}
\end{document}
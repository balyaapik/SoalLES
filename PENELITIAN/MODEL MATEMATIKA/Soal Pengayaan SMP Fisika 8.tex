\documentclass[12pt,a4paper,draft,final,oneside,twoside,openright,openany]{article}
\usepackage[latin1]{inputenc}
\usepackage{amsmath}
\usepackage{amsfonts}
\usepackage{multicol}
\usepackage{textcomp}
\usepackage{amssymb}
\usepackage{graphicx}
\usepackage[left=3.00cm, right=3.00cm, top=3.00cm, bottom=3.00cm]{geometry}
\author{Balya Rochmadi}		
\title{Soal Pengayaan Fisika SMP 7 HOTS\\Mapel : Hukum Kepler dan Mekanika Fluida \\Closed Book}
\begin{document}
	\maketitle
	\Large
	\noindent\makebox[\linewidth]{\rule{\paperwidth}{0.4pt}}
	\paragraph{Petunjuk:}
	\begin{enumerate}
		\item Hukum Kepler :
			\begin{enumerate}
				\item Hukum 1 : Semua planet bergerak pada lintasan elips mengitari Matahari dengan Matahari berada di salah satu fokus elips.
				\item Hukum 2 : Suatu garis khayal yang menghubungkan Matahari dan planet menyapu luas juring yang sama dalam selang waktu yang sama.
				\item Hukum 3 : Perbandingan kuadrat period terhadap pangkat tiga dari setengah sumbu panjang elips adalah sama untuk semua planet.\\
				\begin{center}
				$\dfrac{T_{B}^{2}}{R_{B}^{3}} =\dfrac{T_{J}^{2}}{R_{J}^{3}}$
				\end{center}
				\item Mencari Massa Planet/Objek Angkasa : \\
				\begin{center}
					 $M=\dfrac{4\pi^{2}R^{3}}{GT^{2}}$ 
				\end{center}
				\item Tekanan\\
				 $P={\dfrac {F}{A}}$
				
				Keterangan:\\
				
				P: Tekanan ($N/m^2 $atau $dn/cm^2$)\\
				F: Gaya (N atau dn)\\
%				A: Luas alas/penampang $(m^2 atau cm^2)$\\
				\item Tekanan Pipa U\\
					$\rho_1h_1=\rho_2h_2$
				
				\item Tekanan Hidrostatis
					\begin{center}
						\includegraphics[width=0.8\linewidth]{/home/dila/Desktop/99Pcture/Selection_050}
					\end{center}
				\item Tekanan Gas Mutlak atau Gauge dan kedalaman zat cair
				\begin{center}
					\includegraphics[width=1\linewidth]{/home/dila/Desktop/99Pcture/Selection_052}
				\end{center}
				\item Hukum Pascal 
				\begin{center}
					\includegraphics[width=1\linewidth]{/home/dila/Desktop/99Pcture/Selection_054}\\
					\includegraphics[width=1\linewidth]{/home/dila/Desktop/99Pcture/Selection_055}
				\end{center}
				\item Hukum Archimedes
				\begin{center}
					\includegraphics[width=1\linewidth]{/home/dila/Desktop/99Pcture/Selection_056}
				\end{center}
				\item Daya Apung Archimedes
				\begin{center}
					\includegraphics[width=1\linewidth]{/home/dila/Desktop/99Pcture/Selection_057}
				\end{center}
			\end{enumerate}
	\end{enumerate}
		\noindent\makebox[\linewidth]{\rule{\paperwidth}{0.4pt}}
		
		
	\begin{enumerate}
		\item Periode revolusi planet Venus mengelilingi 
		matahari adalah 0.62 tahun. Berapakah jarak 
		planet Venus terhadap matahari? Nyatakan 
		jawaban dalam SA (Satuan Astronomi)!       
		(1 SA = jarak bumi ke matahari) 
		\item Periode revolusi planet Venus mengelilingi 
		matahari adalah 0.62 tahun. Berapakah jarak 
		planet Venus terhadap matahari? Nyatakan 
		jawaban dalam SA (Satuan Astronomi)!       
		(1 SA = jarak bumi ke matahari) 
		\item Asumsikan jarak bulan ke Bumi adalah $3,84\times 10^8$ meter dan periode bulan mengelilingi Bumi adalah $2,5\times 10^6$ detik. Hitung massa Bumi!
		\item Perbandingan jari-jari orbit planet A dan B saat mengorbit Matahari adalah 1 : 5. Perbandingan periode revolusi kedua planet adalah?
		\item Periode revolusi Bumi adalah dan jarak Bumi ke Matahari adalah 1 SA. Periode revolusi planet Jupiter adalah sekitar 12 tahun. Jarak Jupiter ke Matahari mendekati nilai?
		\item Luas penampang penghisap yang kecil dan yang besar dari suatu pompa hidrolik adalah 6 $cm^2$ dan 20 $cm^2$. Jika pada penghisap yang kecil bekerja gaya 50 N, berapakah besar gaya timbul pada penghisap yang besar ?
		\item Apabila sebuah kapal selam menyelam sedalam 60 m, berapa besar tekanan yang
		dialami kapal selam tersebut. (massa jenis air laut = 1,03 g/$cm^3$).
		\item Pompa hidrolik mempunyai penghisap dengan luas penampang 15 $cm^2$ dan 3 $dm^2$. Jika pada penghisap yang kecil diberi beban 400 N. Berapa besar gaya pada penghisap yang besar agar terjadi keseimbangan ?
		\item Gaya besarnya 5 N pada penghisap yang kecil dari suatu pompa hidrolik dapat mengangkat beban beratnya 600 N yang terdapat pada penghisap yang besar. Jika penghisap yang kecil berpenampang 400 $cm^2$, berapakah luas penampang yang besar ?
		\item Suatu kempa hidrolik dapat mengangkat 1 ton mobil, jika diameter penghisap besar 50 cm, diameter penghisap kecil 10 cm. Tentukan gaya yang harus dikerjakan pada penghisap kecil.
		\item Sebuah kempa hidrolik mempunyai torak yang berdiameter 20 cm dan 2 m untuk mengangkat mobil. Pada torak kecil dilakukan gaya sebesar 100 N, sehingga torak besar naik setinggi 1 cm. Tentukan massa mobil dan berapa m turunnya torak kecil tersebut
		\item Suatu bejana berbentuk pipa U mula-mula diisi dengan air raksa yang massa jenisnya 13,6 g/$cm^3$, kemudian kaki kanan dituangkan 14 cm air lalu di atas air ini dituangkan minyak yang massa jenisnya 0,8 g/$cm^3$, ternyata dalam keadaan setimbang selisih tinggi permukaan air raksa dalam kedua kaki 2 cm. Hitung berapa cm tinggi lajur minyak pada kaki kanan.
		
		\item Dalam pipa U terdapat Hg (massa jenisnya 13,6 g/$cm^3$ ). Pada kaki kiri dituangkan air setinggi 20 cm kemudian minyak (massa jenisnya 0,9 g/$cm^3$ ) tingginya 8 cm. Pada kaki kanan ditambahkan alkohol (massa jenisnya 0,8 g/$cm^3$ ) sehingga permukaan minyak dan permukaan alkohol sebidang. Berapa beda tinggi Hg pada kedua kaki pipa ?
		\item Sepotong logam beratnya di udara 4 N, tetapi beratnya tinggal 2,5 N bila dibenamkan dalam
		zat cair. Berapakah gaya tekan ke atas yang diderita benda?
		\item Sebuah silinder aluminium pejal mempunyai massa jenis 2700 kg/$m^3$ , massanya 77 gram.
		Berat aluminium itu tinggal 450 N bila dibenamkan dalam minyak tanah Berapa massa jenis
		minyak tanah?
		\item Sebuah benda terapung di atas minyak yang mempunyai massa jenis 0,9 g/$cm^3$ . Tinggi benda
		tersebut adalah 20 cm, sedangkan tinggi benda yang tidak tercelup adalah 2 cm. berapa
		massa jenis benda tersebut?
		\item Sepotong gabus terapung di atas air dengan 0,25 bagian terendam. Jika berat jenis air adalah 1 gr/$cm^3$, hitunglah berat jenis gabus!
		\item Sebuah patung berongga mempunyai berat 210 N dan jika ditimbang di dalam air beratnya 190 N. Patung tersebut terbuat dari logam (massa jenisnya 21 g/$cm^3$). Tentukan volume rongga patung tersebut. (g = 10 m/det2)!
		\item Sebatang emas (massa jenisnya 19,3 g/$cm^3$) dicurigai mempunyai rongga. Beratnya di udara 0,3825 N dan di air 0,3622 N. Berapa besar rongga tersebut ?
		\item 50 gram gabus (massa jenisnya 0,25 g/$cm^3$) diikatkan pada timbal sehingga gabungan benda melayang di dalam air. Berapa berat timbal ( massa jenisnya 11,3 g/cm3)?
		\item Sebongkah es (massa jenisnya 0,9 g/$cm^3$ ) terapung pada air laut (massa jenisnya 1,03 g/$cm^3$ ).Jika es yang timbul di permukaan air laut 7,8 $dm^3$ . Hitunglah volume es!
		
	\end{enumerate}
\end{document}
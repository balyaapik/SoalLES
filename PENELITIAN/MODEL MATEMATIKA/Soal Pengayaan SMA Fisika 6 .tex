\documentclass[12pt,a4paper,draft,final,oneside,twoside,openright,openany]{article}
\usepackage[latin1]{inputenc}
\usepackage{amsmath}
\usepackage{amsfonts}
\usepackage{multicol}
\usepackage{amssymb}
\usepackage{graphicx}
\usepackage[left=3.00cm, right=3.00cm, top=3.00cm, bottom=3.00cm]{geometry}
\author{Balya Rochmadi}		
\title{Soal Pengayaan Fisika SMA 6 HOTS\\Mapel : Fisika,Slider dan Tegangan tali\\ Closed Book, Individu}
\begin{document}
	\maketitle
	\noindent\makebox[\linewidth]{\rule{\paperwidth}{0.4pt}}
	\paragraph{\Large Petunjuk}
	\begin{enumerate}
		\item Equilibrium jika $\Sigma F=0$
		\item Gaya x : $T-f=ma_x$
		\item Gaya y : $T+F_g=ma_y$
		\item $f_s=\mu_s N$ atau $f_s=\mu_s T$
		\item $f_k=\mu_k N $ atau $f_s=\mu_k T$
	\end{enumerate}
	\noindent\makebox[\linewidth]{\rule{\paperwidth}{0.4pt}}
	\begin{enumerate}
		\item Jika $M_a, M_b, M_c$ berturut-turut 6,8, dan 10 kilogram. Berapakah tegangan tali pada bandul C? Anggap mekanika tersebut licin tanpa gaya gesek!
	\begin{center}
		\includegraphics[width=0.5\linewidth]{/home/dila/Desktop/99Pcture/Selection_029}
	\end{center}
		\item Dua buah balok $m_1,m_2$ masing masing memiliki massa 1kg dan 2kg, jika balok yang menggantung didorong dengan gaya 6 Newton yang mana balok tersebut memiliki percepatan kebawah sebesar 5,5 $m/s^2$, Berapakah :
		\begin{enumerate}
			\item Tegangan tali M2?
			\item Berapakah besar sudut $\beta$
		\end{enumerate}
			\begin{center}
				\includegraphics[width=0.4\linewidth]{/home/dila/Desktop/99Pcture/Selection_030}
			\end{center}
		\item Balok 1 dan 2 masing masing bermassa 3kg dan 2kg, jika sudut 1 dan 2 memiiki besaran $30^o$ dan $20^o$, Berapakah tegangan talinya?
					\begin{center}
						\includegraphics[width=0.4\linewidth]{/home/dila/Desktop/99Pcture/Selection_031}
					\end{center}
		\item Jika $m_2=1kg$ ditarik dengan gaya $\vec{F}$ terikat oleh $m_1=2kg$ dengan sudut elevasi $30^o$ berapakah tegangan talinya?
		\begin{center}
			\includegraphics[width=0.4\linewidth]{/home/dila/Desktop/99Pcture/Selection_032}
		\end{center}
		\item Box 1 dan 2 memiliki massa 1 dan 2 kilogram saling terikat pada slider. Besaran gaya horizontal adalah $\vec{F}=2,3 N$ berapakah tegangan talinya? (sudut depresi= 30 derajat)
			\begin{center}
				\includegraphics[width=0.4\linewidth]{/home/dila/Desktop/99Pcture/Selection_033}
			\end{center}
	\end{enumerate}
\end{document}
\paragraph*{}
	Perbandingan lebih dari 2 pembanding didefinisikan sebagai,\\
	\begin{center}
		$a_1:a_2:a_3:a_4:...:a_n$
	\end{center}
	Contoh dari perbandingan lebih dari 2 adalah, $A:B:C=1:2:3$. Perbandingan lebih dari 2 dapat dihitung berdasarkan beberapa perbandingan dua pembanding yang digabung.
	\\
	Karakteristik perbandingan lebih dari dua:
	\begin{enumerate}
		\item Didapatkan dari perbandingan dua angka
		\item Memiliki cara pengoperasian sama persis dengan perbandingan dua angka
	\end{enumerate}
	
	\subsection{Penggabungan Dua Perbandingan atau lebih menjadi Perbandingan lebih dari dua}
	\begin{enumerate}
		\item Perbandingan dengan dua perbandingan yang akan digabung dengan cara.
		\begin{center}
			\begin{tabular}{ccccc}
				
				A&:&$B_1$& & \\
				& &$B_2$&:&C\\
				\hline
				$\frac{KPK(B_1,B_2)}{B_1}\times A$&: &$KPK(B_1,B_2)$&:&$\frac{KPK(B_1,B_2)}{B_2}\times C$
			\end{tabular}
		\end{center}
		\item Perbandingan dengan perbandingan lebih dari dua yang akan digabung.
	\end{enumerate}
	Jika terdapat $A:B_1=1:3$, dan $B_2:C=4:5$ berapakah $A:B:C$?
	\begin{center}
		\begin{tabular}{ccccc}
			
			1&:&$3$& & \\
			& &$4$&:&5\\
			\hline
			4&: &12&:&15
		\end{tabular}
	\end{center}
	Jadi A:B:C=4:12:15
	
	
	
	
	
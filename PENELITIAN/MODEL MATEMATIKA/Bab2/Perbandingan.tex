\paragraph*{}
	Perbandingkan adalah sebuah cara untuk mendiskripsikan beberapa nilai yang berhubungan sehingga tercipta sebuah pecahan yang merepresentasikan nilai-nilai tersebut. Perlu diingat bahwa perbandingan adalah pecahan.
	\par
	Jika diketahui nilai A dan B maka perbandingan nilai A dan B adalah bentuk paling sederhana dari $\dfrac{A}{B}$ atau bisa ditulis sebagai $A:B$ Contoh :\\
	Jika terdapat A=250 dan B=150 maka perbandingan A dan B adalah $\dfrac{250}{150}=\dfrac{5}{3}$ jadi perbandingannya ditulis sebagai $5:3$
	\subsection{Perhitungan Perbandingan}
		\begin{enumerate}
			\item Salah satu nilai tidak diketahui maka untuk mengetahui nilai tersebut dapat dihitung dengan menggunakan
			\begin{center}
				$\dfrac{\text{perbandingan nilai tidak diketahui}}{\text{perbandingan nilai diketahui}}\times \text{nilai diketahui}$
			\end{center} 
			Jadi dalam contoh: A:B=3:7misal nilai A tidak diketahui dan nilai B diketahui bernilai 49 berapakah nilai A? \\
			\textbf{Jawab :} Jika B diketahui maka nilai A adalah $\dfrac{3}{7}\times 49=21$
			\item Jika terjadi Nilai perbandingan yang diketahui jumlah atau selisihnya semisal A : B, dan $A\mypm B$ diketahui maka,
			\begin{center}
				$\dfrac{\text{perbandingan nilai tidak diketahui}}{A\mypm B}\times \text{nilai diketahui}$
							\end{center}
			\textbf{Contoh 1} : Jika $A:B=2:3$ berapakah A jika diketahui selisih A dan B adalah 20?. \\
			\begin{center}
				$\dfrac{A}{B-A} \times 20=\dfrac{2}{3-2}\times 20=\dfrac{2}{1}\times 20=40$
			\end{center}
			\textbf{Contoh 2} : Jika $A:B=3:4$ dan jumlah A dan B adalah 35, berapakah selisih A dan B? 
			\begin{center}
				$\dfrac{B-A}{A+B} \times 35=\dfrac{4-3}{3+4}\times 35=\dfrac{1}{7}\times 35=5$
			\end{center}
			\item Jika diketahui perbandingan A dan B, dan diketahui AB serta terdapat nilai yang tidak diketahui maka, 
			\begin{center}
				$\dfrac{\text{nilai yang tidak diketahui}}{AB}\times \text{nilai AB}$
			\end{center}
			\textbf{Contoh :} A:B adalah 3:2 dan AB adalah 150, berapakah A?
			\begin{center}
				$\dfrac{3}{2\times 3}\times 150=\dfrac{3}{6}\times 150=75$
			\end{center}
		
		
		
		
		
		
			\end{enumerate}
			
			
			
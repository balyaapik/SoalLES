\paragraph*{}
	Skala adalah perbandingan yang menunjukkan panjang dalam realita sebenarnya dan panjang dalam peta yang digambarkan dalam bentuk pecahan perbandingan.\par
	Karakteristik Skala adalah sebagai berikut:
	\begin{enumerate}
		\item Digunakan dalam navigasi dan peta
		\item Hanya menghitung panjang, bukan dimensi geometri yang lain.
		\item Biasanya dinyatakan dalam centimeter, kecuali ditetapkan satuan lain
		\item Merupakan sebuah perbandingan.
		\item Merupakan perbandingan jarak pada peta dan jarak sebenarnya
	\end{enumerate}
	\subsection{Perhitungan Skala}
	Skala dihitung jika diketahui hal-hal sebagai berikut:
	\begin{enumerate}
		\item Mencari Skala, Diketahui : Panjang dalam Peta, Panjang Sebenarnya
		\begin{center}
			$\text{Skala}=\dfrac{\text{Panjang dalam Peta}}{\text{Panjang Sebenarnnya}}$
		\end{center}
		
		\textbf{Contoh :} Jika seseorang ingin membuat sebuah denah dan jarak A dan B pada denah tersebut adalah 5 cm, tapi jarak sebenarnya adalah 10 m.
		\begin{center}
			$\text{Skala}=\dfrac{5 cm}{1000 cm}=\dfrac{1}{200}$
		\end{center}
		Jadi Skalanya, $1:200$		
		\item Mencari Jarak sebenarnya, Diketahui : Panjang dalam peta dan Skala
		\begin{center}
			$\text{Jarak Sebenarnya}=\text{Jarak Pada Peta} \div \text{Skala}$
		\end{center}
		\textbf{Contoh:} Jika jarak pada peta adalah 2 cm, sedangkan skalanya adalah $1: 500$. Berapakah Jarak Sebenarnya?
		\begin{center}
			$\text{Jarak Sebenarnya}=2 \div \dfrac{1}{500}=2 \times 500=1000$cm
		\end{center}
		
		\item Mencari Jarak dalam Peta, Diketahui : Panjang dalam Sebenarnya dan Skala
		\begin{center}
			$\text{Jarak pada Peta}= \text{Panjang Sebenarnya} \times \text{Skala}$
		\end{center}
		Contoh : Jika diketahui jarak A, B adalah 40m dan Skala pada peta adalah 1:5000, berapakah jarak pada peta?
		\begin{center}
			$\text{Jarak pada Peta}= 40000 cm \times \dfrac{1}{5000} = 8 cm$
		\end{center}
	\end{enumerate}
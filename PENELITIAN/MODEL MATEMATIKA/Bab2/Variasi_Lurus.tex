\paragraph{}
	Variasi lurus didefinisikan bahwa ketika sebuah nilai naik, maka nilai lain yang berhubungan dengan nilai tersebut naik secara proporsional.Misal, jika harga minyak naik, maka harga roti juga naik, berarti variasinya adalah berbanding lurus.
	\par
	Secara khusus variasi berbanding lurus memiliki persamaan sebagai berikut dengan $a=kc$ dan $k$ adalah konstanta. 
	\begin{center}
		
	\end{center}
	Secara umum variasi berbanding lurus didefinisikan sebagai,
	\begin{center}
		$a_1a_2a_3a_4...a_n=c$ misal, Jika $a_1$ naik dan semua variabel selain itu dianggap tetap, maka c juga ikut naik atau sebaliknya
	\end{center}
	Contoh : Sebuah pabrik memiliki pekerja sebanyak 50 orang saat ini dan menghasilkan laba sebanyak 100 \textdollar per hari. Jika pada tahun sebelumnya pabrik tersebut memiliki pekerja sebanyak 25 dengan laba 50 \textdollar. Berapakah laba yang akan diperoleh jika pada tahun berikutnya perusahaan memperkerjakan 100 pekerja?
	\par
	Jawab: Variasi yang didapatkan adalah variasi berbanding lurus. Jadi perhitungannya adalah sebagai berikut, $100=50k$ jadi $k=2$ atau dapat dihitung dengan $50=100k$ jadi $k=0.25$. Karena yang ditanyakan adalah laba tahun berikutnya, maka yang wajib digunakan adalah cara pertama dengan $k=2$. Jadi jika mempekerjakan 100 pekerja akan memperoleh laba sebanyak $2\cdot 100=200 \textdollar$
	
	
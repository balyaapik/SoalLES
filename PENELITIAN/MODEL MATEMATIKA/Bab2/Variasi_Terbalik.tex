\paragraph*{}
	Variasi berbanding terbalik adalah variasi yang menunjukkan bahwa ketika sebuah nilai naik maka secara proporsional nilai yang lain turun.
	\par
	Contoh : Peristiwa naiknya permukaan air laut mengakibatkan produksi ikan menurun
	\par
	Secara khusus variasi berbanding lurus memiliki persamaan sebagai berikut,
	\begin{center}
		$y=\dfrac{k}{x}$ dengan $k$ adalah konstan,
	\end{center}
	Secara umum variasi berbanding terbalik memiliki persamaan sebagai berikut, 
	\begin{center}
		$y_1y_2y_3...y_n=\dfrac{k_1k_2...k_n}{x_1x_2x_3...x_n}$
	\end{center}
	\textbf{Contoh :} Proyek perbaikan jalan selesai selama 30 hari dengan pekerja sebanyak 15 orang. Setelah 6 hari pelaksanaan, proyek tersebut dihentikan selama 4 hari karena suatu hal. Jika kemampuan bekerja setiap orang sama dan agar proyek dapat selesai tepat waktu, pekerja tambahan yang diperlukan adalah?
	
	\textbf{Jawab:} Perbandingan tersebut adalah perbandingan berbanding terbalik, karena semakin sedikit waktu penyelesaian pengerjaan, semakin banyak pekerja yang harus mengerjakan.
	\begin{enumerate}
		\item Pada enam hari pengerjaan yaitu  $H-24$, pekerjaan dihentikan, pekerja sampai dengan hari itu adalah 15 orang.
		\item Empat hari kemudian, pekerjaan dilakukan lagi yaitu di hari $H-20$
		\item Jadi persamaannya adalah $15\times 24=20\times x$, $x=18$.
		\item Jadi pekerja tambahan yang diperlukan adalah $18-15=3$ orang. 
	\end{enumerate}
	
	
		
	
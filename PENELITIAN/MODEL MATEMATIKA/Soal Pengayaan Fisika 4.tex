\documentclass[12pt,a4paper,draft,final,oneside,twoside,openright,openany]{article}
\usepackage[latin1]{inputenc}
\usepackage{amsmath}
\usepackage{amsfonts}
\usepackage{amssymb}
\usepackage{graphicx}
\usepackage[left=3.00cm, right=3.00cm, top=3.00cm, bottom=3.00cm]{geometry}
\author{Balya Rochmadi}		
\title{Soal Pengayaan Fisika SMA 3 HOTS}
\begin{document}
	\maketitle
	\noindent\makebox[\linewidth]{\rule{\paperwidth}{0.4pt}}
	\paragraph{Petunjuk}
	\begin{enumerate}
		\item Hukum Newton 
		\begin{enumerate}
			\item Inersia $\Sigma F=0$
			\item $\Sigma F=m\vec{a}$
			\item $\Sigma F_{aksi}=-\Sigma F_{reaksi}$
		\end{enumerate}
		\item Komponen Vektor 
		\begin{enumerate}
			\item Resultan Gaya $F=\sqrt{F_x^2+F_y^2}$
			\item Komponen Vektor x dan y : $F_x=F \cos{\theta}$, $F_y=F \sin{\theta}$
			\item Komponen akselerasi x dan y: $a_x=\dfrac{F_x}{m}$ dan $a_y=\dfrac{F_y}{m}$
			\item Resultan akselerasi, $a=\sqrt{a_x^2+a_y^2}$
			\item arah akselerasi, $\tan^{-1}(\dfrac{a_y}{a_x})=\theta$
		\end{enumerate}
		\item Gaya Gravitasi Newton
		\begin{enumerate}
			\item $F_g=G\dfrac{m_1m_2}{r^2}$, konstanta $G=6,77 \times 10^{-11}Nm^2/kg$
			\item Berat, $w=mg$
			\item Berat dalam massa Bumi $w=G\dfrac{M_em}{r^2}$
			\item Berat dari radius pusat bumi\\
		 	$w=G\dfrac{M_em}{r^2}$\\
			$m.g=G\dfrac{M_em}{r^2}$\\
			$g=G\dfrac{M_e}{r^2}$
			\item Konstanta-konstanta:\\
			\begin{enumerate}
				\item radius bumi :$6,38\times10^6 m$
				\item satuan astronomi : $1,496\times10^11 m$
			\end{enumerate}
		\end{enumerate}
		\item Tegangan Tali
		\begin{enumerate}
			\item Tarikan gravitasi $w=F_g=m.g$
			\item Benda Equilibrium $\Sigma F=T-F_g=0$
			\item Benda tidak equilibrium $\Sigma F\neq 0$
		\end{enumerate}
	\end{enumerate}
	\noindent\makebox[\linewidth]{\rule{\paperwidth}{0.4pt}}
	\begin{enumerate}
		\item Temukan semua tegangan tali yang berlaku pada seorang pencuri kucing dengan bobot $600 N$ pada gambar berikut ini!		
		\begin{center}
		\includegraphics[width=0.4\linewidth]{../../Desktop/99Pcture/Selection_012}
		\end{center}
		\item Dua bua beban ditempatkan dengan massa masing-masing $3,5kg$, berapakah tegangan tali bagian atas dan bagian bawah dari benda tersebut jika percepatan awal adalah 1,60 $m/s^2$ berapkah Tegangan tali ($T_1, T_2$) nya?, Jika tegangan tali bagian atas diberikan gaya 85N berapakah percepatan maksimal sebelum tali tersebut putus?		
		\begin{center}
		\includegraphics[width=0.15\linewidth]{../../Desktop/99Pcture/Selection_013}
		\end{center}
		\item Sebuah truk yang membawa benda kotak bermassa 62kg yang diikat dengan tali, tali tersebut dapat menahan gaya hingga 65 Newton. Berapakah akselerasi maksimal dari truk tersebut sebelum tali itu putus?
		\begin{center}
		\includegraphics[width=0.7\linewidth]{../../Desktop/99Pcture/Selection_014}
		\end{center}
		\item Berapakah berat seseorang yang berada pada jarak 2000 meter dari atas permukaan bumi?
		\item 150 N digantungkan seperti pada gambar berikut! temukan tegangan tali pada gambar berikut!
				\begin{center}
					\includegraphics[width=0.4\linewidth]{../../Desktop/99Pcture/Selection_015}
				\end{center}
	\end{enumerate}
	

\end{document}
\documentclass[12pt,a4paper,draft,final,oneside,twoside,openright,openany]{article}
\usepackage[latin1]{inputenc}
\usepackage{amsmath}
\usepackage{multicol}
\usepackage{amsfonts}
\usepackage{amssymb}
\usepackage{pdfpages}
\usepackage{graphicx}
\usepackage[left=3.00cm, right=3.00cm, top=3.00cm, bottom=3.00cm]{geometry}
\author{Balya Rochmadi}		
\title{Soal Pengayaan SD 1 HOTS\\ Mapel : Matematika\\ Sub: Aritmatika Sosial, Bangun 2D \\Codename: Pythagoras\\ \small waktu pengerjaan : 1 jam.}
\begin{document}
	\maketitle
	\Large
	\begin{enumerate}
	\item Selembar seng berbentuk persegipanjang berukuran $50cm \times 40 cm$.Seng itu dibuat tutup kaleng berbentuk lingkaran dengan jari-jari 20 cm. Luas seng yang tidak digunakan adalah...
	\item Luas daerah bangun pada gambar adalah:
			\begin{center}
				\includegraphics[width=0.4\linewidth]{/home/dila/Desktop/99Pcture/3}
			\end{center}
	\item Seorang pedagang membeli 3 kodi pakaian dengan harga Rp 600.000,- perkodi. Pakaian tersebut ia jual kembali dengan harga Rp 400.000,- perlusin. Dalam waktu dua hari pakaian tersebut sudah habis. Keuntungan yang diperoleh pedagang tersebut adalah
	\item Seorang pedagang membeli sebuah TV dengan harga Rp 2.000.000,-. Jika TV tersebut ia jual kembali dengan harga Rp 2.400.000,- maka persentase keuntungan yang diperoleh pedagang tersebut adalah ?
	\item  Seorang pedagang membeli 1 rim kertas A4 dengan harga Rp 50.000,-. Kertas tersebut dijual secara ecer per 5 lembar. Agar pedagang tersebut untung Rp 20.000,- dari hasil penjualan kertas itu, maka harga ecer per 5 lembar kertas adalah ?
	\item Seorang pedagang membeli 20 kg salak seharga Rp 140.000,-. Setengahnya ia jual kembali dengan harga Rp 10.000,-/kg dan setengahnya lagi ia jual dengan harga Rp 6.000,- karena sudah mulai rusak. Jika seluruh salak terjual habis, maka keuntungan yang diperoleh pedagang adalah 
	\item Ibu membeli 1 lusin pensil dengan harga Rp 20.000,-. Jika pensil tersebut dijual lagi oleh ibu dengan harga Rp 2.000,- per batang, maka persentase untung yang diperoleh ibu dari penjualan seluruh pensil adalah 
	\item Koperasi sekolah membeli suatu barang dengan harga 
	 Rp 500.000 Apabila koperasi sekolah itu menginginkan untung 20$\%$, maka barang itu harus dijual dengan harga?
	\item Pak Budi membeli mobil dengan harga 125.000.000. Mobil tersebut kemudian dijual kembali dengan harga Rp120.000.000,00. Tentukan:
	a) kerugian yang dialami Pak Budi
	b) persentase kerugian
	\item Seorang pedagang memiliki barang yang dijual dengan harga Rp126.000,00. Jika dari harga tersebut pedagang mendapatkan keuntungan 5$\%$, tentukan harga pembelian barang! 
	\end{enumerate}
		\includepdf[pages=-]{/home/dila/Desktop/99Pcture/PSD1}
\end{document}
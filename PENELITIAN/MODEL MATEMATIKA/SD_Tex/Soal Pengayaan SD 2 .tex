\documentclass[12pt,a4paper,draft,final,oneside,twoside,openright,openany]{article}
\usepackage[latin1]{inputenc}
\usepackage{amsmath}
\usepackage{multicol}
\usepackage{amsfonts}
\usepackage{amssymb}
\usepackage{pdfpages}
\usepackage{graphicx}
\usepackage[left=3.00cm, right=3.00cm, top=3.00cm, bottom=3.00cm]{geometry}
\author{Balya Rochmadi}		
\title{Soal Pengayaan SD 4 HOTS\\ Mapel : Matematika\\ Sub: Teori Bilangan\\ \small waktu pengerjaan : 1 jam.}
\begin{document}
	\maketitle
	\Large
	\noindent\makebox[\linewidth]{\rule{\paperwidth}{0.4pt}}
	\begin{enumerate}
	
	
	\item Carilah bilangan terkecil yang dapat dibagi oleh semua bilangan dari 1 sampai dengan 12.
	\item Pada persamaan $a\times b = c$, jika a ditambah dengan 24 dan b tetap tidak diubah, maka c
	bertambah dengan 120. Jika b ditambah 24 dan a tetap tidak diubah, maka c bertambah
	dengan 288. Carilah nilai c mula-mula.
	\item Nita merayakan ulang tahunnya pada bulan april 2005. Pada hari itu usianya sama
	dengan jumlah angka-angka pada tahun lahirnya. Pada tahun berapa nita lahir?
	\item Nomor polisi mobil-mobil di suatu negara selalu berupa bilangan 4-angka.
	Selain itu, jumlah keempat angka pada setiap nomor juga harus habis dibagi 5. Nomor
	polisi terbesar yang dibolehkan di negara itu adalah
	\item Suatu tim dokter ahli bedah dapat melakukan operasi pada pasiennya
	dengan keberhasilan 65$\%$. Bila operasi pertama gagal, tim dokter tersebut melakukan
	operasi kedua tetapi dengan keberhasilan $20\%$. Setelah operasi kedua, maka tidak ada
	pasien yang dapat diselamatkan lagi. Berapakah banyak pasien yang dapat diselaatkan
	dari setiap 100 orang pasien yang dioperasi?
	\item Bilangan 15 dapat dinyatakan sebagai jumlah dua atau lebih bilangan asli
	berurutan dalam tiga cara, yaitu:\\
	15 = 1 + 2 + 3 + 4 + 5\\
	15 = 4 + 5 + 6\\
	15 = 7 + 8\\
	\begin{enumerate}
		\item Nyatakan bilangan 18 sebagai jumlah dua atau lebih bilangan asli berurutan. Tuliskan
		dengan sebanyak-banyaknya cara.
		\item Nyatakan bilangan 210 sebagai jumlah dua atau lebih bilangan asli berurutan.
		Tuliskan dengan sebanyak-banyaknya cara.
		\item  Tentukan sebuah bilangan di antara 10 dan 100 yang tidak dapat dituliskan sebagai
		jumlah dua atau lebih bilangan asli berurutan
	\end{enumerate}
	\item Banyaknya  kelereng  Yudi  dua  kali  banyaknya  kelereng  Satrio.  Banyaknya  kelereng  Adi
	tiga  kali  banyaknya  kelereng  Yudi.  Jumlah  seluruh  kelereng  mereka  ada  846  butir.
	Banyaknya  kelereng  Adi  adalah ?
	\item Harga  9  bungkus  cokelat  adalah  Rp72.000,00.  Harga  3  bungkus  permen  adalah
	Rp12.000,00.  Ani  ingin  membeli  4  bungkus  cokelat  dan  2  bungkus  permen.  Uang  yang
	harus  dibayar  Ani  adalah ?
	\item Seorang pedagang menjual 254 butir telur pada hari Senin. Pada hari Selasa, ia menjual
	telur dua kali banyaknya telur yang dijual pada hari Senin. Jumlah telur yang tersisa ada
	150  butir.  Jika  ia  ingin  menjual  habis  telur-telur  tersebut  hanya  dalam  dua  hari  dengan
	jumlah  yang  sama,  banyaknya  telur  yang  harus  ia  jual  setiap  harinya  adalah ?
	\item Lintang dan Ikal memiliki jumlah permen yang sama. Setelah Lintang memakan 16 permen
	miliknya,  banyaknya  permen  Ikal  3  kali  banyaknya  permen  Lintang.  Banyaknya  permen
	Ikal  adalah  
	\item Hari ini hari Jum?at, 200 hari yang akan datang adalah hari ?
\end{enumerate}
\end{document}
\documentclass[12pt,a4paper,draft,final,oneside,twoside,openright,openany]{article}
\usepackage[latin1]{inputenc}
\usepackage{amsmath}
\usepackage{amsfonts}
\usepackage{multicol}
\usepackage{amssymb}
\usepackage{graphicx}
\usepackage[left=3.00cm, right=3.00cm, top=3.00cm, bottom=3.00cm]{geometry}
\author{Balya Rochmadi}		
\title{Soal Pengayaan Matematika SD \\CC:291118**\\ sesuai kisi-kisi \\
	Closed Book}
\begin{document}
	\maketitle
	\Large
		\noindent\makebox[\linewidth]{\rule{\paperwidth}{0.4pt}}
	\begin{enumerate}
		\item Perbandingan umur Ramzi dan Rahman 10 tahun yang lalu adalah $1:3$. Perbandingan umur mereka 5 tahun yang akan  datang adalah $2:3$ Berapakah perbandingan umur mereka sekarang?
		\item $5\% \text{ dari } \dfrac{20}{15}$ adalah?
		\item Ada berapa bilangan prima antara 1 sampai 50?
		\item $4x-37=63$ berapakah x?
		\item Banyaknya bilangan antara 0 sampai 100 yang menggunakan angka 1 untuk menuliskannya, adalah...
		\item Perbandingan murid laki-iaki dengan murid perempuan di sebuah kelas yang jumlah muridnya 32 adalah 3 : 5. Berapa selisih murid perempuan dengan laki-laki
		\item Kelas A terdiri dari 30 siswa. Lima belas anak berusia 11 tahun,7 anak berusia 12 tahun dan 8 anak berusia 13 tahun. Berapa rata-rata usia mereka?
		\item Kecepatan rata-rata sebuah mobil adalah $1\dfrac{4}{5}$
		kali kecepatan rata-rata sebuah bis. Sebuah traktor berjalan 575 km selama 23 jam. Berapakah jarak yang ditempuh sebuah mobil
		dalam waktu 4 jam,jika kecepatan bis dua kali kecepatan traktor?
		\item Cuaca dikota jombang dan Surabaya akhir-akhir ini berubah-ubah, terkadang hujan deras
		terkadang cerah. Bila deny mengendarai mobil dengan cuaca hujan deras, deny hanya
		dapat memacu mobilnya dengan kecepatan 45km/jam. Jika cuacah cerah kecepatan mobil
		adalah 60 km/jam. Berapa lama deny menempuh perjalanan dari kota jombang menuju
		Surabaya, bila jaraknya adalah 270 km dan bila 2/5 dari perjalanannya turun hujan deras.
		\item Bryan dan Dika melakukan perjalanan dari kota A ke kota B yang berjarak 9 km.
		Pertama-tama Bryan menggunakan sepeda dengan kecepaatan 8 km/jam dan
		meninggalkan sepeda tersebut disuatu tempat. Kemudian dia berjalan kaki dengan
		kecepatan 5 km/jam untuk sampai di kota B. Disisi lain Dika berjalan kaki dengan
		kecepatan 4 km/jam, kemudian Dika menggunakan sepeda yang ditinggalkan oleh Bryan
		dengan kecepatan 10 km/jam untuk sampai di kota B. Jika mereka memulai perjalanan
		dan sampai di kota B pada saat yang sama, maka berapa menit Bryan meninggalkan
		sepedanya sebelum di gunakan oleh Dika?
		\item Suatu sekolah melakukan study tour dengan menggunakan Bus. Dalam setiap bus
		terdapat jumlah siswa yang sama rata. Dan dalam setiap bus setiap siswa memiliki tempat
		duduk. Ditengah perjalanan satu Bus mengalami pecah ban dan siswa siswanya di
		distribusikan ke Bus yang masih tersisa, sekrang disetiap Bus terdapat 4 siswa yang tidak
		mendapat tempat duduk disetiap busnya. Dalam perjalanan pulang, dua Bus mengalami
		pecah ban dan siswanya didistribusikan ke Bus yang tersisa, sekarang ada 18 siswa tida
		mendapat tempat duduk di setiap busnya. Berapa banyak siswa yang mengikuti study
		tour.
		\item Koran Diskon memberikan kupon diskon sebesar $8\%$ yang dapat digabungkan dengan promo
		diskon yang lain. Alven dan Riki berniat untuk membeli sepatu A dengan harga Rp
		699.000,00. Toko Murah memberikan diskon $30\%$ untuk sepatu A. Jika Alven dan Riki
		membeli sepatu A di toko tersebut dengan menambahkan kupon diskon yang di dapatkan dari
		koran, maka berapakah selisih yang harus pembayaran yang dilakukan oleh Alven dan Riki
		jika
		\begin{enumerate}
			\item  Alven meminta kasir untuk memotong harga sepatu A dengan kupon terlebih dahulu baru
			kemudian diskon toko
			\item Riki meminta kasir untuk memotong harga sepatu A dengan diskon toko terlebih dahulu
			baru kemudian kupon yang didapat.
		\end{enumerate}
	\end{enumerate}
\end{document}
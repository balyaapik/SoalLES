\documentclass[12pt,a4paper,draft,final,oneside,twoside,openright,openany]{article}
\usepackage[latin1]{inputenc}
\usepackage{amsmath}
\usepackage{amsfonts}
\usepackage{multicol}
\usepackage{amssymb}
\usepackage{graphicx}
\usepackage[left=3.00cm, right=3.00cm, top=3.00cm, bottom=3.00cm]{geometry}
\author{Balya Rochmadi}		
\title{Soal Pengayaan Matematika SD \\CC:011218 Aritmatika **\\ \small sesuai kisi-kisi  \\
	Closed Book}
\begin{document}
	\maketitle
	\Large
	\begin{enumerate}
		\item \textbf{(HOTS-USBN)}Sepertiga umur Luna ditambah seperempat umur Lana adalah sebelas tahun. Duaperlima umur Lina ditambah duapertiga umur Luna adalah tiga belas tahun. Tigaperempat umur Lana ditambah tigaperlima umur Lina adalah enam belas tahun. Jumlah umur ketiga anak tersebut adalah ...
		\item\textbf{(HOTS-USBN)} Sebanyak 100 jeruk akan dibagi habis kepada 30 siswa didalam kelas. Siswa laki-laki mendapat 5 jeruk dan siswa perempuan mendapat 3 jeruk. Perbandingan banykanya siswa laki- laki terhadap banyak siswa perempuan didalam kelas tersebut adalah?
		\item \textbf{(HOTS-USBN)} -rata sembilan bilangan adalah 6. Satu diantaranya diabaikan sehingga rata-rata delapan bilangan yang tersisa adalah 6,5. Bilangan yang diabaikan itu adalah?
		\item \textbf{(HOTS-USBN)}Aku adalah sebuah bilangan. Jika aku dibagi oleh 16,maka hasilnya 8.Jika adalah suatu bilangan yang 30 kurangnya dari aku,maka nilai adalah ...
		\item \textbf{(HOTS-USBN)} Tujuh puluh persen dari setengah jumlah siswa dikelas V adalah laki-laki. Jika ada 26 siswa perempuan di kelas tersebut, maka paling sedikit siswa di kelas V adalah...
		\item \textbf{(HOTS-USBN)} Andre dan Sule sedang berbincang-bincang. Mereka saling menunjukkan jumlah uang yang ada di dompet masing-masing. Andre berkata kepada Sule, "Jika kamu memberiku Rp 1.500,00 maka uangku 4 kali lipat dari uangmu?". Kemudian, Sule berkata kepada Andre "Jika kamu memberiku Rp 1.500,00, maka jumlah uang kita akan sama?". Berapakah jumlah uang yang dimiliki Andre dan Sule?
		\item\textbf{(HOTS-USBN)} Ibu Halini memiliki 5 anak yang bernama Dadu, Didi, Dudu, Dodo, dan Dede. Diketahui Didi memiliki tinggi badan 174 cm, Dudu tingginya 178 cm, dan Dede tingginya 172 cm. Jika Dodo yang tertinggi dalam 5 bersaudara itu dan Dede yang paling pendek serta rata-rata tinggi mereka tidak lebih dari 176,6 cm dan tinggi 5 bersaudara itu berupa bilangan bulat semua, maka berapakah tinggi maksimal yang mungkin dari Dodo
		\end{enumerate}
\end{document}
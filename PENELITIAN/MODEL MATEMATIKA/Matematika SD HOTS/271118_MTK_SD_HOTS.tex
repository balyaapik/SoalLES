\documentclass[12pt,a4paper,draft,final,oneside,twoside,openright,openany]{article}
\usepackage[latin1]{inputenc}
\usepackage{amsmath}
\usepackage{amsfonts}
\usepackage{multicol}
\usepackage{textcomp}
\usepackage{amssymb}
\usepackage{pdfpages}
\usepackage{graphicx}
\usepackage[left=3.00cm, right=3.00cm, top=3.00cm, bottom=3.00cm]{geometry}
\author{Balya Rochmadi}		
\title{Soal Pengayaan Matematika SD HOTS \\Mapel : Teori Bilangan \\Closed Book}
\begin{document}
	\maketitle
	\Large
			\noindent\makebox[\linewidth]{\rule{\paperwidth}{0.4pt}}
	\begin{enumerate}
		\item Berat Amir, Badu dan Ciko adalah 150 kg, berat Amir dan Badu adalah 112 kg. Jika berat Amir dan Ciko adalah 103 kg. Tentukan berat Amir!
		\item Tiga tahun yang lalu umurku 12 tahun dan umur Ayah tiga kali umurku. Berapakah umur ayah tahun depan?
		\item Sebuah pekerjaan dapat diselesaikan oleh seseorang dalam waktu 8 hari.Bila pekerjaan tersebut diselesaikan oleh 2 orang pekerja, maka dapat selesai dalam waktu ...hari
		\item Umur ayah saat ini 45 tahun dan lebih tua 7 tahun dari paman. Jika umur saya saat ini 12 tahun, maka perbandingan umur paman dengan saya 6 tahun yang lalu adalah ...
		\item Nilai tiga kali ulangan harian Ghifari adalah 85, 70, dan 85. Setelah ujian ke-4 dan ke-5, ternyata rata-rata ulangan harian Ghifari menjadi 76. Tentukan rata-rata dua ujian terakhir Ghifari?
		\item  Harga sebuah baju mula-mula adalah Rp 60.000,00 kemudian harga turun menjadi Rp 40.000,00. Ternyata pedagang masih mempunyai untung 10$\%$. Berapakah keuntungan yang diperoleh pedagang dengan memjual harga awal ?
		\item Harga pulpen warna merah Rp2.500,00 dan harga pulpen warna biru Rp3.000,00. Pak guru membeli 20 pulpen dengan mengeluarkan uang sebesar Rp51.500,00. Berapakah jumlah pulpen merah yang dibeli?
		\item Jumlah faktor-faktor prima dari 2016 adalah ...
		\item Abi  mendapat  uang  saku  dari  ibunya  $1\dfrac{2}{3}$	dari uang  jajan  adiknya.  Jika  selisih  uang  mereka  adalah Rp3.000,00. Maka jumlah uang mereka adalah...
		\item Jika skala pada peta 1 : 200.000 dan jarak kota A ke kota B 20 km, maka jarak kedua kota pada peta adalah ... mm.
		\item Sebuah foto ukurannya 3 x 4, kemudian diperkecil menjadi 2 x 3. Berapakah perbandingan luas foto yang diperkecil dengan luas ukuran mula-mula?
		\item Dua gajah dalam sehari menghabiskan 3 galon air. Berapa gallon air yang diperlukan tiga gajah dalam	6 hari?
		\item Jika bilangan 2A54B habis dibagi 9, berapakah nilai A + B = ...
		\item Jika membeli 3 buah apel ditambah 2 jeruk, harganya Rp 7.800,00. Jika membeli 2 apel ditambah 3 jeruk, harganya Rp 8.200,00. Berapa harga 1 apel ditambah 1 jeruk ?
		\item Suatu cairan kimia dengan berat 2 kg mempunyai kadar air 95$\%$. Setelah cairan tersebut dijemur ternyata kadar airnya menjadi 90$\%$. Berapakah berat cairan kimia tersebut sekarang?
	\end{enumerate}
\end{document}
\documentclass[12pt,a4paper,draft,final,oneside,twoside,openright,openany]{article}
\usepackage[latin1]{inputenc}
\usepackage{amsmath}
\usepackage{amsfonts}
\usepackage{multicol}
\usepackage{amssymb}
\usepackage{graphicx}
\usepackage[left=3.00cm, right=3.00cm, top=3.00cm, bottom=3.00cm]{geometry}
\author{Balya Rochmadi}		
\title{Soal Pengayaan Matematika SD \\CC:281118 \\
	Closed Book}
\begin{document}
	\maketitle
	\Large
	\noindent\makebox[\linewidth]{\rule{\paperwidth}{0.4pt}}
	\paragraph{Kerjakan Sebaik Mungkin, Ingat Waktu!}
	\begin{enumerate}
		\item Hasil dari $-24\times 3-(-28)-26$ adalah...
		\item Angka puluhan dari $45\times(47+23-2 \times 7)\div 8$ adalah ...
		\item Seekor rusa berlari dengan kecepatan 1200 meter/menit. Berapa detik yang dibutuhkan rusa tersebut untuk menempuh jarak 3.000 meter?
		\item Jika tiga hari yang lalu hari minggu, maka 2016 hari yang akan datang adalah hari \item 70$\%$ dari 2 hari adalah...hari...jam..menit
		\item Median adalah angka yang berada di tengah setelah suatu data diurutkan. Median dari bilangan prima antara 8 dan 32 adalah 
		\item 7 tahun yang lalu umur Ayah 35 tahun dan umur saya 4 tahun. 5 tahun yang akan datang jumlah umur kami adalah 
		\item Setiap membeli 4 buah bungkus kacang, dapat ditukar dengan 1 bungkus kacang gratis. Jika saya ingin makan 14 bungkus kacang, berapa banyak bungkus kacang yang harus saya beli?
		\item Harga 1 buku sama dengan harga 2 pensil. Jika harga 2 buku dan 3 pensil adalah Rp14.000,00, maka harga satu buku adalah 
		\item Untuk bilangan cacah X  berapakah, sehingga $\dfrac{X}{12}$ berada diantara $\dfrac{1}{6}$  dan $\dfrac{1}{3}$ ?
		 \item Nilai $x$dari persamaan
			\begin{center}
				$\dfrac{1}{2}+\dfrac{1}{3}+\dfrac{1}{4}=\dfrac{1}{2x-10}$
			\end{center}
	 
	 
	 
	 \item Hitunglah nilai dari : $1-2+3-4+5-6+...+2013-2014+2015$?
	 
	 \item JIka $5-3\dfrac{x}{6}=1\dfrac{1}{2}$ ; Berapakah $x$?
	 
	 \item Paman memelihara ayam dan kambing. Jumlah ayam dan kambing 35 ekor sedangkan jumlah kaki-kaki ayam dan kambing adalah 106 buah. Berapakah jumlah kambing paman?
	 \item Jari-jari sebuah roda sepeda adalah 28 cm, berapa kalikah roda berputar untuk menempuh jarak 8,8 meter?
	 \item Sebuah kartu permainan memiliki nomor yang berupa bilangan bulat. Seperdelapan dari bilangan tersebut adalah $1\dfrac{7}{8}$ . Berapakah bilangan yang tertera pada kartu tersebut?
	 
	\item Seorang pengendara motor melakukan $\dfrac{1}{4}$ dari perjalanan pertamanya dengan kecepatan 50 km/jam. Dia menyelesaikan sisa perjalananya dalam waktu 30 menit dengan kecepatan 60 km/jam. Berapa lama ia menyelesaikan perjalanan?
	 
	 \item Jika panjang persegi panjang di bawah adalah 24 cm. Berapakah luas daerah yang diarsir?
	\begin{center}
		\includegraphics[width=0.7\linewidth]{/home/dila/Desktop/99Pcture/Selection_062}
	\end{center}
	 
	 
	 \item Mula-mula Arin membelanjakan $\dfrac{1}{3}$ uangnya. Kemudian dibelanjakan lagi $\dfrac{1}{2}$ dari sisanya. Ternyata masih bersisa Rp4.000,00. Berapa uang Arin mula-mula?
	 \item Adik dan kakak saya mempunyai uang bersama sebesar Rp10.000,00. Jika kakak memberikan $\dfrac{1}{5}$uangnya kepada adik, ternyata kakak masih mempunyai uang Rp1.000,00 lebih banyak daripada adik. Berapa uang kakak mula-mula?
	 \end{enumerate}
	 
\end{document}
\paragraph*{}
	Rasionalisasi adalah proses yang mengakibatkan nilai-nilai pembilang yang tidak rasional menjadi hilang dan diganti dengan nilai-nilai rasional.\par
	Peraturan rasionalisasi adalah sebagai berikut:
	\begin{enumerate}
		\item Jika terdapat $\dfrac{a}{\sqrt[n]{b}}$ maka bilangan rasionalnya didapatkan dengan $\dfrac{a}{\sqrt[n]{b}}\times \dfrac{\sqrt[n]{b}}{\sqrt[n]{b}}$
		\item Jika terdapat $\dfrac{a}{\sqrt[n]{b}+\sqrt[m]{c}}$ maka bilangan rasionalnya didapatkan dengan \\ $\dfrac{a}{\sqrt[n]{b}+\sqrt[m]{c}} \times\dfrac{\sqrt[n]{b}-\sqrt[m]{c}}{\sqrt[n]{b}-\sqrt[m]{c}} $
		\item Jika terdapat $\dfrac{a}{\sqrt[n]{b}-\sqrt[m]{c}}$ maka bilangan rasionalnya didapatkan dengan \\ $\dfrac{a}{\sqrt[n]{b}-\sqrt[m]{c}} \times\dfrac{\sqrt[n]{b}+\sqrt[m]{c}}{\sqrt[n]{b}+\sqrt[m]{c}} $
	\end{enumerate}
	Semua rasionalisasi bilangan pecahan yang memiliki akar mengikuti kaidah $(x+y)(x-y)=x^2-y^2$
\documentclass[12pt,a4paper]{article}
\usepackage[latin1]{inputenc}
\usepackage{amsmath}
\usepackage{amsfonts}
\usepackage{amssymb}
\usepackage[left=3.00cm, right=3.00cm, top=3.00cm, bottom=3.00cm]{geometry}
\author{Balya Rochmadi}
\title{MATEMATIKA}
\begin{document}
	\begin{enumerate}
		\item $2x^2-(p+3)x+1=0$ berapakah p jika $m^2+n^2=3$\\
		$(m+n)^2-2mn=3$\\
		$(\dfrac{-(-p+3)}{2})^2-2(\dfrac{1}{2})=3$\\
		$\dfrac{(p-3)^2}{4}-1=3$
		
		\item $3x^2-7x+5=0 $ akar-akarnya m dan n mencari persamaan kuadrat akar-akar : $3m+2$ dan $3n+2$
		\\
		$x1=3m+2$\\
		$x2=3n+2$\\
		\\
		$(x-(3m+2))(x-(3n+2))=0$\\
		$x^2-(3n+2)x-(3m+2)x+(3m + 2)(3n+2)=0$\\
		$x^2-((3n+2)x+(3m+2)x)+(3m+2)(3n+2)=0$\\
		$x^2-x(3n+2+3m+2)+(3m+2)(3n+2)=0$\\
		$x^2-x(3(n+m)+4)+(3m+2)(3n+2)=0$
		
		
	\end{enumerate}
\end{document}
\documentclass[12pt,a4paper,draft,final,oneside,twoside,openright,openany]{article}
\usepackage[latin1]{inputenc}
\usepackage{amsmath}
\usepackage{amsfonts}
\usepackage{amssymb}
\usepackage{graphicx}
\usepackage[left=3.00cm, right=3.00cm, top=3.00cm, bottom=3.00cm]{geometry}
\author{Balya Rochmadi}		
\title{Soal Pengayaan Ekonomi SMA 1\\CLOSED BOOK }
\begin{document}
	\maketitle
	\begin{enumerate}
		\item Kepuasan konsumen dari mengonsumsi
		barang atau jasa dapat diukur atau
		dihitung dengan angka-angka, merupakan
		salah satu pendekatan untuk mempelajari
		perilaku konsumen, yaitu...
		\item Tingkat kepuasan seseorang dalam
		mengonsumsi barang atau jasa tidak dapat
		dihitung dengan angka atau satuan
		lainnya, tetapi dapat dikatakan lebih tinggi
		atau lebih rendah. Pernyataan ini
		merupakan pendapat dari penganut
		pendekatan...
		\item Turunnya harga barang menyebabkan
		pendapatan riil konsumen bertambah,
		sehingga orang dapat membeli barang
		lebih banyak. Kondisi ini disebut
		dengan...
		\item Manfaat total yang diperoleh dari
		mengonsumsi barang atau jasa disebut...
		\item Selama tambahan kepuasan (marginal
		utility) masih lebih tinggi dari harga
		barang, maka konsumen akan...
		\item Kurva
		indifference
		menggambarkan
		kombinasi dari dua macam barang atau
		lebih yang memberikan...dan...
		\item Kurva yang menunjukkan kombinasi
		konsumsi dua macam barang yang
		membutuhkan biaya (anggaran) yang
		sama besar adalah kurva...
		\item Kurva yang menghubungkan titik-titik
		keseimbangan konsumen pada saat
		pendapatan konsumen berubah sedangkan
		harga-harga relatif konstan disebut...
		\item Kurva permintaan secara grafis dapat
		diturunkan dari...
		\item Kesediaan produsen untuk mengorbankan
		faktor produksi yang satu demi menambah
		penggunaan faktor produksi yang lain
		dalam rangka menjaga tingkat produksi
		pada isoquant yang sama disebut...
	\end{enumerate}
\end{document}
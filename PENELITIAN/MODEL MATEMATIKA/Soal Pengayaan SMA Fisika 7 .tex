\documentclass[12pt,a4paper,draft,final,oneside,twoside,openright,openany]{article}
\usepackage[latin1]{inputenc}
\usepackage{amsmath}
\usepackage{amsfonts}
\usepackage{amssymb}
\usepackage{graphicx}
\usepackage[left=3.00cm, right=3.00cm, top=3.00cm, bottom=3.00cm]{geometry}
\author{Balya Rochmadi}		
\title{Soal Pengayaan Fisika SMA 3 HOTS\\ Mapel : Gaya Pada bidang tertentu\\Codename: Edison\\ \small individu,open book}
\begin{document}
	\maketitle
	\Large
	\begin{enumerate}
		\item Tiga buah balok disusun sedemkian rupa dan dikenakan gaya sebesar $42 N$, (asumsikan bahwa lantai licin),Berapakah
		\begin{enumerate}
			\item Percepatan Sistem (a)
			\item Tegangan tali tengah
			\item Gaya yang dihasilkan oleh balok 1 kg ke balok 2kg.
		\end{enumerate}
		\begin{center}
			\includegraphics[width=0.4\linewidth]{/home/dila/Desktop/99Pcture/Selection_035}
		\end{center}
		
		\item Seorang siswa memutuskan untuk memindahkan sekotak buku ke kamar asramanya dengan menarik tali yang menempel di kotak. Dia menarik dengan kekuatan 80,0 N pada sudut $25,0^o$ di atas horizontal. Kotak memiliki massa 25,0 kg, dan koefisien gesekan kinetik antara kotak dan lantai adalah 0,300. (A) Temukan percepatan kotak.
		(B) Siswa sekarang mulai menggerakkan kotak ke atas $10.0^o$ kemiringan, menjaga gaya 80,0 N yang diarahkan pada $25,0^o$ di atas garis lereng. Jika koefisien gesekan tidak berubah, percepatan baru kotak itu adalah?
		
		\item Tiga benda dihubungkan dengan tali. Tali yang menghubungkan benda 4,00 kg
		dan benda seberat 5,00 kg melewati pulley tanpa gesekan.
		Tentukan (a) percepatan masing-masing objek dan (b) ketegangan di dua senar
		 	\begin{center}
		 		\includegraphics[width=0.4\linewidth]{/home/dila/Desktop/99Pcture/Selection_036}
		 	\end{center}
		\item Lantai tanpa gesekan memiliki panjang 10,0 m, dan condong pada 35,0 derajat. Sebuah kereta luncur dimulai di bagian bawah dengan kecepatan awal 5,00 m / s menaiki tanjakan. Ketika kereta luncur mencapai titik di mana ia berhenti sejenak, kereta luncur kedua dilepaskan dari bagian atas lereng dengan kecepatan awal Vi. Kedua kereta luncur mencapai bagian bawah lereng pada saat yang sama. (A) Tentukan jarak bahwa kereta luncur pertama bepergian naik ke lereng. (B) Tentukan kecepatan awal dari kereta luncur kedua
		\item 	Seorang pesulap menarik taplak meja dari bawah gelas 200 g yang terletak 30,0 cm dari tepi kain. Kain memberikan gaya gesekan 0,100 N pada cangkir dan ditarik dengan percepatan konstan 3,00 $m/s^2$. Seberapa jauh cangkir bergerak relatif terhadap meja-atas horisontal sebelum kain benar-benar keluar dari bawahnya? Perhatikan bahwa kain harus bergerak lebih dari 30 cm relatif terhadap meja selama proses terjadi
				
	\end{enumerate}
\end{document}
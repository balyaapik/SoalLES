\documentclass[12pt,a4paper,draft,final,oneside,twoside,openright,openany]{article}
\usepackage[latin1]{inputenc}
\usepackage{amsmath}
\usepackage{amsfonts}
\usepackage{multicol}
\usepackage{textcomp}
\usepackage{amssymb}
\usepackage{graphicx}
\usepackage[left=3.00cm, right=3.00cm, top=3.00cm, bottom=3.00cm]{geometry}
\author{Balya Rochmadi}		
\title{Soal Pengayaan Fisika SMA 10 HOTS\\Mapel : Energi 1I: Konservasi Energi dan Energi Potensial \\Closed Book}
\begin{document}
	\maketitle
	\large
	\noindent\makebox[\linewidth]{\rule{\paperwidth}{0.4pt}}
	\paragraph{Petunjuk:}
		\begin{enumerate}
			\item \textbf{Energi Potensial Gravitasi dan Energi Potensial Benda Elastis}\\
			Usaha yang dilakukan pada suatu partikel oleh gaya gravitasi konstan dapat direpresentasikan sebagai perubahan dalam energi potensial gravitasi $U_{grav} = mgy$. Energi ini dimiliki bersama oleh partikel dan bumi. Energi potensial juga terkait dengan gaya elastis $F_x = -kx$ yang diberikan oleh pegas yang ideal, di mana x adalah jumlah peregangan atau kompresi. Pekerjaan yang dilakukan oleh kekuatan ini dapat direpresentasikan sebagai perubahan dalam energi potensial elastis pegas,$U_{el}=\dfrac{1}{2}kx$\\
						$W_g=mg(y_1-y_0)$ dan $W_e=\dfrac{1}{2}kx_1-\dfrac{1}{2}kx_2$
			\item \textbf{Konservasi(Kekekalan) Energi Mekanik}\\
			Total energi potensial U adalah jumlah energi potensial gravitasi dan elastis: $U = U_{grav} + U_{el}$. Jika tidak ada kekuatan lain selain gaya gravitasi dan elastis melakukan kerja pada sebuah partikel, jumlah energi kinetik dan potensial adalah kekal. Jumlah ini $ E = K + U$ disebut energi mekanik total.\\
						$K_1+U_1=K_0+U_0$\\
						$U=U_{grav}+U_{el}$\\
			\item Energi Mekanik tidak terkonservasi\\
			Ketika gaya selain gaya gravitasi dan elastis bekerja pada sebuah partikel, usaha yang lain dilakukan oleh kekuatan-kekuatan lain sama dengan perubahan total energi mekanik (energi kinetik ditambah total energi potensial).\\
			
						$K_1+U_1+W_{\text{usaha lain}}=K_2+U_2$
			\item \textbf{Gaya, Usaha, dan Energi Internal Konservatif}.\\ 
			Semua gaya bersifat konservatif atau non-konservatif. Gaya konservatif adalah salah satu yang hubungan energi kerja-kinetik sepenuhnya reversibel. Gaya konservatif selalu dapat diwakili oleh fungsi energi potensial, tetapi gaya non-konservatif tidak bisa. Usaha yang dilakukan oleh gaya non-konservatif memanifestasikan dirinya sebagai perubahan dalam energi internal benda. Jumlah energi kinetik, potensial, dan internal selalu konservatif.\\
					$\Delta K+\Delta U+\Delta U_{internal}=0$
					
			\item \textbf{Konservasi Energi dalam Vektor}\\
			Untuk gerakan sepanjang garis lurus, gaya konservatif Fx adalah
			turunan negatif dari fungsi energi potensial yang terkait U. Dalam tiga dimensi, komponen kekuatan konservatif adalah turunan parsial negatif dari U.\\
					$\vec{F}=-\dfrac{\delta U}{\delta x}\hat{i}-\dfrac{\delta U}{\delta y}\hat{j}-\dfrac{\delta U}{\delta z}\hat{k}$
			
		\end{enumerate}
	\noindent\makebox[\linewidth]{\rule{\paperwidth}{0.4pt}}
	\paragraph{Kerjakan Dengan Sungguh-sungguh!}
	\begin{enumerate}
		\item Di lokasi konstruksi, ember seberat 65,0 kg menggantung dari kabel ringan (tapi kuat) yang melewati katrol bebas gesekan dan terhubung ke kotak 80,0 kg di atap horizontal.Kabel menarik secara horizontal pada kotak, dan 50,0 kg kantong kerikil bertumpu di atas kotak. (A) Temukan gaya gesekan pada kantong kerikil dan pada kotak. (B) Tiba-tiba seorang pekerja mengambil tas kerikil. Gunakan konservasi energi untuk menemukan kecepatan ember setelah turun 2,00 m dari posisi diam. (Anda dapat memeriksa jawaban Anda dengan menyelesaikan masalah ini menggunakan hukum Newton.)
		
			\begin{center}
				\includegraphics[width=0.6\linewidth]{/home/dila/Desktop/99Pcture/Selection_063}
			\end{center}
		
		
		\item Anda menguji roller coaster taman hiburan baru dengan mobil kosong berukuran 120 kg. Satu bagian dari trek adalah lingkaran vertikal dengan radius 12,0 m. Di bagian bawah loop (titik A) mobil memiliki kecepatan 25,0 m / s, dan di bagian atas loop (titik B) memiliki kecepatan 8,0 m / s. Ketika mobil berputar dari titik A ke titik B, berapa banyak usaha yang dilakukan oleh gaya gesek tersebut?
	
		\item Di stasiun pemuatan truk di kantor pos, paket kecil 0,200 kg dilepaskan dari posisi berdiri pada titik A di lintasan seperempat lingkaran dengan radius 1,60 m. Ukuran paket kurang dari 1,60 m, sehingga paket dapat diperlakukan sebagai partikel.paket meluncur ke bawah trek dan mencapai titik B dengan kecepatan 4,80 m / s. Dari titik B, itu meluncur di permukaan dengan jarak 3,00 m ke titik C, dan kemudian berhenti. (a) Berapa koefisien gesekan kinetik pada permukaan horizontal? (b) Seberapa besar usaha yang dilakukan oleh gaya gesek ketika paket tersebut melalui busur lingkaran dari A ke B?
	
			\begin{center}
				\includegraphics[width=0.6\linewidth]{/home/dila/Desktop/99Pcture/Selection_060}
			\end{center}
			
		\item Sistem dua ember cat yang dihubungkan dengan tali ringan dilepaskan dari sisa dengan ember 12,0 kg 2,00 m di atas lantai. Gunakan prinsip konservasi energi untuk menemukan kecepatan ember ini menyentuh lantai. Anda dapat mengabaikan gesekan dan massa katrol.
		
		\begin{center}
			\includegraphics[width=0.6\linewidth]{/home/dila/Desktop/99Pcture/Selection_061}
		\end{center}
		
		
		
	\end{enumerate}
\end{document}
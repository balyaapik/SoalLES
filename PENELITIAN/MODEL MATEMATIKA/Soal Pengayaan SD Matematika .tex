\documentclass[12pt,a4paper,draft,final,oneside,twoside,openright,openany]{article}
\usepackage[latin1]{inputenc}
\usepackage{amsmath}
\usepackage{amsfonts}
\usepackage{multicol}
\usepackage{amssymb}
\usepackage{graphicx}
\usepackage[left=3.00cm, right=3.00cm, top=3.00cm, bottom=3.00cm]{geometry}
\author{Balya Rochmadi}		
\title{Soal Pengayaan Matematika SD 3 HOTS \\Closed Book}
\begin{document}
	\maketitle
		\noindent\makebox[\linewidth]{\rule{\paperwidth}{0.4pt}}
	\Large
	\begin{enumerate}
		\item Sebuah bak mandi dapat diisi penuh air dengan cara mengisinya dengan sembilan ember ke-
		cil air dan empat ember besar air. Bak mandi tersebut dapat pula diisi penuh air dengan cara
		mengisinya dengan enam ember besar air dan enam ember kecil air. Perbandingan volume
		ember besar dan volume ember kecil adalah
		\item Anto diminta menentukan kelipatan persekutuan terkecil (KPK) dari empat bilangan, yaitu 11,
		12, 13, dan A. Ketika menghitung, ia salah menuliskan 12 menjadi 21. Meskipun demikian,
		nilai KPK yang diperolehnya benar. Nilai terkecil yang mungkin untuk A adalah?
		\item Sebuah perusahaan obat mengirimkan
		produknya ke sebuah apotek sebanyak 1,5 kodi
		paket kotak. Di setiap paket, terdapat 2 lusin
		botol yang berisi 3 gros butir obat. Berapa
		butirkah obat yang dikirimkan perusahaan
		tersebut?
		\item Pak Didi membagikan 288 buah buku kepada 8 kelompok. Masing-masing kelompok beranggotakan 4 anak. Maka, setiap anggota menerima?
		\item Pak Raka menjual tanah dengan harga Rp 1.500.000 per $m^2$ . Dari hasil penjualan 
		tanah, ia memberikan 25$\%$ kepada panti asuhan. Saat ini uang Pak Raka ialah Rp 
		168.750.000. Berpakah luas tanah Pak Raka? 
		\item Suatu pipa air mempunyai 625 cabang saluran air untuk memenuhi keperluan 
		keluarga dalam satu minggu tercapai 112.500 liter.Berapa rata-rata dipakai tiap 
		keluarga dalam satu minggu? (liter)
		\item Fauzi mengendarai sepeda dalam 5 hari dapat menempuh jarak 426 kilometer. 
		Hari pertama mencapai 90 kilometer, hari kedua mencapai 65 kilometer, hari 
		ketiga mencapai 110 kilometer, dan hari keempat 95 kilometer. 
		Berapa kilometer ditempuh pada hari kelima? 
		\item Satu potong bebek beratnya berkurang 20$\%$ setelah dimasak menjadi bebek 
		goreng. Bila berat bebek tersebut dalam bentuk sepiring bebek panggang  ialah 8 
		ons. Berat bebek semula adalah...kg 
		\item Rata-rata tinggi sekelompok anak laki-laki
		dan anak perempuan masing-masing adalah
		170,5 cm dan 162,5 cm. Jika rata-rata gabungan
		dari keduanya adalah 167,5 cm, perbandingan
		jumlah anak laki-laki dan perempuan adalah?
		\item Sebuah tongkat memiliki panjang 3 m + 0,2
		dam + 5 dm + 23 cm. Panjang tongkat tersebut
		sama dengan ... (1 kaki = 30,48 cm, 1 yard =
		91,44 cm, dan 1 inchi = 2,54 cm)
		\item Sebuah kolam memiliki panjang 270 dm,
		lebar 2,1 dam, dan tinggi 35 m. Berapa literkah
		volume kolam tersebut?
		\item Seorang penjual beras membeli persediaan
		beras jenis A sebanyak 1 kuintal, 70 kg beras
		jenis B dengan harga masing-masing
		Rp4.100,00 dan Rp4.300,00 per kilonya. Ia
		mencampur kedua macam beras yang telah ia
		beli tersebut kemudian menjualnya dengan
		harga rata-rata uang yang ia keluarkan dengan
		banyaknya beras yang ia dapatkan (belum
		termasuk perhitungan laba). Jika ia ingin
		mendapat keuntungan bersih Rp96.500,00 dari
		total beras yang ada, berapa harga per kilo yang
		harus ia tawarkan ke konsumen?
		\item Fandy membeli 1 lusin kaos. Harga setiap lusin Rp 150.000. Jika Fandy membeli 
		28 kaos, berapa ia harus membayar?
		\item Sebuah sepatu dijual dengan harga Rp 250.000. Keuntungan dari harga jual 
		tersebut adalah 20$\%$. Tentukan harga pembelian sepatu tersebut!
		\item Seorang pedagang membeli 0,5 ton beras, 1,5 kuintal tepung terigu, dan 3 
		karung gula pasir dengan berat masing-masing 50 kg. Berat seluruh belanjaan 
		pedagang tersebut adalah .... kuintal.
	\end{enumerate}
\end{document}
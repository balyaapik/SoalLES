\documentclass[12pt,a4paper,twoside]{article}
\usepackage[latin1]{inputenc}
\usepackage{amsmath}
\usepackage{amsfonts}
\usepackage{amssymb}
\usepackage[left=3.00cm, right=3.00cm, top=3.00cm, bottom=3.00cm]{geometry}
\author{Balya Rochmadi}
\title{EKONOMI SBMPTN}
\begin{document}
	\section{PRINSIP EKONOMI}
	\paragraph {Masalah Ekonomi}
	\subparagraph{}
		Masalah ekonomi dasar adalah kelangkaan sumber daya
	\paragraph{Pengertian Sistem Ekonomi}
	\subparagraph{}
	Kelangkaan dalam sumber daya ekonomi yang menimbulkan masalah ekonomi, sehingga akan melahirkan tindakan untuk memecahkannya. Suatu negara memiliki cara tersendiri dalam memecahkan permasalahan ekonomi negaranya, yang kemudian disebut dengan sistem ekonomi.
	\paragraph{}
	\textbf{Sistem Ekonomi Adalah: }
	\begin{enumerate}
		\item Mekanisme untuk mengatasi masalah kelangkaan.
		\item Aturan main hidup berekonomiss
		\item Aturan-aturan atau cara-cara yag menjadi satu kesatuan dan digunakan untuk mencapai tujuan dalam perekonomian.
	\end{enumerate}
	
	\textbf{Perbedaan Sistem Ekonomi Setiap Negara Tergantung:}
	\begin{enumerate}
		\item Perbedaan Sumber Daya
		\item Perbedaan Sistem Pemerintahan Negara
	\end{enumerate}
	
	\textbf{Fungsi Sistem Ekonomi}
	\begin{enumerate}
		\item Sarana Pendorong Produksi
		\item Cara untuk mengkoordinasi kegiatan individu (dalam memproduksi barang dan jasa)
		\item Menciptakan mekanisme pendistribusian barang dan jasa dengan baik
	\end{enumerate}

	\textbf{text}
		
		
		
		
\end{document}
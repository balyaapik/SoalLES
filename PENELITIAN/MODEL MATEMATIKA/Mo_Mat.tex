\documentclass[12pt,a4paper,twoside,openany]{book}
\usepackage[latin1]{inputenc}
\usepackage[bahasa]{babel}
\usepackage{amsmath}
\usepackage{amsfonts}
\usepackage{amssymb}
\usepackage{makeidx}
\usepackage{graphicx}
\usepackage{lmodern}
\usepackage{fourier}
\usepackage{tcolorbox}
\usepackage{booktabs}
\usepackage{multicol,lipsum}
\usepackage[left=3cm,right=2cm,top=2cm,bottom=2cm]{geometry}


%----------------TABLE--------------------%
\usepackage{siunitx}
\usepackage{booktabs}
\usepackage{array}
\newcolumntype{C}[1]{>{\centering\arraybackslash}p{#1}}
%-----------------TABLE-------------------%

%---------------MULTIPLICATION-----------%
\usepackage{xlop}
%---------------XXXX-----------%


%-----------------CANCELLING-------------%
\usepackage{cancel}
%---------------XXXXX--------------------------$

%---------------------------------------%
\usepackage{forest,mathtools,siunitx}
\makeatletter
\def\ifNum#1{\ifnum#1\relax
	\expandafter\pgfutil@firstoftwo\else
	\expandafter\pgfutil@secondoftwo\fi}
\forestset{
	num content/.style={
		delay={
			content/.expanded={\noexpand\num{\forestoption{content}}}}},
	pt@prime/.style={draw, circle},
	pt@start/.style={},
	pt@normal/.style={},
	start primeTree/.style={%
		/utils/exec=%
		% \pt@start holds the current minimum factor, we'll start with 2
		\def\pt@start{2}%
		% \pt@result will hold the to-be-typeset factorization, we'll start with
		% \pgfutil@gobble since we don't want a initial \times
		\let\pt@result\pgfutil@gobble
		% \pt@start@cnt holds the number of ^factors for the current factor
		\def\pt@start@cnt{0}%
		% \pt@lStart will later hold "l"ast factor used
		\let\pt@lStart\pgfutil@empty,
		alias=pt-start,
		pt@start/.try,
		delay={content/.expanded={$\noexpand\num{\forestove{content}}
				\noexpand\mathrlap{{}= \noexpand\pt@result}$}},
		primeTree},
	primeTree/.code=%
	% take the content of the node and save it in the count
	\c@pgf@counta\forestove{content}\relax
	% if it's 2 we're already finished with the factorization
	\ifNum{\c@pgf@counta=2}{%
		% add the factor
		\pt@addfactor{2}%
		% finalize the factorization of the result
		\pt@addfactor{}%
		% and set the style to the prime style
		\forestset{pt@prime/.try}%
	}{%
	% this simply calculates content/2 and saves it in \pt@end
	% this is later used for an early break of the recursion since no factor
	% can be greater then content/2 (for integers of course)
	\edef\pt@content{\the\c@pgf@counta}%
	\divide\c@pgf@counta2\relax
	\advance\c@pgf@counta1\relax % to be on the safe side
	\edef\pt@end{\the\c@pgf@counta}%
	\pt@do}}

%%% our main "function"
\def\pt@do{%
	% let's test if the current factor is already greather then the max factor
	\ifNum{\pt@end<\pt@start}{%
		% great, we're finished, the same as above
		\expandafter\pt@addfactor\expandafter{\pt@content}%
		\pt@addfactor{}%
		\def\pt@next{\forestset{pt@prime/.try}}%
	}{%
	% this calculates int(content/factor)*factor
	% if factor is a factor of content (without remainder), the result will
	% equal content. The int(content/factor) is saved in \pgf@temp.
	\c@pgf@counta\pt@content\relax
	\divide\c@pgf@counta\pt@start\relax
	\edef\pgf@temp{\the\c@pgf@counta}%
	\multiply\c@pgf@counta\pt@start\relax
	\ifNum{\the\c@pgf@counta=\pt@content}{%
		% yeah, we found a factor, add it to the result and ...
		\expandafter\pt@addfactor\expandafter{\pt@start}%
		% ... add the factor as the first child with style pt@prime
		% and the result of int(content/factor) as another child.
		\edef\pt@next{\noexpand\forestset{%
				append={[\pt@start, pt@prime/.try]},
				append={[\pgf@temp, pt@normal/.try]},
				% forest is complex, this makes sure that for the second child, the
				% primeTree style is not executed too early (there must be a better way).
				delay={
					for descendants={
						delay={if n'=1{primeTree, num content}{}}}}}}%
	}{%
	% Alright this is not a factor, let's get the next factor
	\ifNum{\pt@start=2}{%
		% if the previous factor was 2, the next one will be 3
		\def\pt@start{3}%
	}{%
	% hmm, the previos factor was not 2,
	% let's add 2, maybe we'll hit the next prime number
	% and maybe a factor
	\c@pgf@counta\pt@start
	\advance\c@pgf@counta2\relax
	\edef\pt@start{\the\c@pgf@counta}%
}%
% let's do that again
\let\pt@next\pt@do
}%
}%
\pt@next
}

%%% this builds the \pt@result macro with the factors
\def\pt@addfactor#1{%
	\def\pgf@tempa{#1}%
	% is it the same factor as the previous one
	\ifx\pgf@tempa\pt@lStart
	% add 1 to the counter
	\c@pgf@counta\pt@start@cnt\relax
	\advance\c@pgf@counta1\relax
	\edef\pt@start@cnt{\the\c@pgf@counta}%
	\else
	% a new factor! Add the previous one to the product of factors
	\ifx\pt@lStart\pgfutil@empty\else
	% as long as there actually is one, the \ifnum makes sure we do not add ^1
	\edef\pgf@tempa{\noexpand\num{\pt@lStart}\ifnum\pt@start@cnt>1 
		^{\noexpand\num{\pt@start@cnt}}\fi}%
	\expandafter\pt@addfactor@\expandafter{\pgf@tempa}%
	\fi
	% setup the macros for the next round
	\def\pt@lStart{#1}% <- current (new) factor
	\def\pt@start@cnt{1}% <- first time
	\fi
}
%%% This simply appends "\times #1" to \pt@result, with etoolbox this would be
%%% \appto\pt@result{\times#1}
\def\pt@addfactor@#1{%
	\expandafter\def\expandafter\pt@result\expandafter{\pt@result \times #1}}

%%% Our main macro:
%%% #1 = possible optional argument for forest (can be tikz too)
%%% #2 = the number to factorize
\newcommand*{\PrimeTree}[2][]{%
	\begin{forest}%
		% as the result is set via \mathrlap it doesn't update the bounding box
		% let's fix this:
		tikz={execute at end scope={\pgfmathparse{width("${}=\pt@result$")}%
				\path ([xshift=\pgfmathresult pt]pt-start.east);}},
		% other optional arguments
		#1
		% And go!
		[#2, start primeTree]
	\end{forest}}
	\makeatother
%--------------------------------------

%--------------pm--------------------%
\makeatletter
\newcommand{\mypm}{\mathbin{\mathpalette\@mypm\relax}}
\newcommand{\@mypm}[2]{\ooalign{%
		\raisebox{.1\height}{$#1+$}\cr
		\smash{\raisebox{-.6\height}{$#1-$}}\cr}}
\makeatother
%-------------------------------------%



\author{MGMP MATEMATIKA SMP PROVINSI JAWA TENGAH}
\title{MODEL DAN MATERI UNBK MATEMATIKA SMP T.A 2017/2018\\ Pendalaman dan Pengayaan Materi Matematika}
\begin{document}
\maketitle
%empty {} for no dots. you can have any symbol inside. For example put {\ensuremath{\ast}} and see what happens.
\tableofcontents
\chapter{Bilangan Dan Teori Bilangan}
\section{Bilangan Bulat}
	
	\paragraph*{}
	
	Bilangan bulat adalah bilangan yang terdiri atas $...,-1,0,1,2,..$. Atau dapat dikatakan bahwa bilangan bulat memenuhi $x \in N$ dengan $N$ adalah bilangan bulat positif dan negatif, sedangkan nol termasuk kedalam bilangan nonnegatif dan nonpositif atau bisa disebut sebagai bilangan identitas. Siswa diminta untuk dapat memahami operasi dari bilangan bulat tersebut terutama dalam operasi pembagian, perkalian, penambahan dan pengurangan, termasuk jarak,temperatur, dan selisih.  Bilangan bulat termasuk bilangan rasional karena dapat diubah dalam bentuk $\frac{a}{b}$. Operasi bilangan bulat adalah sebagai berikut.
	\begin{enumerate}
		\item \textbf{Operasi Penambahan} 	: $x+y=m$\\
		Jika $x<y$ atau $x>y$ dan $x,y>0$ maka $m>0$ misal $1+2=3$\\
		Jika $x<y$, dengan kondisi $x<0$,$y>0$  maka $m>0$ misal $-3+5=2$\\
		Jika $x>y$, dengan kondisi $x>0$,$y<0$, maka $m<0$ misal $3+(-5)=-2$\\
		Jika $x<y$ atau $x>y$, dengan kondisi $y<0$,$x<0$  maka $m<0$ misal $-3+(-5)=-8$
		%masih harus diperbaiki..........
		\item \textbf{Operasi Pengurangan} 	: $x-y=m$\\
		JIka $y<0$ maka $x-y=x+y=m$ misal $5-(-5)=5+5=10$\\
		Jika $y>0$ maka $x-y=m$ misal $5-(+5)=5-5=0$
		\item \textbf{Operasi Perkalian}	 	: $xy=m$\\
		Jika $x,y>0$ maka $m>0$, Jika $x \text{ atau} y < 0$ maka $m<0$		
		\item \textbf{Operasi Pembagian}   	: $x\div y=m$\\
		Jika $x,y>0$ maka $m>0$, Jika $x \text{ atau} y < 0$ maka $m<0$
		\item \textbf{Jarak, Selisih, Suhu }\\
		Dihitung dengan menggunakan $D=|x-y|$\\
		
	\end{enumerate}
\newpage
\paragraph{MODEL SOAL UNBK 2017/2018}
\begin{enumerate}
	\item A membeli 40 bungkus makanan dengan 10 potong ayam disetiap bungkusnya. Dia ingin membagikan bungkusan tersebut kepada 7 temannya, berapa bungkuskah sisa dari makanan tersebut?
	\item R berada di kota X dan akan menuju kota Y yang jaraknya 10km, dan dari kota Y ke kota Z dengan jarak 25km. Saat di kota Z, R teringat bahwa ada barang yang tertinggal dia kembali lagi ke kota X untuk mengambil barang yang ketinggalan tersebut dan kembali lagi ke kota Z. berapakah total jarak yang R tempuh?
	\item Suhu di kamar ber AC adalah $17^o$ C. Setelah
	AC dimatikan suhunya naik $3^o$ C setiap menit.
	Suhu kamar setelah 4 menit adalah ....
	\item Pada lomba Matematika ditentukan untuk
	jawaban yang benar mendapat skor 2,
	jawaban yang salah mendapat skor $-1$,
	sedangkan bila tidak menjawab mendapat
	skor 0. Dari 75 soal yang diberikan, seorang
	anak menjawab 50 soal dengan benar dan 10
	soal tidak dijawab. Skor yang diperoleh anak
	tersebut adalah ....
	\item Suhu di Jakarta
	pada termometer menunjukkan $34^o$ C(di atas $0^o$). Jika pada saat itu suhu di Jepang ternyata $37^o C$di bawah suhu Jakarta, maka suhu di Jepang adalah ....
	\item Suatu turnamen catur ditentukan bahwa
	peserta yang menang memperoleh skor 6, seri
	mendapat skor 3, dan bila kalah mendapat
	skor $-2$. Jika hasil dari 10 pertandingan
	seorang peserta menang 4 kali dan seri 3 kali,
	maka skor yang diperoleh peserta tersebut
	adalah ....
	\item Suhu dalam ruang tamu $23^o C$. Suhu di dalam rumah $17^o C$ lebih tinggi dari suhu di ruang tamu dan suhu di dalam kulkas $28^o C$ lebih rendah dari ruang tamu. Oleh karena itu suhu di kulkas adalah ....
	
	\item Suhu pagi hari di suatu tempat adalah $-9^o$ C.
	Pada siang harinya mengalami kenaikan
	sebesar $4^o$ C dan pada malam hari suhu
	mengalami penurunan sebesar $8^o$C dan
	bertahan hingga pagi. Suhu pada pagi hari
	berikutnya adalah ....
\end{enumerate}
		
		
		
		
		
		
		
		
		
		
\section{Keterbagian}
	\paragraph*{}
		Suatu bilangan dikatakan habis dibagi oleh suatu bilangan lain jika hasil baginya adalah bilangan bulat. Misalnya 6 habis dibagi 3 karena $6 : 3 = 2$ sedangkan 7 tidak habis dibagi 3 karena $7 : 3 = 2$. 
\par		
		6 habis dibagi 3 dinotasikan dengan $3|6$ (hasilnya bukan bilangan bulat) sedangkan 7 tidak habis dibagi 3 dinotasikan dengan $3\nmid 7$. Dengan demikian dapat dikatakan bahwa a membagi b (diberi simbol a|b), jika ada suatu bilangan bulat k sedemikian hingga $b = ka$.
		Sedangkan jika a tidak habis membagi b, maka ada bilangan bulat c yang
		merupakan sisa dari pembagian b oleh a, sehingga dapat ditulis sebagai bentuk $b = ka + c.$\par 
		Sifat: Jika $a|b$ dan $c|b$ maka $ac|b$. Contoh : $3|72$ dan $4|72$, maka $12|72$ \par 
		Sifat-sifat khusus pada pembagian bilangan bulat : 
		\begin{enumerate}
			\item Suatu bilangan terbagi oleh 9 jika dan hanya jika jumlah angka-angkanya
			terbagi oleh 9
			\item Suatu bilangan terbagi oleh 3 jika dan hanya jika jumlah angka-angkanya
			terbagi oleh 3
			\item Suatu bilangan terbagi oleh 2 jika dan hanya jika angka terakhirnya terbagi
			oleh 2
			\item Suatu bilangan terbagi oleh 4 jika dan hanya jika dua angka terakhirnya
			habis dibagi oleh 4
			\item Suatu bilangan terbagi oleh 8 jika dan hanya jika tiga angka terakhir
			bilangan tersebut habis dibagi 8
			\item Suatu bilangan terbagi oleh 6 jika dan hanya jika bilangan tersebut habis
			dibagi oleh 2 dan habis pula dibagi oleh 3
			\item Jika abcdefg.... adalah suatu bilangan, maka abcdefg habis dibagi 11 jika
			dan hanya jika 11| (a+c+e+g+...)-(b+d+f+...)
			\item suatu bilangan habis dibagi a dan juga habis dibagi b, jika dan hanya
			bilangan tersebut akan habis dibagi ab dengan syarat a dan b relatif prima. \textit{(Dua bilangan dikatakan relatif prima, jika faktor persekutuan terbesarnya
			(FPB) dua bilangan tersebut sama dengan 1). Contoh: 36 habis dibagi 4
			dan 3, maka 36 habis dibagi 12 ($4\times 3$). Sedangkan 36 habis dibagi 12 dan 6,
			tetapi 36 tidak habis dibagi 72 ($12 \times 6$)	}
		\end{enumerate}		
\paragraph{Model Soal UNBK 2017/2018}
	\begin{enumerate}
		\item Tentukan nilai p yang merupakan digit dalam bilangan dalam persamaan
		berikut ini.
		\begin{center}
			$81 \times 586794=475p0p14$\\
		\end{center}
		\textbf{Jawab} : Bilangan habis dibagi 9 karena hasil perkalian dari 81. Oleh
		karena itu, jumlah digit-digitnya habis dibagi 9 pula. Dengan kata lain 9|
		(4+7+5+0+1+4+2p) = 9| (21+ 2p). Sedangkan bilangan terdekat dengan 21
		yang habis dibagi 9 adalah 27, maka persamaan menjadi 21+ 2p = 27,
		sehingga p = 3.
	\item Berapakah sisa hasil bagi dari,
	\begin{center}
			$878787878787....8724 \div 4 = ....$
	\end{center}
	Karena dua angka yang paling belakang dapat dibagi empat sisanya berarti nol
	\item Dapatkah,
	\begin{center}
		$11|625812?$
	\end{center}
	Ya karena $11|(6+5+1)-(2+8+2)=11|12-12=11|0$.
	\item 100 dibagi x sisa 9, dan 80 dibagi x sisa 8, berapakah x?\\
	\textbf{Jawab}: $100=mx+c=mx+9$, dan $80=nx+8$, dapat kita sederhanakan menjadi $90=mx$ dan $72=nx$, FPB dari keduanya adalah 9 jadi $x=9$;
	\item Dalam sebuah permainan jika $x+y=45$ dan $xy=25$, berapakah $x^2+y^2$. \\
	\textbf{Jawab :} Karena,\\ 
	$(x+y)^2=45^2$\\
	$x^2+2xy+y^2=2025$\\
	$x^2+y^2=2025-50=1975$
	\end{enumerate}
	
				
\section{Bilangan Pecahan}
	\paragraph{Operasi Pecahan Campuran (Desimal dan Biasa)}
	Siswa diminta untuk menentukan nilai dari pecahan campuran yang telah terlebih dahulu diacak oleh komputer, dalam hal ini masalah pecahan yang dimaksud berupa soal cerita.
	\par
	\begin{enumerate}	
	\item \textbf{Operasi Penjumlahan:} Dalam operasi penjumlahan ada dua hal penting yang harus diperhatikan. Pertama, ketika kita akan menjumlahkan pecahan dengan penyebutnya yang telah sama, maka kita dapat secara langsung menjumlahkan pembilang-pembilangnya saja.
	
	Contoh 1:\\
	$\frac{2}{5}+\frac{6}{5}=\frac{2+6}{5}=\frac{8}{5}$
	
	Jika Penyebut belum sama maka disamakan terlebih dahulu \\
	Contoh 2:\\
	
	$\frac{2}{3}+\frac{4}{5}=\frac{2\times5}{3\times5}+\frac{4\times3}{5\times3}=\frac{10}{15}+\frac{12}{15}=\frac{22}{15}$
	
	Agar lebih mudahnya, perhatikan formula berikut ini:\\
	\\
	$\frac{a}{b}+\frac{c}{b}=\frac{a+c}{b}$
	\\
	$\frac{a}{b}+\frac{c}{d}=\frac{ad+bc}{bd}$
	\\
	\textbf{SIFAT PENJUMLAHAN PADA PECAHAN}
	\begin{enumerate}
		\item Sifat Komutatif Penjumlahan\\
		$\frac{a}{b}+\frac{c}{b}=\frac{c}{b}+\frac{a}{b}$\\
		\item Sifat Asosiatif Penjumlahan\\
		$\left(\frac{a}{b}+\frac{c}{b}\right)+\frac{d}{b}=\frac{a}{b}+\left(\frac{c}{b}+\frac{d}{b}\right)$
	\end{enumerate}
	
	
	\textbf{PENGURANGAN PADA PECAHAN}
	
	Perhatikan contoh berikut.
	
	$\frac{5}{8}-\frac{4}{8}=\frac{1}{8}
	\frac{3}{7}-\frac{2}{8}=\frac{3\times8}{7\times8}-\frac{2\times7}{8\times7}=\frac{24}{56}-\frac{14}{56}=\frac{24-14}{56}=\frac{10}{56}$
	
	Agar lebih mudahnya, perhatikan kedua formula berikut ini:
	
	$\frac{a}{b}-\frac{c}{b}=\frac{a-c}{b}$\\
	
	$\frac{a}{b}-\frac{c}{d}=\frac{ad-bc}{bd}$
	
	Pengurangan Pecahan Tidak Bersifat Komutatif
	
	$\frac{a}{b}-\frac{c}{b}\neq\frac{c}{b}-\frac{a}{b}$
	
	Perkalian Pecahan
	
	Pada operasi perkalian pecahan berlaku pengerjaan-pengerjaan seperti berikut ini.
	
	
	SIFAT OPERASI PERKALIAN PADA PECAHAN
	
	Sifat 1 : Sifat Komutatif Perkalian
	
	$\frac{a}{b}\times\frac{c}{d}=\frac{c}{d}\times\frac{a}{b}$
	
	Sifat 2 : Sifat Asosiatif Perkalian
	
	$\left(\frac{a}{b}\times\frac{c}{d}\right)\times\frac{p}{q}=\frac{a}{b}\times\left(\frac{c}{d}\times\frac{p}{q}\right)$
	
	Sifat 3 : Sifat Distributif Perkalian terhadap Penjumlahan
	
	$\frac{a}{b}\times\left(\frac{c}{d}+\frac{p}{q}\right)=\left(\frac{a}{b}\times\frac{c}{d}\right)+\left(\frac{a}{b}\times\frac{p}{q}\right)$
	
	Sifat 4 : Sifat Distributif Perkalian terhadap Pengurangan
	
	$\frac{a}{b}\times\left(\frac{c}{d}-\frac{p}{q}\right)=\left(\frac{a}{b}\times\frac{c}{d}\right)-\left(\frac{a}{b}\times\frac{p}{q}\right)$
	
	Sifat 5 : Sifat Perkalian Pecahan dengan Bilangan 1
	
	$\frac{a}{b}\times1=\frac{a}{b}$
	
	Sifat 6 : Sifat Perkalian Pecahan dengan Bilangan 0
	
	$\frac{a}{b}\times0=0\times\frac{a}{b}=0$
	
	Sifat 7 : Sifat Urutan Pecahan
	
	$\frac{a}{b}>\frac{c}{d}\Longleftrightarrow ad>bc$
	
	
	
	Pembagian Pecahan
	
	Dalam operasi pembagian pecahan, sembarang $\frac{a}{b} dan \frac{c}{d} dengan b\neq0 dan d\neq 0$
	
	\end{enumerate}
 
	 	\opmul[displayshiftintermediary=all]{123}{456}
	 	\oplput(1,3){(this is $123 \times 6$.)}
	 	\oprput(-5,2){$+$}
	 	\oplput(1,2){(this is $123 \times 5$, shifted one position to the left.)}
	 	\oplput(1,1){(this is $123 \times 4$, shifted two positions to the left.)}
	
\section{Bilangan Prima}
	\paragraph*{}
	Bilangan prima adalah bilangan yang hanya dapat dibagi oleh bilangan itu sendiri dan 1. Bilangan prima merupakan faktor dari semua bilangan komposit.\textit{Fundamental Theorem of Arithmetic}: Semua bilangan komposit merupakan hasil dari perkalian bilangan-bilangan prima yang unik.
	Contoh:
	\begin{enumerate}
		\item Bilangan 12 merupakan faktorisasi yang unik dari bilangan $12=2^2\times 3$  
	\end{enumerate}
	\textbf{Karakteristik Bilangan Prima :}
	\begin{enumerate}
		\item Jumlahnya tak terbatas
		\item Hanya dapat terbagi oleh 1 dan bilangan itu sendiri
		\item Angka dua adalah satu-satunya prima genap
		\item Bilangan komposit adalah bilangan-bilangan yang dapat difaktorkan menjadi angka-angka prima.
	\end{enumerate}
	Bilangan-bilangan yang \textbf{relatif prima} adalah bilangan-bilangan yang tidak memiliki faktor persekutuan satu sama lain. Contoh : $2,3,\text{dan} 5$ adalah relatif prima karena masing-masing tidak memiliki faktor persekutuan.
	\textbf{Contoh :} 
	\begin{enumerate}
		\item 17 dan 24 adalah relatif prima karena masing-masing tidak memiliki faktor persekutuan.
		\item 9 dan 8  adalah relatif prima karena masing-masing tidak memiliki faktor persekutuan
	\end{enumerate}
	\textbf{Kegunaan Relatif Prima:}
	\begin{enumerate}
		\item Relatif prima digunakan untuk menentukan nilai terkecil penyederhanaan akar\\
		Contoh : $\sqrt{54}=\sqrt{3^3 \times 2}$\\
				 $3\sqrt{3\times 2}=3\sqrt{6} $ karena 3 dan 2 adalah relatif prima maka tidak bisa diakar.
		\begin{enumerate}
			\item $ $
		\end{enumerate}
		\item Relatif prima digunakan untuk menentukan penyerdahanaan pecahan apakah dapat disederhanakan lagi atau tidak. Contoh:  $\dfrac{8}{21}$ tidak dapat disederhanakan karena $\dfrac{2^3}{3\times 7}$ adalah \textbf{relatif prima}
	\end{enumerate}
	
	\textbf{Model Soal UNBK 2017/2018}
	\begin{enumerate}
		\item Berapakah penyederhanaan dari $\dfrac{10}{66}$\\
		\textbf{Jawab:} $\dfrac{5\times \cancelto{1}{2}}{11\times \cancelto{1}{2} \times 3}=\dfrac{5}{33}$ \textbf{ingat relatif prima}
		\item Berapakah penyederhanaan dari $\sqrt{2\times 3\times 7\times 11\times 13\times ......}$\\
		\textbf{Jawab:} Tidak bisa disederhanakan. Ingat relatif prima.
		\item Berapakah penyederhanaan dari $\sqrt{150}$\\
		\textbf{Jawab :} 
		$\sqrt{5^2\times 3 \times 2}$, karena 3 dan 2 adalah relatif prima maka $5\sqrt{2\times 3}=5\sqrt{6}$
	\end{enumerate}
	
	
	
	
	
\section{FPB dan KPK}
	\paragraph{Faktor Persekutuan Terbesar}
	Adalah faktor yang dapat membagi bilangan-bilangan yang saling memiliki hubungan. Contoh : FPB dari $25,50,\text{dan} 35$ adalah 5 karena 5 dapat membagi ketiganya
\paragraph{Kelipatan Persekutuan Terkecil}
	Adalah bilangan komposit yang dapat dibagi oleh bilangan-bilangan yang berhubungan. Contoh KPK dari 25,50 adalah 50 karena 50 dapat dibagi oleh kedua bilangan.
\paragraph{Faktorisasi Prima} Adalah sebuah cara untuk memperoleh faktor-faktor prima dari sebuah bilangan. Contoh:\\
	\PrimeTree{36}
	\PrimeTree{81}
	\PrimeTree{20351}
	\PrimeTree{2234}
	
\section{Analisis BAB}
\section{Kelemahan Siswa}

\chapter{Perbandingan dan Skala}
\section{Perbandingan}
	\paragraph*{}
	Perbandingkan adalah sebuah cara untuk mendiskripsikan beberapa nilai yang berhubungan sehingga tercipta sebuah pecahan yang merepresentasikan nilai-nilai tersebut. Perlu diingat bahwa perbandingan adalah pecahan.
	\par
	Jika diketahui nilai A dan B maka perbandingan nilai A dan B adalah bentuk paling sederhana dari $\dfrac{A}{B}$ atau bisa ditulis sebagai $A:B$ Contoh :\\
	Jika terdapat A=250 dan B=150 maka perbandingan A dan B adalah $\dfrac{250}{150}=\dfrac{5}{3}$ jadi perbandingannya ditulis sebagai $5:3$
	\subsection{Perhitungan Perbandingan}
		\begin{enumerate}
			\item Salah satu nilai tidak diketahui maka untuk mengetahui nilai tersebut dapat dihitung dengan menggunakan
			\begin{center}
				$\dfrac{\text{perbandingan nilai tidak diketahui}}{\text{perbandingan nilai diketahui}}\times \text{nilai diketahui}$
			\end{center} 
			Jadi dalam contoh: A:B=3:7misal nilai A tidak diketahui dan nilai B diketahui bernilai 49 berapakah nilai A? \\
			\textbf{Jawab :} Jika B diketahui maka nilai A adalah $\dfrac{3}{7}\times 49=21$
			\item Jika terjadi Nilai perbandingan yang diketahui jumlah atau selisihnya semisal A : B, dan $A\mypm B$ diketahui maka,
			\begin{center}
				$\dfrac{\text{perbandingan nilai tidak diketahui}}{A\mypm B}\times \text{nilai diketahui}$
							\end{center}
			\textbf{Contoh 1} : Jika $A:B=2:3$ berapakah A jika diketahui selisih A dan B adalah 20?. \\
			\begin{center}
				$\dfrac{A}{B-A} \times 20=\dfrac{2}{3-2}\times 20=\dfrac{2}{1}\times 20=40$
			\end{center}
			\textbf{Contoh 2} : Jika $A:B=3:4$ dan jumlah A dan B adalah 35, berapakah selisih A dan B? 
			\begin{center}
				$\dfrac{B-A}{A+B} \times 35=\dfrac{4-3}{3+4}\times 35=\dfrac{1}{7}\times 35=5$
			\end{center}
			\item Jika diketahui perbandingan A dan B, dan diketahui AB serta terdapat nilai yang tidak diketahui maka, 
			\begin{center}
				$\dfrac{\text{nilai yang tidak diketahui}}{AB}\times \text{nilai AB}$
			\end{center}
			\textbf{Contoh :} A:B adalah 3:2 dan AB adalah 150, berapakah A?
			\begin{center}
				$\dfrac{3}{2\times 3}\times 150=\dfrac{3}{6}\times 150=75$
			\end{center}
		
		
		
		
		
		
			\end{enumerate}
			
			
			

\section{Skala}
	\paragraph*{}
	Skala adalah perbandingan yang menunjukkan panjang dalam realita sebenarnya dan panjang dalam peta yang digambarkan dalam bentuk pecahan perbandingan.\par
	Karakteristik Skala adalah sebagai berikut:
	\begin{enumerate}
		\item Digunakan dalam navigasi dan peta
		\item Hanya menghitung panjang, bukan dimensi geometri yang lain.
		\item Biasanya dinyatakan dalam centimeter, kecuali ditetapkan satuan lain
		\item Merupakan sebuah perbandingan.
		\item Merupakan perbandingan jarak pada peta dan jarak sebenarnya
	\end{enumerate}
	\subsection{Perhitungan Skala}
	Skala dihitung jika diketahui hal-hal sebagai berikut:
	\begin{enumerate}
		\item Mencari Skala, Diketahui : Panjang dalam Peta, Panjang Sebenarnya
		\begin{center}
			$\text{Skala}=\dfrac{\text{Panjang dalam Peta}}{\text{Panjang Sebenarnnya}}$
		\end{center}
		
		\textbf{Contoh :} Jika seseorang ingin membuat sebuah denah dan jarak A dan B pada denah tersebut adalah 5 cm, tapi jarak sebenarnya adalah 10 m.
		\begin{center}
			$\text{Skala}=\dfrac{5 cm}{1000 cm}=\dfrac{1}{200}$
		\end{center}
		Jadi Skalanya, $1:200$		
		\item Mencari Jarak sebenarnya, Diketahui : Panjang dalam peta dan Skala
		\begin{center}
			$\text{Jarak Sebenarnya}=\text{Jarak Pada Peta} \div \text{Skala}$
		\end{center}
		\textbf{Contoh:} Jika jarak pada peta adalah 2 cm, sedangkan skalanya adalah $1: 500$. Berapakah Jarak Sebenarnya?
		\begin{center}
			$\text{Jarak Sebenarnya}=2 \div \dfrac{1}{500}=2 \times 500=1000$cm
		\end{center}
		
		\item Mencari Jarak dalam Peta, Diketahui : Panjang dalam Sebenarnya dan Skala
		\begin{center}
			$\text{Jarak pada Peta}= \text{Panjang Sebenarnya} \times \text{Skala}$
		\end{center}
		Contoh : Jika diketahui jarak A, B adalah 40m dan Skala pada peta adalah 1:5000, berapakah jarak pada peta?
		\begin{center}
			$\text{Jarak pada Peta}= 40000 cm \times \dfrac{1}{5000} = 8 cm$
		\end{center}
	\end{enumerate}
\section{Perbandingan Lebih dari 2 pembanding}
	\paragraph*{}
	Perbandingan lebih dari 2 pembanding didefinisikan sebagai,\\
	\begin{center}
		$a_1:a_2:a_3:a_4:...:a_n$
	\end{center}
\section{Variasi}
\subsection{Variasi Berbanding Lurus}
	\paragraph{}
	Variasi lurus didefinisikan bahwa ketika sebuah nilai naik, maka nilai lain yang berhubungan dengan nilai tersebut naik secara proporsional.Misal, jika harga minyak naik, maka harga roti juga naik, berarti variasinya adalah berbanding lurus.
	\par
	Secara khusus variasi berbanding lurus memiliki persamaan sebagai berikut dengan $a=kc$ dan $k$ adalah konstanta. 
	\begin{center}
		
	\end{center}
	Secara umum variasi berbanding lurus didefinisikan sebagai,
	\begin{center}
		$a_1a_2a_3a_4...a_n=c$ misal, Jika $a_1$ naik dan semua variabel selain itu dianggap tetap, maka c juga ikut naik atau sebaliknya
	\end{center}
	Contoh : Sebuah pabrik memiliki pekerja sebanyak 50 orang saat ini dan menghasilkan laba sebanyak 100 \textdollar per hari. Jika pada tahun sebelumnya pabrik tersebut memiliki pekerja sebanyak 25 dengan laba 50 \textdollar. Berapakah laba yang akan diperoleh jika pada tahun berikutnya perusahaan memperkerjakan 100 pekerja?
	\par
	Jawab: Variasi yang didapatkan adalah variasi berbanding lurus. Jadi perhitungannya adalah sebagai berikut, $100=50k$ jadi $k=2$ atau dapat dihitung dengan $50=100k$ jadi $k=0.25$. Karena yang ditanyakan adalah laba tahun berikutnya, maka yang wajib digunakan adalah cara pertama dengan $k=2$. Jadi jika mempekerjakan 100 pekerja akan memperoleh laba sebanyak $2\cdot 100=200 \textdollar$
	
	
\subsection{Variasi Berbanding Terbalik}
	\paragraph*{}
	Variasi berbanding terbalik adalah variasi yang menunjukkan bahwa ketika sebuah nilai naik maka secara proporsional nilai yang lain turun.
	\par
	Contoh : Peristiwa naiknya permukaan air laut mengakibatkan produksi ikan menurun
	\par
	Secara khusus variasi berbanding lurus memiliki persamaan sebagai berikut,
	\begin{center}
		$y=\dfrac{k}{x}$ dengan $k$ adalah konstan,
	\end{center}
	Secara umum variasi berbanding terbalik memiliki persamaan sebagai berikut, 
	\begin{center}
		$y_1y_2y_3...y_n=\dfrac{k_1k_2...k_n}{x_1x_2x_3...x_n}$
	\end{center}
	\textbf{Contoh :} Proyek perbaikan jalan selesai selama 30 hari dengan pekerja sebanyak 15 orang. Setelah 6 hari pelaksanaan, proyek tersebut dihentikan selama 4 hari karena suatu hal. Jika kemampuan bekerja setiap orang sama dan agar proyek dapat selesai tepat waktu, pekerja tambahan yang diperlukan adalah?
	
	\textbf{Jawab:} Perbandingan tersebut adalah perbandingan berbanding terbalik, karena semakin sedikit waktu penyelesaian pengerjaan, semakin banyak pekerja yang harus mengerjakan.
	\begin{enumerate}
		\item Pada enam hari pengerjaan yaitu  $H-24$, pekerjaan dihentikan, pekerja sampai dengan hari itu adalah 15 orang.
		\item Empat hari kemudian, pekerjaan dilakukan lagi yaitu di hari $H-20$
		\item Jadi persamaannya adalah $15\times 24=20\times x$, $x=18$.
		\item Jadi pekerja tambahan yang diperlukan adalah $18-15=3$ orang. 
	\end{enumerate}
	
	
		
	

\chapter{Pangkat dan Akar}
\section{Bilangan Pangkat}
		\paragraph*{}
	Pangakat adalah bilangan hasil perkalian bilangan itu sendiri, yang dinyatakan dalam.
	\begin{center}
		$a^n=a_1\times a_2 \times a_3 \times ... a_n$
		dengan $a_1=a_2=a_3=...=a_n$
	\end{center}
	\subsection{Sifat Pangkat}
	\begin{multicols}{2}
		\begin{enumerate}
			\item $a^0=1$
			\item $a^1=a$
			\item $a^m\times b^n=b^{m+n}$
			\item $\dfrac{a^m}{a^n}=a^{m-n} $
			\item $(\dfrac{a}{b})^n=\dfrac{a^n}{b^n}$
			\item $a^{\frac{1}{n}}=\sqrt[n]{a}$
			\item $a^{\frac{m}{n}}=\sqrt[n]{a^m} $
			\end{enumerate}
	\end{multicols}
	\subsection{Contoh soal yang berkaitan dengan sifat Pangkat}
	\begin{enumerate}
		\item 
		
	\end{enumerate}
	
	
			
\section{Bilangan Akar }
		\paragraph*{}
	Akar adalah kebalikan dari pangkat yang merupakan penjabaran dari pangkat $a^{\frac{m}{n}}$.
\subsection{Sifat Akar}
	\begin{multicols}{2}
		\begin{enumerate}
			\item $\sqrt[n]{\dfrac{a}{b}}=\dfrac{\sqrt{a}}{\sqrt{b}}$
			\item $\sqrt[n]{a\times b \times c}=\sqrt[n]{a}+\sqrt[n]{b}+\sqrt[n]{c}$ \textbf{vice versa}
			\item Jika akar adalah akar genap semisal 2,4,... maka nilai hasil akar memiliki dua nilai yaitu positif dan negatif. Contoh :$\sqrt{4}=\mypm 2$
			\item Jika akar adalah akar ganjil, maka hanya memeliki satu nilai hasil. Nilai yang diakar dapat berupa angka negatif Contoh : $\sqrt[3]{-8}=-2$ 
			\item $b\sqrt[m]{a}=\sqrt[m]{ab^m}$
		\end{enumerate}
	\end{multicols}
\section{Rasionalisasi Bilangan Pecahan}
		\paragraph*{}
	Rasionalisasi adalah proses yang mengakibatkan nilai-nilai pembilang yang tidak rasional menjadi hilang dan diganti dengan nilai-nilai rasional.\par
	Peraturan rasionalisasi adalah sebagai berikut:
	\begin{enumerate}
		\item Jika terdapat $\dfrac{a}{\sqrt[n]{b}}$ maka bilangan rasionalnya didapatkan dengan $\dfrac{a}{\sqrt[n]{b}}\times \dfrac{\sqrt[n]{b}}{\sqrt[n]{b}}$
		\item Jika terdapat $\dfrac{a}{\sqrt[n]{b}+\sqrt[m]{c}}$ maka bilangan rasionalnya didapatkan dengan \\ $\dfrac{a}{\sqrt[n]{b}+\sqrt[m]{c}} \times\dfrac{\sqrt[n]{b}-\sqrt[m]{c}}{\sqrt[n]{b}-\sqrt[m]{c}} $
		\item Jika terdapat $\dfrac{a}{\sqrt[n]{b}-\sqrt[m]{c}}$ maka bilangan rasionalnya didapatkan dengan \\ $\dfrac{a}{\sqrt[n]{b}-\sqrt[m]{c}} \times\dfrac{\sqrt[n]{b}+\sqrt[m]{c}}{\sqrt[n]{b}+\sqrt[m]{c}} $
	\end{enumerate}
	Semua rasionalisasi bilangan pecahan yang memiliki akar mengikuti kaidah $(x+y)(x-y)=x^2-y^2$
\section{Penyederhanaan Bilangan Akar campuran}
		\input(./Bab3/PenyederhanaanAkar)
\section{Notasi Ilmiah}
\section{Modulus dan Pangkat}
\section{Akar Rekursif}


\chapter{Barisan dan Deret}
\section{Pola Barisan dan Deret}
\subsection{Pola Barisan Aritmatika}
\subsection{Pola Barisan Geometrik}

\section{Barisan Aritmatik}
\section{Deret Aritmatik}
\section{Barisan Geometrik}
\section{Deret Geometrik}
\section{Barisan Harmonik}
\section{Deret Harmonik}
\section{Deret Bertingkat}
\subsection{Aritmatik Bertingkat}
\subsection{Deret Geometrik Bertingkat}
\subsection{Deret Harmonik Bertingkat}

\chapter{Bentuk Aljabar}
\section{Nilai Absolut}
\section{Ekspresi Linier}
\section{Ekspresi Kuadrat}
\section{Ekspresi Polinomial Derajat $>2$}
\section{Ekspresi Pecahan Rasional}
\section{Ekspresi Imajiner}

\chapter{Persamaan}
\section{Persamaan Linier/Garis Lurus}
\section{Persamaan Kuadrat}
\section{Persamaan Pecahan Rasional}
\section{Persamaan Campuran}
\section{Persamaan Garis Lurus}
\section{Persamaan Lingkaran}

\chapter{Pertidaksamaan}
\section{Pertidaksamaan linier}
\section{Pertidaksamaan kuadrat}
\section{Pertidasamaan Absolut}
\section{Pertidaksamaan AM-GM-HM}

\chapter{Himpunan}
\section{Teori Himpunan}
\section{Himpunan dan Peluang}

\chapter{Fungsi dan Relasi}
\section{Relasi}
\section{Fungsi}
\section{Fungsi Linier}
\section{Fungsi Kuadrat}
\section{Fungsi Rasional}


\chapter{Aljabar Linier Dasar}
\section{Bentuk Aljabar Linier}
\section{Penyelesaian Aljabar Linier dua persamaan}
\section{Matriks}
\section{Penyelesaian Aljabar Linier lebih dari dua persamaan dengan Matriks}
\section{Pemrograman Linier}

\chapter{Geometri}
\section{Garis dan Sudut}
\section{Segitiga}
\section{Segi empat dan Segi $> 4$}
\section{Lingkaran}
\section{Bangun datar gabungan}

\section{Prisma}
\section{Limas}
\section{Bola}
\section{Bangun Ruang Gabungan}

\chapter{Trigonometri Dasar}
\section{Sudut dan Radian}
\section{Trigonometri Segitiga Kanan}
\section{Trigonometri Unit Lingkaran}
\section{Identitas Trigonometri}

\chapter{Kesebangunan dan Kongruensi}
\section{Kesebangunan Segitiga}
\section{Kongruensi Segitiga}
	\subsection{Kongruensi karena SSS}
	\subsection{Kongruensi karena ASA}
	\subsection{Kongruensi karena SAS}
	\subsection{Kongruensi karena HL}

\chapter{Statistika}
\section{Statistika Diskriptif}
\section{Penggambaran Datar}
\section{Mean/Rata-Rata}
\section{Median}
\section{Modus}

\chapter{Kombinatorik}
\section{Aturan Perkalian}
\paragraph*{}
	Jika suatu prosedur dapat dipecah menjadi beberapa kejadian (kejadian 1, kejadian 2, kejadian 3, dan seterusnya).
	Jika kejadian pertama dapat terjadi dengan n1 cara,
	kejadian kedua terjadi dengan n2 cara,
	kejadian ketiga dapat terjadi dengan n3 cara,\par
	...................
	kejadian ke-p dapat terjadi dengan np cara,
	maka kejadian - kejadian dengan urutan yang demikian dapat terjadi dengan
	$(n1\times n2\times n3\times...\times np)$ cara.

	

\section{Permutasi}
\section{Kombinasi}
\section{Aturan Penjumlahan}
\section{Peluang}







	
\end{document}
\documentclass[12pt,a4paper,draft,final,oneside,twoside,openright,openany]{article}
\usepackage[latin1]{inputenc}
\usepackage{amsmath}
\usepackage{amsfonts}
\usepackage{amssymb}
\usepackage{graphicx}
\usepackage[left=3.00cm, right=3.00cm, top=3.00cm, bottom=3.00cm]{geometry}
\author{Balya Rochmadi}		
\title{Soal Pengayaan Fisika SMP 4 HOTS}
\begin{document}
	\maketitle
	\begin{enumerate}
		\item Bayangkan terdapat satu lajur gelombang dengan awal gelombang adalah sebuah bukit dan diakhiri dengan sebuah lembah gelombang. Bila diantara awal dan akhir gelombang tersebut terdapat 3 bukit gelombang berapakah frekuensi dan periodenya. (panjang total gelombang adalah 4 m, dan ditempuh dalam waktu 12 sekon)
		\item Perhatikan gambar berikut!
		\begin{center}
			\includegraphics[width=0.7\linewidth]{/home/dila/Desktop/99Pcture/Selection_009}
		\end{center}
		Jika panjang gelombang pada gambar berikut adalah 250cm, berapakah frekuensi dan periodenya? 
		\item Perhatikan gambar berikut!
		\begin{center}
			\includegraphics[width=0.7\linewidth]{/home/dila/Desktop/99Pcture/1.png}
		\end{center}
		Jika waktu tempuhnya adalah 150 sekon, berapakah frekuensi, periode dan kecepatan gelombangnya?
		\item Perhatikan gambar berikut!
		\begin{center}
				\includegraphics[width=0.7\linewidth]{/home/dila/Desktop/99Pcture/Selection_010}
		\end{center}
		Jika Osilasi dari bandul tersebut adalah P-Q-R-Q-P-Q-R-Q dilakukan selama 12 detik, berapakah frekuensi dan periodenya?
		\item Sebuah pegas berosilasi dengan pola rapat-renggang-rapat-renggang-rapat, jika waktu yang dibutuhkan untuk osilasi tersebut adalah 20 sekon, berapakah frekuensi dan periodenya?
		
	\end{enumerate}

		
\end{document}
\documentclass[12pt,a4paper,draft,final,oneside,twoside,openright,openany]{article}
\usepackage[latin1]{inputenc}
\usepackage{amsmath}
\usepackage{amsfonts}
\usepackage{amssymb}
\usepackage{graphicx}
\usepackage[left=3.00cm, right=3.00cm, top=3.00cm, bottom=3.00cm]{geometry}
\author{Balya Rochmadi}		
\title{Soal Pengayaan Fisika SMA 3 HOTS\\ Mapel : Gaya Pada bidang tertentu\\Codename: Tesla}
\begin{document}
	\maketitle
	\Large
	\begin{enumerate}
		
		\item Sebuah balok 10 kg diam di atas lantai datar. Koefisien gesekan statis $\mu_s = 0,4$ dan koefisien gesek kinetis $\mu_k$ = 0,3. Tentukanlah gaya gesekan yang bekerja pada balok jika gaya luar F diberikan dalam arah horizontal sebesar?
		\item  Sebuah balok bermassa 20 kg berada di atas lantai mendatar kasar. $\mu_s$ = 0,6 dan $\mu_k$ = 0,3. Kemudian balok ditarik gaya sebesar F mendatar. g = $10 m/s^2$. Tentukan gaya gesek yang dirasakan balok dan percepatan balok jika:
		\begin{enumerate}
			\item F = 100 N
			\item F = 140 N
		\end{enumerate}
		\item Sebuah balok yang awalnya diam dikenakan gaya 0,500 N pada sudut $20^o$, seperti pada gambar, Berapakah besarkah percepatan yang terjadi jika koefisien gaya gesek adalah
		\begin{center}
			\includegraphics[width=0.4\linewidth]{../../Desktop/99Pcture/Selection_016}
		\end{center}
		\begin{enumerate}
			\item $\mu_s=0,6$ dan $\mu_k=0,5$
			\item $\mu_s=0,4$ dan $\mu_k=0,3$ 
		\end{enumerate}
		\item Sebuah box berisi pinguin dengan berat 80N,diam, di sebuah turunan,dengan sudut turunan sebesar $20^o$ dari horizontal.Antara turunan dan box tersebut terdapat koefisien gesek statis sebesar 0,25 dan koefisien gesek kinetis sebesar 0,15. 
		\begin{center}
				\includegraphics[width=0.4\linewidth]{../../Desktop/99Pcture/Selection_017}
		\end{center}
				
		 \begin{enumerate}
		 	\item  Berapakah gaya minimal, yang dapat diaplikasikan pada box tersebut sehingga pinguin tersebut tidak jatuh tergelincir.
		 	\item  Berapakah gaya minimum yang dapat diapliasikan agar box tersebut bergerak maju
		 	\item Berapakah gaya minimum yang dapat diaplikasikan agar pinguin tersebut dapat berjalan maju dengan kecepatan konstan?
		 \end{enumerate}
		 
		
		\item  Sebuah peti kayu bermassa 60 kg didorong oleh seseorang dengan gaya 800 N ke atas sebuah truk menggunakan papan yang disandarkan membentuk bidang miring. Ketinggian bak truk tempat papan bersandar adalah 2 m dan panjang papan yang digunakan adalah 2,5 m. Jika peti bergerak ke atas dengan percepatan $2 m/s^2$ dan g = $10 m/s^2$ maka tentukan koefisien gesek kinetis antara peti kayu dengan papan.
	\end{enumerate}
\end{document}
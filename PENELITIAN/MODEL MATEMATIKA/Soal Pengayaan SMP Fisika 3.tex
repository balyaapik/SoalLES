\documentclass[12pt,a4paper,draft,final,oneside,twoside,openright,openany]{article}
\usepackage[latin1]{inputenc}
\usepackage{amsmath}
\usepackage{amsfonts}
\usepackage{amssymb}
\usepackage{graphicx}
\usepackage[left=3.00cm, right=3.00cm, top=3.00cm, bottom=3.00cm]{geometry}
\author{Balya Rochmadi}
\title{Soal Pengayaan Fisika SMP 5 HOTS\\ Mapel : Listrik Statis\\Codename : Faraday\\ \small closed book}
\begin{document}
	\Large

	\maketitle
		\noindent\makebox[\linewidth]{\rule{\paperwidth}{0.4pt}}
		\paragraph{
			\Large Petunjuk}
		
		\begin{enumerate}
			\item Rumus Coloumb	: $F_c=k\dfrac{q_1q_2}{r^2}$, $k=9\times 10^9$
			\item Energi yang dibutuhkan untuk mengalirkan arus listrik $W=q\Delta V$ dan  adalah $\Delta V=\text{beda daya/voltase}$
			\item Gaya Medan Listrik : $F_{e}=qE$ ($q=\text{muatan listrik}, E=\text{kuat medan listrik}$)
			\item Muatan diapit diantara dua muatan listrik : $\Sigma F_c=0$
			\item Kuat medan Listrik : $E=k(\dfrac{Q}{r^2})$
			\item Kuat medan Listrik diapit $E=k(\dfrac{Q_1}{r_1^2}+\dfrac{Q_2}{r_2^2})$
		\end{enumerate}
		\noindent\makebox[\linewidth]{\rule{\paperwidth}{0.4pt}}
	\begin{enumerate}
		\item Dua buah partikel memliki daya elektrik sebesar $5Q$ dan $6P$, jika daya keduanya diubah menjadi 10Q dan -2P sedangkan jaraknya menjadi 4r, berapakah perubahan gaya elektriknya?
		\item Sebutkan masing-masing (a)tipe muatan elektrik pada masing masing benda dibawah ini! Dan Jawab Pertanyaan (b) Kemanakah elektron berpindah !
		\begin{enumerate}
			\item Kaca digosok Kain Sutera
			\item Mistar Plastk digosok kain Wol
			\item Sisir digosok rambut manusia
			\item Balon digosok kain wool
			\item Ebonit digosok kain wool
		\end{enumerate}
		\item Dua buah muatan masing-masing q1 = 6 $\mu C$ dan q2 = 12 $\mu C$ terpisah sejauh 30 cm. Tentukan besar gaya yang terjadi antara dua buah muatan tersebut, gunakan tetapan $k = 9 \times 10^9$ dalam satuan standar!
		\item Dua buah muatan listrik memiliki besar yang sama yaitu 6 $\mu C$. Jika gaya coulomb yang terjadi antara dua muatan tadi adalah  $1,6 N$, tentukan jarak pisah kedua muatan tersebut!
		\item Dua buah benda bermuatan listrik tidak sejenis, tarik-menarik dengan gaya sebesar F. Jika jarak kedua muatan didekatkan menjadi 1/3 kali semula, maka gaya tarik-menarik antara kedua muatan menjadi...F
		\item Dua buah partikel bermuatan listrik didekatkan pada jarak tertentu hingga timbul gaya sebesar F. Jika besar muatan listrik partikel pertama dijadikan 1/2 kali muatan semula dan besar muatan partikel kedua dijadikan 8 kali semula maka gaya yang timbul menjadi....
		\item  Titik A dan titik B mempunyai beda potensial listrik sebesar 12 volt. Tentukan energi yang diperlukan untuk membawa muatan listrik 6 $\mu C$ dari satu titik A ke titik B!
		\item Titik A terletak dalam medan listrik. Kuat medan listrik di titik A= 0,5 $\mu C$. Jika di titik A diletakkan benda bermuatan listrik 0,25 C, maka pada benda tersebut bekerja gaya Coulomb sebesar ?
		\item Perhatikan gambar berikut!
			\begin{center}
				\includegraphics[width=0.7\linewidth]{../../Desktop/99Pcture/ML2}
			\end{center}
		Berapakah besarnya medan listrik A?
		\item Perhatikan gambar berikut!
		\begin{center}
		\includegraphics[width=0.7\linewidth]{../../Desktop/99Pcture/Listrik-Statis-1}
		\end{center}
		Berapakah besarnya muatan P?
	\end{enumerate}
\end{document}
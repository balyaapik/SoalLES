\documentclass[12pt,a4paper,draft,final,oneside,twoside,twocoloumn,openright,openany]{article}
\usepackage[latin1]{inputenc}
\usepackage{amsmath}
\usepackage{amsfonts}
\usepackage{amssymb}
\usepackage{graphicx}
\usepackage[left=3.00cm, right=3.00cm, top=3.00cm, bottom=3.00cm]{geometry}
\author{Balya Rochmadi}
\title{Rumus Singkat SMP IPA FISIKA HOTS}
\begin{document}
	\begin{enumerate}
		\item Kinematika
		\begin{enumerate}
			\item Gerak Lurus
				\begin{enumerate}
					\item Gerak Lurus $a=0$
					\item $S=Vt$
				\end{enumerate}
			\item Gerak Lurus Berubah beraturan
				\begin{enumerate}
					\item $v_t=v_0+at$
					\item $S_t=v_0t+1/2at^2$
					\item $v_t^2=v_0^2+2aS_t$
				\end{enumerate}
			\item Gerak Jatuh Bebas
				\begin{enumerate}
					\item $h_{max}=\dfrac{v_0^2}{2g}$
					\item Kecepatan ketika $t$ detik : $v_t=v_0+gt$
					\item Kecepatan pada ketnggian $h$ meter: $v_t^2=2gh$
					\item Kecepatan sesaat sebelum jatuh : $v_t^2=2gh_{max}$
				\end{enumerate}
			\item Benda dilempar keatas tegak lurus
				\begin{enumerate}
					\item Dilempar keatas $v_t=v_0-gt$, Jatuh kebawah $v_t=gt$ atau $v_t^2=2gh$
				\end{enumerate}
			\item Gerak parabola\ proyektil
				\begin{enumerate}
					\item Jarak horizontal $\Delta x=(v_0 \cos{\theta_0})t$
					\item Jarak vertikal $\Delta y= (v_0 \sin{\theta_0})t-gt$
					\item Kecepatan horizontal $v_x=\dfrac{\Delta x}{t}$
					\item Kecepatan vertikal $v_y=(v_0 \sin{\theta_0})-gt$
					\item Kecepatn vertikal terhadap ketinggian $v_y^2=(v_0 \sin{\theta_0})^2+2g(\Delta y)$
			\item Vektor integral dan hubungannya
				\begin{enumerate}
					\item Vektor posisi $\vec{r}=r_x\hat{i}+r_y\hat{j}+r_z\hat{k}$
					\item Vektor kecepatan $\vec{v}=v_x\hat{i}+v_y\hat{j}+v_z\hat{k}$, atau $r=\dfrac{d\vec{r}}{dt}$, atau $  $
				\end{enumerate}
				\end{enumerate}
		\end{enumerate}
	\end{enumerate}
\end{document}
\documentclass[12pt,a4paper,draft,final,oneside,twoside,openright,openany]{article}
\usepackage[latin1]{inputenc}
\usepackage{amsmath}
\usepackage{amsfonts}
\usepackage{multicol}
\usepackage{amssymb}
\usepackage{graphicx}
\usepackage[left=3.00cm, right=3.00cm, top=3.00cm, bottom=3.00cm]{geometry}
\author{Balya Rochmadi}		
\title{Soal Pengayaan Fisika SMP 5 HOTS\\Mapel : Fisika,Listrik Dinamis, Hukum Kirchoff\\Closed Book}
\begin{document}
	\maketitle
	\Large
	\noindent\makebox[\linewidth]{\rule{\paperwidth}{0.4pt}}

	\paragraph{Petunjuk: }
				\begin{enumerate}
					\item Hukum Ohm : $I=\dfrac{V}{R}$
					\item Power(Dissipasi Elektrik) : $P=iE$ atau $P=VI$ atau $P=I^2R$
					\item Power(Dissipasi Elektrik) setiap waktu : $P=iEt$ atau $P=VIt$ atau $P=I^2Rt$
					\item Resistor Pararel $\dfrac{1}{R}=\Sigma \dfrac{1}{R_n}$, Resistor seri= $R_s=\Sigma R_n$
					\item Hukum Kirchoff 1, Percabangan $I_{in}=I_{out}$
					\item Hukum Kirchoff 2, Loop $\Sigma E -\Sigma IR= 0$
				\end{enumerate}

	
		
	\noindent\makebox[\linewidth]{\rule{\paperwidth}{0.4pt}}
	
	
		
	\begin{enumerate}
	\item Berapa banyak hambatan yang diperlukan untuk membatasi arus hingga 1,5 mA jika voltasenya adalah 6 V?
	\item Sebuah peralatan listrik dipasang pada tegangan listrik sebesar 12 volt dan arus yang mengalir adalah sebesar 750 mA. Hitunglah besarnya energi listrik yang dibutuhkan dalam jangka waktu 1 menit !
	\item Sebuah elemen pemanas listrik yang digunakan untuk memanaskan air memiliki hambatan 24 ohm dihubungkan dengan sumber tegangan 240 V. Berapa energi listrik yang dihasilkan oleh pemanas tersebut selama 1 menit ?. 
	\item Jika energi yang kita perlukan dalam memindahkan muatan listrik 4 Coulomb dari titik A ke titik B adalah 20 Joule. Hitunglah perbedaan potensial antara titik A dan B ? 
	\item Sebuah lampu bertuliskan 40 W/110 V dinyalakan selama 10 menit. Berapakah arus listrik dan energi listrik yang diperlukan ?
	\item Suatu rangkaian listrik yang memiliki hambatan 4 ohm dialiri suatu arus listrik 8 Ampere selama 30 menit. Tentukan energi yang digunakan dalam satuan Joule, Kalori dan kWh ? 
	\item Sebuah keluarga menggunakan daya listrik 1200 watt selama 400 jam. Jika harga listrik 1 kWh = Rp 400,- maka berapa biaya yang harus dikeluarkan keluarga tersebut 
	\item Jika sebuah lampu pijar tertulis 220 V/100 W. Jika lampu tersebut digunakan selam 10 jam dalam sehari. Hitunglah berapa biaya listrik yang harus dibayarkan jika 1 kWh = Rp 1000,- dalam 1 bulan (1 bulan = 30 hari) ? 
	\item Sebuah mesin sepeda motor melakukan usaha sebesar 10.000 joule. Jika daya motor itu 2000 watt, hitunglah waktu yang digunakan ?
	\item Berapakah hambatan sebuah kawat besi yang memiliki panjang 0,5 cm, dan luas $1,3 x 10^{-2}$ $cm^2$. Jika hambatan jenis kawat besi tersebut $9,7 x 10^{-8}$ Ohmmeter? 
	\item Berapakah arus yang mengalir di $I_3$ dan $I_4$?
		\begin{center}
			\includegraphics[width=0.7\linewidth]{/home/dila/Desktop/99Pcture/Selection_025}
		\end{center}
	\item Berapakah arus yang mengalir di $I_1$, $I_3$, $I_4$, $I_5$?
			\begin{center}
				\includegraphics[width=0.7\linewidth]{/home/dila/Desktop/99Pcture/Selection_026}
			\end{center}
	\item Perhatikan gambar !
		\begin{center}
			\includegraphics[width=0.5\linewidth]{/home/dila/Desktop/99Pcture/Selection_027}
		\end{center}
		Jawab pertanyaan berikut ini
	\begin{enumerate}
		\item Berapakah total Resistensi sirkuit tersebut?
		\item Berapakah arus total sirkuit?
		\item Berapakah $I_1$ dan $1_2$?
		\item Berapakah Daya/Power yang diserap oleh $R_1$ dan $R_2$?
		\item Berpakah Daya total yang dimiliki sirkuit tersebut?
	\end{enumerate}
	\item Diberikan rangkaian seperti gambar dibawah.
		Jika R1 = 50 $\Omega$, R2 = 60 $\Omega$, R3 = 40 $\Omega$, R4 = 20 $\Omega$, R5 = 30 $\Omega$
		\begin{center}
			\includegraphics[width=0.5\linewidth]{/home/dila/Desktop/99Pcture/we1}
		\end{center}
	Gambarkanlah hambatan penggantinya!
	\item Perhatikan gambar berikut! Jika $\varepsilon_1=12V \text{dan} \varepsilon_2=6V$ sedangkan $R_1=4,0\Omega \text{dan} R_2=8.0 \Omega$ maka,
	 \begin{center}
		\includegraphics[width=0.4\linewidth]{/home/dila/Desktop/99Pcture/Selection_024}
	\end{center}
	\begin{enumerate}
		\item Berapakah dissipasi daya R1 dan R2?
		\item Berapakah transfer daya E1 dan E2?
		\item Baterai manakah yang mengisi daya?
	\end{enumerate}
	
	\end{enumerate}
\end{document}

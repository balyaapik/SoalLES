\documentclass[12pt,a4paper,draft,final,oneside,twoside,openright,openany]{article}
\usepackage[latin1]{inputenc}
\usepackage{amsmath}
\usepackage{amsfonts}
\usepackage{amssymb}
\usepackage{graphicx}
\usepackage[left=3.00cm, right=3.00cm, top=3.00cm, bottom=3.00cm]{geometry}
\author{Balya Rochmadi}		
\title{Soal Pengayaan Fisika SMA 2 HOTS}
\begin{document}
	\maketitle
	\begin{enumerate}
		\item Sebuah bola kasti bergerak pada bidang xy. Koordinat x dan y bola tersebut dinyatakan oleh persamaan x = 18t dan $y=4t-5t^2$ dengan x dan y dalam meter serta t dalam sekon. Tuliskan persamaan vektor posisi r dengan menggunakan vektor satuan i dan j.
		\item Persamaan kecepatan sebuah partikel adalah $\vec{v} = (v_xi+ v_yj)$ m/s dengan $v_x = 2t m/s$ dan $v_y = (1+ 3t2) m/s$. Pada saat awal, partikel berada di titik pusat koordinat (0,0).
		Tentukan percepatan rata-rata dalam selang waktu $t = 0$ sampai $t = 2$ sekon.Nyatakan persamaan umum vektor percepatan sebagai fungsi waktu.Tentukan posisi partikel pada saat $t = 2$ sekon.
		Tentukan besar dan arah percepatan dan kecepatan pada saat $t = 2$ sekon.
		\item Dari soal berikut, tentukan resultan posisi dari grafik kecepatan tersebut!
		\begin{center}
		\includegraphics[width=0.4\linewidth]{../../Desktop/99Pcture/unnamed}
		\end{center}
		\item Sebuah kurva percepatan linier terhadap waktu dengan persamaan $a=1,5t+2$ dengan batas-batas waktu $8s$ dan $12s$, berapakah kecepatan dari grafik tersebut?
		\item Berikut ini sebuah bola dilempar dengan kecepatan awal $20m/s$ dan sudut awal adalah $30^o$ dengan dan dilempar pada ketinggian 45 m.\\
		\begin{center}
		\includegraphics[width=0.7\linewidth]{../../Desktop/99Pcture/Selection_008}
		\end{center}
		Berapakah :
		\begin{enumerate}
			\item Berapa lama bola tersebut hingga menyentuh tanah?
			\item Berapakah jarak horizontal dari bola tersebut?
			\item Berapakah kecepatan bola sesaat sebelum menyentuh tanah?
		\end{enumerate}
\end{enumerate}
\end{document}
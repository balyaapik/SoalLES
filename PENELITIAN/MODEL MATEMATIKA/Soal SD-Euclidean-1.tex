\documentclass[12pt,a4paper,draft,final,oneside,twoside,openright,openany]{article}
\usepackage[latin1]{inputenc}
\usepackage{amsmath}
\usepackage{multicol}
\usepackage{amsfonts}
\usepackage{amssymb}
\usepackage{pdfpages}
\usepackage{graphicx}
\usepackage[left=3.00cm, right=3.00cm, top=3.00cm, bottom=3.00cm]{geometry}
\author{Balya Rochmadi}		
\title{Soal Pengayaan SD 1 HOTS\\ Mapel : Matematika\\ Sub: Dimensi 2 dan 3 \\Codename: Euclidean\\ \small waktu pengerjaan : 1 jam.}
\begin{document}
	\maketitle
	\Large
	\begin{enumerate}
		\item Pak Amir memiliki sebidang tanah seperti pada gambar di bawah.
		Tanah tersebut akan ditanami padi, biaya
		pembelian benih $Rp3.000,00/m^2$ .
		Biaya untuk pembelian benih yang
		diperlukan untuk menanami
		padi seluruhnya adalah ....
		\begin{center}
		\includegraphics[width=0.5\linewidth]{../../Desktop/99Pcture/Selection_019}
		\end{center}
		\item Perhatikan gambar berikut!
		Gambar di bawah adalah topi yang bagian
		luarnya ditempel dengan kertas manila,
		berapa luas kertas minimum yang diperlukan?
		\begin{center}
			\includegraphics[width=0.5\linewidth]{../../Desktop/99Pcture/Selection_020}
		\end{center}
		\item Pak Tauhid memiliki kawat panjangnya 5m akan digunakan untuk membuat kerangka prisma
		yang alasnya berbentuk segitiga sama kaki dengan panjang rusuk alas 13 cm, 13 cm, dan 14 cm
		dan tinggi prisma 20 cm. Maka banyak kerangka prisma yang dapat dibuat oleh Pak Tauhid
		adalah .....
		\item Luas Juring pada sebuah lingkaran yang sudut pusatnya $120^o$ dan diameter 42 cm
		adalah ....
		\item ABCH dan DEFG adalah persegi panjang.
		Jika AB = 20 cm, BC = 16 cm, dan DE = GF = 8 cm,
		maka keliling daerah yang diarsir adalah ....
			\begin{center}
				\includegraphics[width=0.5\linewidth]{../../Desktop/99Pcture/Selection_021}
			\end{center}
		\item Sebuah akuarium berbentuk balok dengan ukuran panjang 100 cm, lebar 40 cm dan tinggi 50 cm.
		Jika ke dalam akuarium dimasukkan batu-batu hiasan, tinggi air naik 6 cm. Volume batu-batu itu adalah?
		\item Sebuah aula berbentuk balok dengan ukuran panjang 9 meter, lebar 6 meter, dan tinggi 5 meter.
		Dinding bagian dalamnya akan dicat dengan biaya Rp50.000,00 per meter persegi. Seluruh biaya
		pengecatan aula tersebut adalah ....
		\item Perhatikan gambar! Sebuah benda
		berbentuk tabung dengan tutup dan alasnya
		berbentuk belahan bola, jika diameter bola 14 cm ,
		maka volumenya adalah ....
			\begin{center}
				\includegraphics[width=0.5\linewidth]{../../Desktop/99Pcture/Selection_022}
			\end{center}
		\item Amir ingin membuat 5 buah sekop pembuang sampah yang
		terbuat dari seng, seperti gambar sebelah kanan. Jika harga
		seng adalah Rp 5,00 per $cm^2$ dan harga tangkai pegangan Rp
		2000,00 per buah maka biaya yang diperlukan adalah
			\begin{center}
				\includegraphics[width=0.5\linewidth]{../../Desktop/99Pcture/Selection_023}
			\end{center}
		\item Sebuah taman berbentuk persegi panjang yang panjangnya 42 m dan lebar 18 m. Di
		sekeliling taman ditanami pohon cemara dengan jarak antar pohon 6 m. Jika harga
		pohon Rp.65.000,00 per batang, maka biaya yang diperlukan untuk membeli pohon
		seluruhnya adalah ....
	\end{enumerate}
		\includepdf[pages=-]{../../Desktop/99Pcture/PSD1}
\end{document}
\documentclass[12pt,a4paper,draft,final,oneside,twoside,openright,openany]{article}
\usepackage[latin1]{inputenc}
\usepackage{amsmath}
\usepackage{amsfonts}
\usepackage{multicol}
\usepackage{amssymb}
\usepackage{graphicx}
\usepackage[left=3.00cm, right=3.00cm, top=3.00cm, bottom=3.00cm]{geometry}
\author{Balya Rochmadi}		
\title{Soal Pengayaan Matematika SMP HOTS \\CC:301118 \\
	Closed Book}
\begin{document}
		\maketitle
		\Large
		\noindent\makebox[\linewidth]{\rule{\paperwidth}{0.4pt}}
		\paragraph{Petunjuk:}
		\begin{enumerate}
			\item Penjabaran Fungsi Polinomial
			\begin{enumerate}
				\item $(a+b)^2=a^2+b^2+2ab$
				\item $(a+b+c)^2=a^2+b^2+c^2+2ab+2bc+2ac$
				\item $(a+b+c+d+e+...)=a^2+b^2+c^2+d^2+...+(2ab+2ac+2ad+...)$
				\item $a^2-b^2=(a+b)(a-b)$
			\end{enumerate}	
			\item Fungsi kuadrat berbentuk akar:
			\begin{enumerate}
				\item JIka terdapat $\sqrt{(a+b)+2\sqrt{ab}}=\sqrt{a}+\sqrt{b}$ dengan $a>0$ dan $b>0$
			\end{enumerate}
			\item Akar Kuadrat Berurutan dengan urutan minimal 4:
			\begin{enumerate}
				\item Kasus empat urutan:
				$\sqrt{(a)(a+1)(a+2)(a+3)+1}=a^2+3a+1$
			\end{enumerate}
			\item Teorema Sophie Germain 
			\begin{center}
				$a^4+4b^4=(a^2-2ab+2b^2)(a^2+2ab+2b^2)$
			\end{center}
			\item Akar-akar kuadrat
			\begin{center}
				$x_1,x_2=\dfrac{-b \pm \sqrt{b^2-4(a)(c)}}{2a}$
			\end{center}
			\item Pangkat tiga (koreksi cc:281118)
				\begin{enumerate}
					\item $(a+b)^3=a^3+b^3+3ab(a+b)$
					\item $a^3+b^3=(a+b)^3-3ab(a+b)$
					\item $a^3+b^3=(a+b)(a^2-ab+b^2)$
					\item $a^3-b^3=(a-b)(a^2+ab+b^2)$
				\end{enumerate}
			\item Identitas Aljabar Lanjutan
				\begin{enumerate}
					\item $a^2-b^2=(a+b)(a-b)$
					\item $a^4-b^4=(a-b)(a^2+ab+b^2)$ 
					\item $a^4-b^4=(a-b)(a^3+a^2b+ab^2+b^3)$
					\item $a^5-b^5=(a-b)(a^4+a^3b+a^2b^2+ab^3+b^4)$
				\end{enumerate}
			\item Faktorial
			\begin{center}
				$n!=(n)(n-1)(n-2)(n-3)...(2)(1)$
			\end{center}
			\item Koefisien Binomial
				\begin{center}
					$\binom {n}{k}=\dfrac {n!}{k!(n-k)!}$
				\end{center}
			\item Ekspansi Binomial
			\begin{center}
				$(x+y)^{n}={n \choose 0}x^{n}y^{0}+{n \choose 1}x^{n-1}y^{1}+{n \choose 2}x^{n-2}y^{2}+\cdots +{n \choose n-1}x^{1}y^{n-1}+{n \choose n}x^{0}y^{n}$\\
				
				atau
				
				$(x+y)^{n}=\sum _{k=0}^{n}{n \choose k}x^{n-k}y^{k}=\sum _{k=0}^{n}{n \choose k}x^{k}y^{n-k}$
			\end{center}
			
			\item Identitas Aljabar umum
			\begin{center}
				$a^n-b^n=(a-b)(a^n+a^{n-1}b+a^{n-2}b^{2}+...+ab^{n-2}+b^{n-1})$
			\end{center}
			
			\begin{center}
				
				$a^n+b^n=(a+b)(a^n-a^{n-1}b+a^{n-2}b^{2}-...-ab^{n-2}+b^{n-1}$
			\end{center}
			\item Sifat Keterbagian identitas Aljabar Umum
			\begin{enumerate}
				\item apabila $x\neq y$ dan $n\in N^+$ maka $x-y$ habis membagi $x^n-y^n$
				\item apabila $x\neq y$ dan $n\in N^+$ maka $x+y$ habis membagi $x^n+y^n$
			\end{enumerate}
			\item Deret dan Seri
			\begin{enumerate}
				\item Deret Aritmatika
					\begin{center}
						$U_n=a+(n-1)b$ (Suku ke-n)\\
						$S_n=\dfrac{n}{2}(2a+(n-1)b)$ (Jumlah suku sampai n-suku)
						$U_t=\dfrac{U_1+U_n}{2}$ (Suku Tengah)
					\end{center}
				\item Deret Geometri
					\begin{center}
						$r=\dfrac{U_{n}}{U_{n-1}}$ (rasio)\\
						$U_n=ar^{n-1}$ (Suku ke-n)\\
						$S_n=\dfrac{a(r^n-1)}{r-1}$ (Jumlah suku ke-n jika $r>1$)\\
						$S_n=\dfrac{a(1-r^n)}{1-r}$ (Jumlah suku ke-n jika $0<r<1$)
					\end{center}
			\end{enumerate}
			\item Persamaan Linier
				\begin{enumerate}
					\item Bentuk\\
					$y=mx+c$
					\item Gradien\\
					$m=\dfrac{y_2-y_1}{x_2-x_1}$\\
					\\
					Karakteristik:
			
					\begin{enumerate}
						\item Jika $m<0$ maka persamaan linier tersebut definit negatif (kurva miring ke kiri)
						\item Jika $m=0$ maka persamaan linier tersebut konstan
						\item Jika $m>0$ maka persamaan linier tersebut definit positif (kurva miring ke kanan)
					\end{enumerate} 
					\item Persamaan Linier baru\\
					$y-y_0=m(x-x_0)$
				\end{enumerate}
			\item Pertidaksamaan Linier
				\begin{enumerate}
					\item Pertidaksamaan Linier\\
					\begin{center}
						$ax+b\geq 0$ atau $ax+b\leq 0$
					\end{center}
					Domain : $R^+$\\
					Range  : $R^+$ atau $R^-$
					
					\item Pertidaksamaan Rasional Satu variabel
					
					\begin{center}
						$\dfrac{ax+b}{cx+d}=\dfrac{f(x)}{g(x)}\geq 0$ atau
						$\dfrac{ax+b}{cx+d}=\dfrac{f(x)}{g(x)}\leq 0$ 
					\end{center}
					Domain : $R^+_- \text{dan }g(x)\neq0$\\
					Range : $R^+$ atau $R^-$
					\item Pertidaksamaan Kuadrat
					\begin{enumerate}
						\item Bentuk 
						$ax^2+bx+c\geq 0$ atau $ax^2+bx+c\leq 0$
						\item Cara penyelesaiaan:
						\begin{enumerate}
							\item Faktorkan
							\item Hasil faktor digunakan sebagai titik uji himpunan
						\end{enumerate}
					\end{enumerate}
					
					\item Pertidaksamaan Rasional Kuadrat
					\begin{enumerate}
						\item Bentuk :\\
						$\dfrac{ax^2+bx+c}{dx^2+ex+f}\geq 0$ atau $\dfrac{ax^2+bx+c}{dx^2+ex+f}\leq 0$
						\item Cara penyelesaiaan:
						\begin{enumerate}
							\item Faktorkan
							\item Sederhanakan kalau bisa
							\item Hasil faktor digunakan sebagai titik uji himpunan
						\end{enumerate}
					\end{enumerate}
							
					\item Pertidaksamaan Universal
					\begin{center}
						$\sum_{n=0}^{k} x^2_k \gneq 0$ dengan persamaan berlaku jika $x=0$
					\end{center}
					\item Pertidaksamaan Rata-Rata Kuadratik, Rata-rata Artimatika, Rata-rata Geometrika, Rata-rata Harmonik.
					$\sqrt{\frac{x_1^2+\cdots+x_n^2}{n}} \ge\frac{x_1+\cdots+x_n}{n}\ge\sqrt[n]{x_1\cdots x_n}\ge\frac{n}{\frac{1}{x_1}+\cdots+\frac{1}{x_n}}$
					
					\item Maxima dan Minuma menggunakan AM-GM
					\item Pertidaksamaan Segitiga
					\begin{center}
						 $(a+b)^2\leq a^2+b^2$
					\end{center}
					
				\end{enumerate}
				
			\item Fungsi dan Relasi
				\begin{enumerate}
					\item Semua fungsi adalah relasi, sebagian relasi adalah fungsi.
					\item Fungsi sebagai pemetaan $x$ ke $y$
					\begin{center}
						$f:x\rightarrow y$
					\end{center}
							
					\item Banyaknya pemetaan $A$ ke $B$ adalah 
					\begin{center}
						$n(B)^{n(A)}$
					\end{center}
					
					\item Fungsi berkorespondensi satu-satu berarti memiliki pemetaan daerah domain dan range yang sifatnya tunggal.
					
					\item Fungsi dapat memiliki banyak x untuk satu y tetapi tidak boleh memiliki banyak y untuk satu x.
					
					\item Notasi daerah fungsi
					\begin{center}
						\begin{quote}
							Untuk setiap domain x bilangan real, menghasilkan range y bilangan real dengan anggota tersebut.
							\begin{center}
								$f(x):{x|x\in R \& f(x)\in R}$
							\end{center}
						\end{quote}
					\end{center}
					
					\item Fungsi linier mememiliki persamaan 
						\begin{center}
							$f(x)=mx+c$
						\end{center}
					\item Fungsi kuadrat memiliki persamaan yaitu
						\begin{center}
							$f(x)=ax^2+bx+c$
						\end{center}
					\item \textit{Bodiless Function}
						\begin{center}
							Yaitu fungsi yang tidak memiliki badan fungsi, misalkan
							$f(x)=f(x+y)+f(x)$
						\end{center}
					\item \textit{Recursive Function}
						\begin{center}
							Yaitu fungsi yang memanggil dirinya sendiri
							Misalkan 
							$f(x)=c+f(x)$\\
							*fungsi rekursif memiliki bentuk yang mirip dengan fungsi tanpa badan
						\end{center}
						\item Fungsi Rasional
							\begin{center}
								Adalah fungsi yang berbentuk pecahan semisal,
								$f(x)=\dfrac{ax+b}{cx+d}$
							\end{center}
						\item Dekomposisi Fungsi Rasional Parsial
							\begin{center}
								Semisal terdapat $f(x)=\dfrac{2}{x}+\dfrac{5x}{x+1}$
								\\
								maka akan terdapat konstanta A dan B sehingga 
								
								$f(x)=\dfrac{2}{x}+\dfrac{5x}{x+1}=\dfrac{2(x+1)+5x(x)}{x(x+1)}=\dfrac{A}{x}+\dfrac{B}{x+1} $
								
								maka,\\
								
								$\dfrac{5x^2+4x+1}{x^2+x}=\dfrac{A}{x}+\dfrac{B}{x+1}$
								
								sehingga\\
								
								$5x^2+4x+1=A(x+1)+B(x)=Ax+A+Bx=(A+B)x+A$\\
								
								Jika $x=0$ maka\\
								
								$5(0)^2+4(0)+1=A(0+1)+B(0)=A$ dan $A=1$\\
								
								
							\end{center}
						
				\end{enumerate}
				
			

		
		\end{enumerate}
		
		\noindent\makebox[\linewidth]{\rule{\paperwidth}{0.4pt}}
		\paragraph{Kerjakan Soal Berikut! Ingat Waktu!}
		\begin{enumerate}
			\item 
		\end{enumerate}
		
\end{document}
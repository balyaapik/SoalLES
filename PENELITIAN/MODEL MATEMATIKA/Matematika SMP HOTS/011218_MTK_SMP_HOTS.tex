\documentclass[12pt,a4paper,draft,final,oneside,twoside,openright,openany]{article}
\usepackage[latin1]{inputenc}
\usepackage{amsmath}
\usepackage{amsfonts}
\usepackage{multicol}
\usepackage{amssymb}
\usepackage{graphicx}
\usepackage[left=3.00cm, right=3.00cm, top=3.00cm, bottom=3.00cm]{geometry}
\author{Balya Rochmadi}		
\title{Soal Pengayaan Matematika SMP HOTS \\CC:011218 \\
	Closed Book}
\begin{document}
		\maketitle
		\Large
		\noindent\makebox[\linewidth]{\rule{\paperwidth}{0.4pt}}
				\paragraph{Petunjuk:}
		\begin{enumerate}
			\item Penjabaran Fungsi Polinomial
			\begin{enumerate}
				\item $(a+b)^2=a^2+b^2+2ab$
				\item $(a+b+c)^2=a^2+b^2+c^2+2ab+2bc+2ac$
				\item $(a+b+c+d+e+...)=a^2+b^2+c^2+d^2+...+(2ab+2ac+2ad+...)$
				\item $a^2-b^2=(a+b)(a-b)$
			\end{enumerate}	
			\item Fungsi kuadrat berbentuk akar:
			\begin{enumerate}
				\item JIka terdapat $\sqrt{(a+b)+2\sqrt{ab}}=\sqrt{a}+\sqrt{b}$ dengan $a>0$ dan $b>0$
			\end{enumerate}
			\item Akar Kuadrat Berurutan dengan urutan minimal 4:
			\begin{enumerate}
				\item Kasus empat urutan:
				$\sqrt{(a)(a+1)(a+2)(a+3)+1}=a^2+3a+1$
			\end{enumerate}
			\item Teorema Sophie Germain 
			\begin{center}
				$a^4+4b^4=(a^2-2ab+2b^2)(a^2+2ab+2b^2)$
			\end{center}
			\item Akar-akar kuadrat
			\begin{center}
				$x_1,x_2=\dfrac{-b \pm \sqrt{b^2-4(a)(c)}}{2a}$
			\end{center}
			\item Pangkat tiga (koreksi cc:281118)
				\begin{enumerate}
					\item $(a+b)^3=a^3+b^3+3ab(a+b)$
					\item $a^3+b^3=(a+b)^3-3ab(a+b)$
					\item $a^3+b^3=(a+b)(a^2-ab+b^2)$
					\item $a^3-b^3=(a-b)(a^2+ab+b^2)$
				\end{enumerate}
			\item Identitas Aljabar Lanjutan
				\begin{enumerate}
					\item $a^2-b^2=(a+b)(a-b)$
					\item $a^4-b^4=(a-b)(a^2+ab+b^2)$ 
					\item $a^4-b^4=(a-b)(a^3+a^2b+ab^2+b^3)$
					\item $a^5-b^5=(a-b)(a^4+a^3b+a^2b^2+ab^3+b^4)$
				\end{enumerate}
			\item Faktorial
			\begin{center}
				$n!=(n)(n-1)(n-2)(n-3)...(2)(1)$
			\end{center}
			\item Koefisien Binomial
				\begin{center}
					$\binom {n}{k}=\dfrac {n!}{k!(n-k)!}$
				\end{center}
			\item Ekspansi Binomial
			\begin{center}
				$(x+y)^{n}={n \choose 0}x^{n}y^{0}+{n \choose 1}x^{n-1}y^{1}+{n \choose 2}x^{n-2}y^{2}+\cdots +{n \choose n-1}x^{1}y^{n-1}+{n \choose n}x^{0}y^{n}$\\
				
				atau
				
				$(x+y)^{n}=\sum _{k=0}^{n}{n \choose k}x^{n-k}y^{k}=\sum _{k=0}^{n}{n \choose k}x^{k}y^{n-k}$
			\end{center}
			
			\item Identitas Aljabar umum
			\begin{center}
				$a^n-b^n=(a-b)(a^n+a^{n-1}b+a^{n-2}b^{2}+...+ab^{n-2}+b^{n-1})$
			\end{center}
			
			\begin{center}
				
				$a^n+b^n=(a+b)(a^n-a^{n-1}b+a^{n-2}b^{2}-...-ab^{n-2}+b^{n-1}$
			\end{center}
			\item Sifat Keterbagian identitas Aljabar Umum
			\begin{enumerate}
				\item apabila $x\neq y$ dan $n\in N^+$ maka $x-y$ habis membagi $x^n-y^n$
				\item apabila $x\neq y$ dan $n\in N^+$ maka $x+y$ habis membagi $x^n+y^n$
			\end{enumerate}
			\item Deret dan Seri
			\begin{enumerate}
				\item Deret Aritmatika
					\begin{center}
						$U_n=a+(n-1)b$ (Suku ke-n)\\
						$S_n=\dfrac{n}{2}(2a+(n-1)b)$ (Jumlah suku sampai n-suku)
						$U_t=\dfrac{U_1+U_n}{2}$ (Suku Tengah)
					\end{center}
				\item Deret Geometri
					\begin{center}
						$r=\dfrac{U_{n}}{U_{n-1}}$ (rasio)\\
						$U_n=ar^{n-1}$ (Suku ke-n)\\
						$S_n=\dfrac{a(r^n-1)}{r-1}$ (Jumlah suku ke-n jika $r>1$)\\
						$S_n=\dfrac{a(1-r^n)}{1-r}$ (Jumlah suku ke-n jika $0<r<1$)
					\end{center}
			\end{enumerate}
			\item Persamaan Linier
				\begin{enumerate}
					\item Bentuk\\
					$y=mx+c$
					\item Gradien\\
					$m=\dfrac{y_2-y_1}{x_2-x_1}$\\
					\\
					Karakteristik:
			
					\begin{enumerate}
						\item Jika $m<0$ maka persamaan linier tersebut definit negatif (kurva miring ke kiri)
						\item Jika $m=0$ maka persamaan linier tersebut konstan
						\item Jika $m>0$ maka persamaan linier tersebut definit positif (kurva miring ke kanan)
					\end{enumerate} 
					\item Persamaan Linier baru\\
					$y-y_0=m(x-x_0)$
				\end{enumerate}
			\item Pertidaksamaan Linier
				\begin{enumerate}
					\item Pertidaksamaan Linier\\
					\begin{center}
						$ax+b\geq 0$ atau $ax+b\leq 0$
					\end{center}
					Domain : $R^+$\\
					Range  : $R^+$ atau $R^-$
					
					\item Pertidaksamaan Rasional Satu variabel
					
					\begin{center}
						$\dfrac{ax+b}{cx+d}=\dfrac{f(x)}{g(x)}\geq 0$ atau
						$\dfrac{ax+b}{cx+d}=\dfrac{f(x)}{g(x)}\leq 0$ 
					\end{center}
					Domain : $R^+_- \text{dan }g(x)\neq0$\\
					Range : $R^+$ atau $R^-$
					\item Pertidaksamaan Kuadrat
					\begin{enumerate}
						\item Bentuk 
						$ax^2+bx+c\geq 0$ atau $ax^2+bx+c\leq 0$
						\item Cara penyelesaiaan:
						\begin{enumerate}
							\item Faktorkan
							\item Hasil faktor digunakan sebagai titik uji himpunan
						\end{enumerate}
					\end{enumerate}
					
					\item Pertidaksamaan Rasional Kuadrat
					\begin{enumerate}
						\item Bentuk :\\
						$\dfrac{ax^2+bx+c}{dx^2+ex+f}\geq 0$ atau $\dfrac{ax^2+bx+c}{dx^2+ex+f}\leq 0$
						\item Cara penyelesaiaan:
						\begin{enumerate}
							\item Faktorkan
							\item Sederhanakan kalau bisa
							\item Hasil faktor digunakan sebagai titik uji himpunan
						\end{enumerate}
					\end{enumerate}
							
					\item Pertidaksamaan Universal
					\begin{center}
						$\sum_{n=0}^{k} x^2_k \gneq 0$ dengan persamaan berlaku jika $x=0$
					\end{center}
					\item Pertidaksamaan Rata-Rata Kuadratik, Rata-rata Artimatika, Rata-rata Geometrika, Rata-rata Harmonik.
					$\sqrt{\frac{x_1^2+\cdots+x_n^2}{n}} \ge\frac{x_1+\cdots+x_n}{n}\ge\sqrt[n]{x_1\cdots x_n}\ge\frac{n}{\frac{1}{x_1}+\cdots+\frac{1}{x_n}}$
					
					\item Maxima dan Minima menggunakan AM-GM
					\item Pertidaksamaan Segitiga
					\begin{center}
						$a^2+b^2\geq (a+b)^2$  atau
						$\mid a\mid + \mid b\mid \geq \mid a+b \mid$
					\end{center}
					
					\item Pertidaksamaan Linier Dua Variabel
					\begin{enumerate}
						\item Bentuk \\
							$ax+by\geq 0$ atau $ax+by\leq 0$
						\item Pada soal biasanya memakai soal cerita.
					\end{enumerate}
					\item Pertidaksamaan Nilai Absolut
					\begin{enumerate}
						\item Nila Absolut
						\begin{center}
							$\mid a\mid \geq 0$\\
							Jika $a<0$ maka $\mid a \mid= -a$\\
							Jika $a\geq 0$ maka $\mid a\mid=a$
						\end{center}
					\end{enumerate}
				\end{enumerate}
		\noindent\makebox[\linewidth]{\rule{\paperwidth}{0.4pt}}
		
		\end{enumerate}
		\paragraph{Kerjakan Soal Berikut! Ingat Waktu!}
		\begin{enumerate}
		
		\item \textbf{(OSN-E)}Berapakah nilai minimum dari $x(x-3)^3$ jika $0<x<1$?
		\item \textbf{(Konseptual)} Buktikanlah Pertidaksamaan Berikut ini!
		\begin{enumerate}
			\item $x^2+xy+y^2\geq 0$
			\item $x^2-xy+y^2\geq 0$
			\item $0\leq\dfrac{b-a}{1-ab}\leq 1$ dengan $0\leq b\leq 1$
			\item $0\leq ab^2-ba^2\leq \dfrac{1}{4}$
		\end{enumerate}
		\item \textbf{(IMO-1996\#1)} Misalkan a,b,c adalah bilangan bulat positif dengan $abc=1$, buktikanlah bahwa :
		\begin{center}
			$\dfrac{ab}{a^5+b^5+ab}+\dfrac{bc}{b^5+c^5+bc}+\dfrac{ac}{a^5+b^5+ac}\leq 1$
		\end{center}
		
		\item \textbf{(HOTS/UNBK-K)}Panjang dan lebar persegi panjang ABCD masing-masing 30 cm dan 20 cm. Bagian tepi-tepi persegi panjang itu dipotong selebar x cm sehingga diperoleh persegi panjang PQRS. Perhatikan gambar di bawah ini. Keliling persegi panjang PQRS tidak lebih dari 52 cm. Tentukan batas-batas panjang pemotongan yang dilakukan.
		
		\item\textbf{(HOTS/UNBK-K)}  Pak Irvan memiliki sebuah mobil box pengangkut barang dengan daya angkut tidak lebih dari 500 kg. Berat pak Irvan adalah 60 kg dan dia akan mengangkut kotak barang yang setiap kotak beratnya 20 kg.
		\begin{enumerate}
			\item Tentukan banyak kotak maksimum yang dapat diangkut oleh pak Irvan dalam sekali pengangkutan!
			\item Jika pak Irvan akan mengangkut 115 kota, paling sedikit berapa kali kotak itu akan terangkut semua?
		\end{enumerate}
		
		\item \textbf{(HOTS/UNBK-K)} Jumlah dua bilangan tidak lebih dari 120. Jika bilangan kedua adalah 10 lebihnya dari bilangan pertama, maka tentukan batas nilai untuk bilangan pertama
		
		\item \textbf{(HOTS/UNBK-K)}  Suatu model kerangka balok terbuat dari kawat dengan ukuran panjang $(x + 5)$ cm, lebar $(x-2)$ cm, dan tinggi x cm.
		\begin{enumerate}
			\item Tentukan model matematika dari persamaan panjang kawat yang diperlukan dalam x.
			\item Jika panjang kawat yang digunakan seluruhnya tidak lebih dari 132 cm, tentukan ukuran maksimum balok tersebut?
		\end{enumerate}
		\item  \textbf{(UNBK-M)} Untuk membuat cover (kulit buku) sebuah buku diperlukan kertas berbentuk persegi panjang, dengan selisih panjang dan lebarnya adalah 7cm, serta memiliki luas 450$cm^2$. Hitunglah panjang dan lebar cover (kulit buku) buku itu !
		\item\textbf{(UNBK-M)} Seutas kawat yang panjangnya 16 meter akan dibuat persegi panjang. Berapa ukuran persegi panjang itu agar luasnya maksimum?
		\end{enumerate}
		
\end{document}
\documentclass[12pt,a4paper,draft,final,oneside,twoside,openright,openany]{article}
\usepackage[latin1]{inputenc}
\usepackage{amsmath}
\usepackage{amsfonts}
\usepackage{multicol}
\usepackage{amssymb}
\usepackage{graphicx}
\usepackage[left=3.00cm, right=3.00cm, top=3.00cm, bottom=3.00cm]{geometry}
\author{Balya Rochmadi}		
\title{Soal Pengayaan Matematika SMP HOTS \\CC:281118 \\
	Closed Book}
\begin{document}
		\maketitle
		\Large
		\noindent\makebox[\linewidth]{\rule{\paperwidth}{0.4pt}}
		\paragraph{Petunjuk:}
		\begin{enumerate}
			\item Penjabaran Fungsi Polinomial
			\begin{enumerate}
				\item $(a+b)^2=a^2+b^2+2ab$
				\item $(a+b+c)^2=a^2+b^2+c^2+2ab+2bc+2ac$
				\item $(a+b+c+d+e+...)=a^2+b^2+c^2+d^2+...+(2ab+2ac+2ad+...)$
				\item $a^2-b^2=(a+b)(a-b)$
			\end{enumerate}	
			\item Fungsi kuadrat berbentuk akar:
			\begin{enumerate}
				\item JIka terdapat $\sqrt{(a+b)+2\sqrt{ab}}=\sqrt{a}+\sqrt{b}$ dengan $a>0$ dan $b>0$
			\end{enumerate}
			\item Akar Kuadrat Berurutan dengan urutan minimal 4:
			\begin{enumerate}
				\item Kasus empat urutan:
				$\sqrt{(a)(a+1)(a+2)(a+3)+1}=a^2+3a+1$
			\end{enumerate}
			\item Teorema Sophie Germain 
			\begin{center}
				$a^4+4b^4=(a^2-2ab+2b^2)(a^2+2ab+2b^2)$
			\end{center}
			\item Akar-akar kuadrat
			\begin{center}
				$x_1,x_2=\dfrac{-b \pm \sqrt{b^2-4(a)(c)}}{2a}$
			\end{center}
			\item Pangkat tiga
				\begin{enumerate}
					\item $(a+b)^3=a^3+b^3+3ab(a+b)$
					\item $a^3+b^3=(a+b)^3-3ab(a+b)$
					\item $a^3+b^3=(a+b)(a^2-ab+b^2)$
					\item $a^3-b^3=(a-b)(a^2+ab+b^2)$
				\end{enumerate}
\noindent\makebox[\linewidth]{\rule{\paperwidth}{0.4pt}}
		
		\end{enumerate}
		\paragraph{Kerjakan Soal Berikut! Ingat Waktu!}
		\begin{enumerate}
			\item Jumlah dua bilangan adalah 2 dan hasil kalinya adalah 5. Tentukan jumlah kubik dari kedua bilangan tersebut!
			\item Bilangan x,y memenuhi $x^3+y^3=126$ dan $x^2-xy+y^2=21$. Tentukan nilai $x^2+y^2$
			\item Jika $x^2+x+1=0$ maka tentukan nilai dari $(x^3+\dfrac{1}{x^3})^3$ adalah?
			\item Misalkan $x$ dan $y$ adalah bilangan real sehingga\\ $(x^2-y^2)(x^2-2xy+y^2)=3$
			dan $x-y=1$, tentukan nilai $xy$!
			\item Tentukan nilai dari $a^6+a^{-6}$ jika $a^2+a^{-2}=4$  !
		
		\end{enumerate}
\end{document}
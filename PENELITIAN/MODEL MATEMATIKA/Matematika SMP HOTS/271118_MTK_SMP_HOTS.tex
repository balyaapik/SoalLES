\documentclass[12pt,a4paper,draft,final,oneside,twoside,openright,openany]{article}
\usepackage[latin1]{inputenc}
\usepackage{amsmath}
\usepackage{amsfonts}
\usepackage{multicol}
\usepackage{textcomp}
\usepackage{amssymb}
\usepackage{pdfpages}
\usepackage{graphicx}
\usepackage[left=3.00cm, right=3.00cm, top=3.00cm, bottom=3.00cm]{geometry}
\author{Balya Rochmadi}		
\title{Soal Pengayaan Matematika SMP HOTS \\Mapel : Persamaan Polinomial dan Teori Bilangan \\Closed Book}
\begin{document}
	\maketitle
	\Large
		\noindent\makebox[\linewidth]{\rule{\paperwidth}{0.4pt}}
	\paragraph{Petunjuk:}
	\begin{enumerate}
		\item Penjabaran Fungsi Polinomial
			\begin{enumerate}
				\item $(a+b)^2=a^2+b^2+2ab$
				\item $(a+b+c)^2=a^2+b^2+c^2+2ab+2bc+2ac$
				\item $(a+b+c+d+e+...)=a^2+b^2+c^2+d^2+...+(2ab+2ac+2ad+...)$
				\item $a^2-b^2=(a+b)(a-b)$
			\end{enumerate}	
		\item Fungsi kuadrat berbentuk akar:
			\begin{enumerate}
				\item JIka terdapat $\sqrt{(a+b)+2\sqrt{ab}}=\sqrt{a}+\sqrt{b}$ dengan $a>0$ dan $b>0$
			\end{enumerate}
		\item Akar Kuadrat Berurutan dengan urutan minimal 4:
			\begin{enumerate}
				\item Kasus empat urutan:
				$\sqrt{(a)(a+1)(a+2)(a+3)+1}=a^2+3a+1$
			\end{enumerate}
		\item Teorema Sophie Germain 
		\begin{center}
			$a^4+4b^4=(a^2-2ab+2b^2)(a^2+2ab+2b^2)$
		\end{center}
		\item Akar-akar kuadrat
		\begin{center}
			$x_1,x_2=\dfrac{-b \pm \sqrt{b^2-4(a)(c)}}{2a}$
		\end{center}
		
	\end{enumerate}
		\noindent\makebox[\linewidth]{\rule{\paperwidth}{0.4pt}}
	\paragraph{Kerjakan dengan teliti dan hati-hati. Ingat Waktu!}
	\begin{enumerate}
		\item Jika $x+\dfrac{1}{x} = 3$ berapakah $x^4+\dfrac{1}{x^4}$?
		\item Bilangan a dan b memenuhi $\dfrac{2}{a+b}=\dfrac{1}{a}+\dfrac{1}{b}$, berapakah $\dfrac{a^2}{b^2}$?
		\item Jika $f(1) = 1$ and $f(n+1) = \dfrac{2f(n) + 1}{2}$, maka temukan $f(237)$.
		\item Temukan jumlah semua bilangan bulat $n$ agar $n^2-19n+99$ menjadi kuadrat sempurna!
		\item Hitunglah nilai dari $\dfrac{2014^4+4\times 2013^4}{2013^2+4027^2}-\dfrac{2012^4+4\times 2013^4}{2013^2+4025^2}$ tanpa kalkulator!
	\end{enumerate}
	
	
	
	
\end{document}
\documentclass[12pt,a4paper,draft,final,oneside,twoside,openright,openany]{article}
\usepackage[latin1]{inputenc}
\usepackage{amsmath}
\usepackage{amsfonts}
\usepackage{multicol}
\usepackage{amssymb}
\usepackage{graphicx}
\usepackage[left=3.00cm, right=3.00cm, top=3.00cm, bottom=3.00cm]{geometry}
\author{Balya Rochmadi}		
\title{Soal Pengayaan Matematika SMP HOTS \\CC:291118 \\
	Closed Book}
\begin{document}
		\maketitle
		\Large
		\noindent\makebox[\linewidth]{\rule{\paperwidth}{0.4pt}}
		\paragraph{Petunjuk:}
		\begin{enumerate}
			\item Penjabaran Fungsi Polinomial
			\begin{enumerate}
				\item $(a+b)^2=a^2+b^2+2ab$
				\item $(a+b+c)^2=a^2+b^2+c^2+2ab+2bc+2ac$
				\item $(a+b+c+d+e+...)=a^2+b^2+c^2+d^2+...+(2ab+2ac+2ad+...)$
				\item $a^2-b^2=(a+b)(a-b)$
			\end{enumerate}	
			\item Fungsi kuadrat berbentuk akar:
			\begin{enumerate}
				\item JIka terdapat $\sqrt{(a+b)+2\sqrt{ab}}=\sqrt{a}+\sqrt{b}$ dengan $a>0$ dan $b>0$
			\end{enumerate}
			\item Akar Kuadrat Berurutan dengan urutan minimal 4:
			\begin{enumerate}
				\item Kasus empat urutan:
				$\sqrt{(a)(a+1)(a+2)(a+3)+1}=a^2+3a+1$
			\end{enumerate}
			\item Teorema Sophie Germain 
			\begin{center}
				$a^4+4b^4=(a^2-2ab+2b^2)(a^2+2ab+2b^2)$
			\end{center}
			\item Akar-akar kuadrat
			\begin{center}
				$x_1,x_2=\dfrac{-b \pm \sqrt{b^2-4(a)(c)}}{2a}$
			\end{center}
			\item Pangkat tiga (koreksi cc:281118)
				\begin{enumerate}
					\item $(a+b)^3=a^3+b^3+3ab(a+b)$
					\item $a^3+b^3=(a+b)^3-3ab(a+b)$
					\item $a^3+b^3=(a+b)(a^2-ab+b^2)$
					\item $a^3-b^3=(a-b)(a^2+ab+b^2)$
				\end{enumerate}
			\item Identitas Aljabar Lanjutan
				\begin{enumerate}
					\item $a^2-b^2=(a+b)(a-b)$
					\item $a^4-b^4=(a-b)(a^2+ab+b^2)$ 
					\item $a^4-b^4=(a-b)(a^3+a^2b+ab^2+b^3)$
					\item $a^5-b^5=(a-b)(a^4+a^3b+a^2b^2+ab^3+b^4)$
				\end{enumerate}
			\item Faktorial
			\begin{center}
				$n!=(n)(n-1)(n-2)(n-3)...(2)(1)$
			\end{center}
			\item Koefisien Binomial
				\begin{center}
					$\binom {n}{k}=\dfrac {n!}{k!(n-k)!}$
				\end{center}
			\item Ekspansi Binomial
			\begin{center}
				$(x+y)^{n}={n \choose 0}x^{n}y^{0}+{n \choose 1}x^{n-1}y^{1}+{n \choose 2}x^{n-2}y^{2}+\cdots +{n \choose n-1}x^{1}y^{n-1}+{n \choose n}x^{0}y^{n}$\\
				
				atau
				
				$(x+y)^{n}=\sum _{k=0}^{n}{n \choose k}x^{n-k}y^{k}=\sum _{k=0}^{n}{n \choose k}x^{k}y^{n-k}$
			\end{center}
			
			\item Identitas Aljabar umum
			\begin{center}
				$a^n-b^n=(a-b)(a^n+a^{n-1}b+a^{n-2}b^{2}+...+ab^{n-2}+b^{n-1})$
			\end{center}
			
			\begin{center}
				
				$a^n+b^n=(a+b)(a^n-a^{n-1}b+a^{n-2}b^{2}-...-ab^{n-2}+b^{n-1}$
			\end{center}
			\item Sifat Keterbagian identitas Aljabar Umum
			\begin{enumerate}
				\item apabila $x\neq y$ dan $n\in N^+$ maka $x-y$ habis membagi $x^n-y^n$
				\item apabila $x\neq y$ dan $n\in N^+$ maka $x+y$ habis membagi $x^n+y^n$
			\end{enumerate}
			\item Deret dan Seri
			\begin{enumerate}
				\item Deret Aritmatika
					\begin{center}
						$U_n=a+(n-1)b$ (Suku ke-n)\\
						$S_n=\dfrac{n}{2}(2a+(n-1)b)$ (Jumlah suku sampai n-suku)
						$U_t=\dfrac{U_1+U_n}{2}$ (Suku Tengah)
					\end{center}
				\item Deret Geometri
					\begin{center}
						$r=\dfrac{U_{n}}{U_{n-1}}$ (rasio)\\
						$U_n=ar^{n-1}$ (Suku ke-n)\\
						$S_n=\dfrac{a(r^n-1)}{r-1}$ (Jumlah suku ke-n jika $r>1$)\\
						$S_n=\dfrac{a(1-r^n)}{1-r}$ (Jumlah suku ke-n jika $0<r<1$)
					\end{center}
			\end{enumerate}
			\item Pertidaksamaan Linier
				\begin{enumerate}
					\item Pertidaksamaan linier biasa mengahasilkan himpunan
					\item Pertidaksamaan Universal
					\begin{center}
						$\sum_{n=0}^{k} x^2_k>=0$ dengan persamaan berlaku jika $x=0$
					\end{center}
					\item Pertidaksamaan Rata-Rata Aritmatika, Rata-rata Geometrika, Rata-rata harmonik
					\item 
				\end{enumerate}
			
\noindent\makebox[\linewidth]{\rule{\paperwidth}{0.4pt}}
		
		\end{enumerate}
		\paragraph{Kerjakan Soal Berikut! Ingat Waktu!}
		\begin{enumerate}
		
		\item  Selembar karton berbentuk persegi panjang akan dibuat kotak tanpa tutup dengan cara membuang persegi seluas $3\times 3 cm^2$ di masing-masing pojoknya. Apabila panjang alas kotak 2 cm lebih dari lebarnya dan volum kotak itu adalah 105 $cm^3$. Tentukanlah panjang dan lebar alas kotak tersebut.
		\item Jumlah dua buah bilangan sama dengan 30. Jika hasil kali kedua bilangan itu sama dengan 200, tentukanlah bilangan tersebut.
		\item Berapakah termin ke 4 dari $(2x+y)^{10}$?
		\item Dari sistem persamaan $x^2+6y=7;y^2-4z=1, \text{dan} z^2-10x=-46$ maka nilai dari $x+y+z$ sama dengan...
		\item Temukan rumus jumlah dari $S=1+2+3+4+...+n$
		\item Temukan rumus dari $S=1-2+3-4+...+(-1)^{n-1}n$
		\item Temukan rumus dari $S=n+(n+3)+(n+6)+(n+8)+...+4n$
		\item Jika $x^2+y^2-4x+6y+13=0$ maka nilai dari $x+y+2015=...$
		\item Jika x,y bilangan real yang memenuhi persamaan $x^2-6x+\sqrt{y-2x}=-9$ maka nilai dari $5x-y+2015=...$
		\item Bilangan real x,y memenuhi persamaan sebagai berikut $(4030x-2015)^2+\sqrt{2x^2+10y-2015}=0$ berapakah $x+y$!
		\end{enumerate}
\end{document}
\paragraph{Operasi Pecahan Campuran (Desimal dan Biasa)}
	Siswa diminta untuk menentukan nilai dari pecahan campuran yang telah terlebih dahulu diacak oleh komputer, dalam hal ini masalah pecahan yang dimaksud berupa soal cerita.
	\par
	\begin{enumerate}	
	\item \textbf{Operasi Penjumlahan:} Dalam operasi penjumlahan ada dua hal penting yang harus diperhatikan. Pertama, ketika kita akan menjumlahkan pecahan dengan penyebutnya yang telah sama, maka kita dapat secara langsung menjumlahkan pembilang-pembilangnya saja.
	
	Contoh 1:\\
	$\frac{2}{5}+\frac{6}{5}=\frac{2+6}{5}=\frac{8}{5}$
	
	Jika Penyebut belum sama maka disamakan terlebih dahulu \\
	Contoh 2:\\
	
	$\frac{2}{3}+\frac{4}{5}=\frac{2\times5}{3\times5}+\frac{4\times3}{5\times3}=\frac{10}{15}+\frac{12}{15}=\frac{22}{15}$
	
	Agar lebih mudahnya, perhatikan formula berikut ini:\\
	\\
	$\frac{a}{b}+\frac{c}{b}=\frac{a+c}{b}$
	\\
	$\frac{a}{b}+\frac{c}{d}=\frac{ad+bc}{bd}$
	\\
	\textbf{SIFAT PENJUMLAHAN PADA PECAHAN}
	\begin{enumerate}
		\item Sifat Komutatif Penjumlahan\\
		$\frac{a}{b}+\frac{c}{b}=\frac{c}{b}+\frac{a}{b}$\\
		\item Sifat Asosiatif Penjumlahan\\
		$\left(\frac{a}{b}+\frac{c}{b}\right)+\frac{d}{b}=\frac{a}{b}+\left(\frac{c}{b}+\frac{d}{b}\right)$
	\end{enumerate}
	
	
	\textbf{PENGURANGAN PADA PECAHAN}
	
	Perhatikan contoh berikut.
	
	$\frac{5}{8}-\frac{4}{8}=\frac{1}{8}
	\frac{3}{7}-\frac{2}{8}=\frac{3\times8}{7\times8}-\frac{2\times7}{8\times7}=\frac{24}{56}-\frac{14}{56}=\frac{24-14}{56}=\frac{10}{56}$
	
	Agar lebih mudahnya, perhatikan kedua formula berikut ini:
	
	$\frac{a}{b}-\frac{c}{b}=\frac{a-c}{b}$\\
	
	$\frac{a}{b}-\frac{c}{d}=\frac{ad-bc}{bd}$
	
	Pengurangan Pecahan Tidak Bersifat Komutatif
	
	$\frac{a}{b}-\frac{c}{b}\neq\frac{c}{b}-\frac{a}{b}$
	
	Perkalian Pecahan
	
	Pada operasi perkalian pecahan berlaku pengerjaan-pengerjaan seperti berikut ini.
	
	
	SIFAT OPERASI PERKALIAN PADA PECAHAN
	
	Sifat 1 : Sifat Komutatif Perkalian
	
	$\frac{a}{b}\times\frac{c}{d}=\frac{c}{d}\times\frac{a}{b}$
	
	Sifat 2 : Sifat Asosiatif Perkalian
	
	$\left(\frac{a}{b}\times\frac{c}{d}\right)\times\frac{p}{q}=\frac{a}{b}\times\left(\frac{c}{d}\times\frac{p}{q}\right)$
	
	Sifat 3 : Sifat Distributif Perkalian terhadap Penjumlahan
	
	$\frac{a}{b}\times\left(\frac{c}{d}+\frac{p}{q}\right)=\left(\frac{a}{b}\times\frac{c}{d}\right)+\left(\frac{a}{b}\times\frac{p}{q}\right)$
	
	Sifat 4 : Sifat Distributif Perkalian terhadap Pengurangan
	
	$\frac{a}{b}\times\left(\frac{c}{d}-\frac{p}{q}\right)=\left(\frac{a}{b}\times\frac{c}{d}\right)-\left(\frac{a}{b}\times\frac{p}{q}\right)$
	
	Sifat 5 : Sifat Perkalian Pecahan dengan Bilangan 1
	
	$\frac{a}{b}\times1=\frac{a}{b}$
	
	Sifat 6 : Sifat Perkalian Pecahan dengan Bilangan 0
	
	$\frac{a}{b}\times0=0\times\frac{a}{b}=0$
	
	Sifat 7 : Sifat Urutan Pecahan
	
	$\frac{a}{b}>\frac{c}{d}\Longleftrightarrow ad>bc$
	
	
	
	Pembagian Pecahan
	
	Dalam operasi pembagian pecahan, sembarang $\frac{a}{b} dan \frac{c}{d} dengan b\neq0 dan d\neq 0$
	
	\end{enumerate}
 
	 	\opmul[displayshiftintermediary=all]{123}{456}
	 	\oplput(1,3){(this is $123 \times 6$.)}
	 	\oprput(-5,2){$+$}
	 	\oplput(1,2){(this is $123 \times 5$, shifted one position to the left.)}
	 	\oplput(1,1){(this is $123 \times 4$, shifted two positions to the left.)}
	
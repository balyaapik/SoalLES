\paragraph{Faktor Persekutuan Terbesar}
	Adalah faktor yang dapat membagi bilangan-bilangan yang saling memiliki hubungan. Contoh : FPB dari $25,50,\text{dan} 35$ adalah 5 karena 5 dapat membagi ketiganya
\paragraph{Kelipatan Persekutuan Terkecil}
	Adalah bilangan komposit yang dapat dibagi oleh bilangan-bilangan yang berhubungan. Contoh KPK dari 25,50 adalah 50 karena 50 dapat dibagi oleh kedua bilangan.
\paragraph{Faktorisasi Prima} Adalah sebuah cara untuk memperoleh faktor-faktor prima dari sebuah bilangan. Contoh:\\
	\PrimeTree{36}
	\PrimeTree{81}
	\PrimeTree{20351}
	\PrimeTree{2234}
	
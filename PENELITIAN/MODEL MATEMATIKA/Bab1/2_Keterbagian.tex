\paragraph*{}
		Suatu bilangan dikatakan habis dibagi oleh suatu bilangan lain jika hasil baginya adalah bilangan bulat. Misalnya 6 habis dibagi 3 karena $6 : 3 = 2$ sedangkan 7 tidak habis dibagi 3 karena $7 : 3 = 2$. 
\par		
		6 habis dibagi 3 dinotasikan dengan $3|6$ (hasilnya bukan bilangan bulat) sedangkan 7 tidak habis dibagi 3 dinotasikan dengan $3\nmid 7$. Dengan demikian dapat dikatakan bahwa a membagi b (diberi simbol a|b), jika ada suatu bilangan bulat k sedemikian hingga $b = ka$.
		Sedangkan jika a tidak habis membagi b, maka ada bilangan bulat c yang
		merupakan sisa dari pembagian b oleh a, sehingga dapat ditulis sebagai bentuk $b = ka + c.$\par 
		Sifat: Jika $a|b$ dan $c|b$ maka $ac|b$. Contoh : $3|72$ dan $4|72$, maka $12|72$ \par 
		Sifat-sifat khusus pada pembagian bilangan bulat : 
		\begin{enumerate}
			\item Suatu bilangan terbagi oleh 9 jika dan hanya jika jumlah angka-angkanya
			terbagi oleh 9
			\item Suatu bilangan terbagi oleh 3 jika dan hanya jika jumlah angka-angkanya
			terbagi oleh 3
			\item Suatu bilangan terbagi oleh 2 jika dan hanya jika angka terakhirnya terbagi
			oleh 2
			\item Suatu bilangan terbagi oleh 4 jika dan hanya jika dua angka terakhirnya
			habis dibagi oleh 4
			\item Suatu bilangan terbagi oleh 8 jika dan hanya jika tiga angka terakhir
			bilangan tersebut habis dibagi 8
			\item Suatu bilangan terbagi oleh 6 jika dan hanya jika bilangan tersebut habis
			dibagi oleh 2 dan habis pula dibagi oleh 3
			\item Jika abcdefg.... adalah suatu bilangan, maka abcdefg habis dibagi 11 jika
			dan hanya jika 11| (a+c+e+g+...)-(b+d+f+...)
			\item suatu bilangan habis dibagi a dan juga habis dibagi b, jika dan hanya
			bilangan tersebut akan habis dibagi ab dengan syarat a dan b relatif prima. \textit{(Dua bilangan dikatakan relatif prima, jika faktor persekutuan terbesarnya
			(FPB) dua bilangan tersebut sama dengan 1). Contoh: 36 habis dibagi 4
			dan 3, maka 36 habis dibagi 12 ($4\times 3$). Sedangkan 36 habis dibagi 12 dan 6,
			tetapi 36 tidak habis dibagi 72 ($12 \times 6$)	}
		\end{enumerate}		
\paragraph{Model Soal UNBK 2017/2018}
	\begin{enumerate}
		\item Tentukan nilai p yang merupakan digit dalam bilangan dalam persamaan
		berikut ini.
		\begin{center}
			$81 \times 586794=475p0p14$\\
		\end{center}
		\textbf{Jawab} : Bilangan habis dibagi 9 karena hasil perkalian dari 81. Oleh
		karena itu, jumlah digit-digitnya habis dibagi 9 pula. Dengan kata lain 9|
		(4+7+5+0+1+4+2p) = 9| (21+ 2p). Sedangkan bilangan terdekat dengan 21
		yang habis dibagi 9 adalah 27, maka persamaan menjadi 21+ 2p = 27,
		sehingga p = 3.
	\item Berapakah sisa hasil bagi dari,
	\begin{center}
			$878787878787....8724 \div 4 = ....$
	\end{center}
	Karena dua angka yang paling belakang dapat dibagi empat sisanya berarti nol
	\item Dapatkah,
	\begin{center}
		$11|625812?$
	\end{center}
	Ya karena $11|(6+5+1)-(2+8+2)=11|12-12=11|0$.
	\item 100 dibagi x sisa 9, dan 80 dibagi x sisa 8, berapakah x?\\
	\textbf{Jawab}: $100=mx+c=mx+9$, dan $80=nx+8$, dapat kita sederhanakan menjadi $90=mx$ dan $72=nx$, FPB dari keduanya adalah 9 jadi $x=9$;
	\item Dalam sebuah permainan jika $x+y=45$ dan $xy=25$, berapakah $x^2+y^2$. \\
	\textbf{Jawab :} Karena,\\ 
	$(x+y)^2=45^2$\\
	$x^2+2xy+y^2=2025$\\
	$x^2+y^2=2025-50=1975$
	\end{enumerate}
	
		
\paragraph*{}
	Bilangan prima adalah bilangan yang hanya dapat dibagi oleh bilangan itu sendiri dan 1. Bilangan prima merupakan faktor dari semua bilangan komposit.\textit{Fundamental Theorem of Arithmetic}: Semua bilangan komposit merupakan hasil dari perkalian bilangan-bilangan prima yang unik.
	Contoh:
	\begin{enumerate}
		\item Bilangan 12 merupakan faktorisasi yang unik dari bilangan $12=2^2\times 3$  
	\end{enumerate}
	\textbf{Karakteristik Bilangan Prima :}
	\begin{enumerate}
		\item Jumlahnya tak terbatas
		\item Hanya dapat terbagi oleh 1 dan bilangan itu sendiri
		\item Angka dua adalah satu-satunya prima genap
		\item Bilangan komposit adalah bilangan-bilangan yang dapat difaktorkan menjadi angka-angka prima.
	\end{enumerate}
	Bilangan-bilangan yang \textbf{relatif prima} adalah bilangan-bilangan yang tidak memiliki faktor persekutuan satu sama lain. Contoh : $2,3,\text{dan} 5$ adalah relatif prima karena masing-masing tidak memiliki faktor persekutuan.
	\textbf{Contoh :} 
	\begin{enumerate}
		\item 17 dan 24 adalah relatif prima karena masing-masing tidak memiliki faktor persekutuan.
		\item 9 dan 8  adalah relatif prima karena masing-masing tidak memiliki faktor persekutuan
	\end{enumerate}
	\textbf{Kegunaan Relatif Prima:}
	\begin{enumerate}
		\item Relatif prima digunakan untuk menentukan nilai terkecil penyederhanaan akar\\
		Contoh : $\sqrt{54}=\sqrt{3^3 \times 2}$\\
				 $3\sqrt{3\times 2}=3\sqrt{6} $ karena 3 dan 2 adalah relatif prima maka tidak bisa diakar.
		\begin{enumerate}
			\item $ $
		\end{enumerate}
		\item Relatif prima digunakan untuk menentukan penyerdahanaan pecahan apakah dapat disederhanakan lagi atau tidak. Contoh:  $\dfrac{8}{21}$ tidak dapat disederhanakan karena $\dfrac{2^3}{3\times 7}$ adalah \textbf{relatif prima}
	\end{enumerate}
	
	\textbf{Model Soal UNBK 2017/2018}
	\begin{enumerate}
		\item Berapakah penyederhanaan dari $\dfrac{10}{66}$\\
		\textbf{Jawab:} $\dfrac{5\times \cancelto{1}{2}}{11\times \cancelto{1}{2} \times 3}=\dfrac{5}{33}$ \textbf{ingat relatif prima}
		\item Berapakah penyederhanaan dari $\sqrt{2\times 3\times 7\times 11\times 13\times ......}$\\
		\textbf{Jawab:} Tidak bisa disederhanakan. Ingat relatif prima.
		\item Berapakah penyederhanaan dari $\sqrt{150}$\\
		\textbf{Jawab :} 
		$\sqrt{5^2\times 3 \times 2}$, karena 3 dan 2 adalah relatif prima maka $5\sqrt{2\times 3}=5\sqrt{6}$
	\end{enumerate}
	
	
	
	
	
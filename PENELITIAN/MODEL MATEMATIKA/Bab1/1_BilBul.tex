
	\paragraph*{}
	
	Bilangan bulat adalah bilangan yang terdiri atas $...,-1,0,1,2,..$. Atau dapat dikatakan bahwa bilangan bulat memenuhi $x \in N$ dengan $N$ adalah bilangan bulat positif dan negatif, sedangkan nol termasuk kedalam bilangan nonnegatif dan nonpositif atau bisa disebut sebagai bilangan identitas. Siswa diminta untuk dapat memahami operasi dari bilangan bulat tersebut terutama dalam operasi pembagian, perkalian, penambahan dan pengurangan, termasuk jarak,temperatur, dan selisih.  Bilangan bulat termasuk bilangan rasional karena dapat diubah dalam bentuk $\frac{a}{b}$. Operasi bilangan bulat adalah sebagai berikut.
	\begin{enumerate}
		\item \textbf{Operasi Penambahan} 	: $x+y=m$\\
		Jika $x<y$ atau $x>y$ dan $x,y>0$ maka $m>0$ misal $1+2=3$\\
		Jika $x<y$, dengan kondisi $x<0$,$y>0$  maka $m>0$ misal $-3+5=2$\\
		Jika $x>y$, dengan kondisi $x>0$,$y<0$, maka $m<0$ misal $3+(-5)=-2$\\
		Jika $x<y$ atau $x>y$, dengan kondisi $y<0$,$x<0$  maka $m<0$ misal $-3+(-5)=-8$
		%masih harus diperbaiki..........
		\item \textbf{Operasi Pengurangan} 	: $x-y=m$\\
		JIka $y<0$ maka $x-y=x+y=m$ misal $5-(-5)=5+5=10$\\
		Jika $y>0$ maka $x-y=m$ misal $5-(+5)=5-5=0$
		\item \textbf{Operasi Perkalian}	 	: $xy=m$\\
		Jika $x,y>0$ maka $m>0$, Jika $x \text{ atau} y < 0$ maka $m<0$		
		\item \textbf{Operasi Pembagian}   	: $x\div y=m$\\
		Jika $x,y>0$ maka $m>0$, Jika $x \text{ atau} y < 0$ maka $m<0$
		\item \textbf{Jarak, Selisih, Suhu }\\
		Dihitung dengan menggunakan $D=|x-y|$\\
		
	\end{enumerate}
\newpage
\paragraph{MODEL SOAL UNBK 2017/2018}
\begin{enumerate}
	\item A membeli 40 bungkus makanan dengan 10 potong ayam disetiap bungkusnya. Dia ingin membagikan bungkusan tersebut kepada 7 temannya, berapa bungkuskah sisa dari makanan tersebut?
	\item R berada di kota X dan akan menuju kota Y yang jaraknya 10km, dan dari kota Y ke kota Z dengan jarak 25km. Saat di kota Z, R teringat bahwa ada barang yang tertinggal dia kembali lagi ke kota X untuk mengambil barang yang ketinggalan tersebut dan kembali lagi ke kota Z. berapakah total jarak yang R tempuh?
	\item Suhu di kamar ber AC adalah $17^o$ C. Setelah
	AC dimatikan suhunya naik $3^o$ C setiap menit.
	Suhu kamar setelah 4 menit adalah ....
	\item Pada lomba Matematika ditentukan untuk
	jawaban yang benar mendapat skor 2,
	jawaban yang salah mendapat skor $-1$,
	sedangkan bila tidak menjawab mendapat
	skor 0. Dari 75 soal yang diberikan, seorang
	anak menjawab 50 soal dengan benar dan 10
	soal tidak dijawab. Skor yang diperoleh anak
	tersebut adalah ....
	\item Suhu di Jakarta
	pada termometer menunjukkan $34^o$ C(di atas $0^o$). Jika pada saat itu suhu di Jepang ternyata $37^o C$di bawah suhu Jakarta, maka suhu di Jepang adalah ....
	\item Suatu turnamen catur ditentukan bahwa
	peserta yang menang memperoleh skor 6, seri
	mendapat skor 3, dan bila kalah mendapat
	skor $-2$. Jika hasil dari 10 pertandingan
	seorang peserta menang 4 kali dan seri 3 kali,
	maka skor yang diperoleh peserta tersebut
	adalah ....
	\item Suhu dalam ruang tamu $23^o C$. Suhu di dalam rumah $17^o C$ lebih tinggi dari suhu di ruang tamu dan suhu di dalam kulkas $28^o C$ lebih rendah dari ruang tamu. Oleh karena itu suhu di kulkas adalah ....
	
	\item Suhu pagi hari di suatu tempat adalah $-9^o$ C.
	Pada siang harinya mengalami kenaikan
	sebesar $4^o$ C dan pada malam hari suhu
	mengalami penurunan sebesar $8^o$C dan
	bertahan hingga pagi. Suhu pada pagi hari
	berikutnya adalah ....
\end{enumerate}
		
		
		
		
		
		
		
		
		
		
\documentclass[12pt,a4paper,draft,final,oneside,twoside,openright,openany]{article}
\usepackage[latin1]{inputenc}
\usepackage{amsmath}
\usepackage{amsfonts}
\usepackage{multicol}
\usepackage{textcomp}
\usepackage{amssymb}
\usepackage{graphicx}
\usepackage[left=3.00cm, right=3.00cm, top=3.00cm, bottom=3.00cm]{geometry}
\author{Balya Rochmadi}		
\title{Soal Pengayaan Fisika SMA 10 HOTS\\Mapel : Energi 1 \\Closed Book}
\begin{document}
	\maketitle
	\Large
	\noindent\makebox[\linewidth]{\rule{\paperwidth}{0.4pt}}
	\begin{enumerate}
		\item Sebuah blok bergerak ke atas 308 miring di bawah aksi gaya tertentu. Tiga di antaranya ditunjukkan pada Gambar. F1 adalah gaya horizontal dan besarnya 40 N. F2 adalah gaya normal ke bidang dan besarnya 20 N. F3 sejajar dengan bidang dan besarnya 30 N. Tentukan pekerjaan yang dilakukan oleh masing-masing gaya sebagai blok (dan titik aplikasi dari setiap gaya) bergerak 80 cm ke atas tanjakan!
		\begin{center}
			\includegraphics[width=1\linewidth]{/home/dila/Desktop/99Pcture/Selection_058}
		\end{center}
		\item Objek 300 gr meluncur 80 cm di sepanjang meja horizontal. Berapa besar usaha yang dilakukan dalam mengatasi gesekan antara objek dan meja jika koefisien gesekan kinetik adalah 0,20?
		\item Seorang anak mengaplikaskan gaya F kedalam sebuah kolam dengan mendorong kapal mainan di permukaan kolam tersebut, jika perpindahan kapal dapat digambarkan seperti gambar dibawah
		\begin{center}
			\includegraphics[width=0.5\linewidth]{/home/dila/Desktop/99Pcture/Selection_059}
		\end{center}
		Berapakah :
		\begin{enumerate}
			\item Usaha dari x=0 dan x=8
			\item Usaha dari x=8 dan x=12
			\item Usaha total.
		\end{enumerate}
		\item Sebuah box bermassa 6kg diluncurkan dari permukaan yang sangat licin dengan fungsi gaya sebesar $F(x)=18N-(0,563N/m)x$. Jika box dianggap meluncur ke arah kuadran pertama dari diagram kertasius dua dimensi dan $F(x)$ adalah satu-satunya gaya yang ada didalam box, maka berapakah usaha yang dilakukan :
		\begin{enumerate}
			\item Jika box meluncur dair x=0, dan mencapai 14 meter.
			\item Jika box meluncur dari x=15 dan mencapai x=20?
		\end{enumerate}
		\item Sebuah gaya diaplikasikan di sebuah box dengan $\vec{F}=30N\hat{i}-40N\hat{j}$ dan mengalami perpindahan sebesar $\vec{s}=-(9.0m)\hat{i}-(3m)\hat{j}$. Berapakah besar usaha yang dilakukan?
		\item Kamu mengaplikasikan gaya konstan sebesar $\vec{F}=(-68 N)\hat{i}+(36N)\hat{j}$ bearapakah usaha yang dilakukan jika benda berpindah sejauh 48 meter ke arah 240\textdegree berkebalikan jarum jam dari sumbu x positif?
		\item 
		
	\end{enumerate}
\end{document}
\documentclass[12pt,a4paper,draft,final,oneside,twoside,openright,openany]{article}
\usepackage[latin1]{inputenc}
\usepackage{amsmath}
\usepackage{amsfonts}
\usepackage{amssymb}
\usepackage{graphicx}
\usepackage[left=3.00cm, right=3.00cm, top=3.00cm, bottom=3.00cm]{geometry}
\author{Balya Rochmadi}		
\title{Soal Pengayaan Ekonomi SMA 1 }
\begin{document}
	\maketitle
	\begin{enumerate}
		\item Toko Sepatu Sahabat pada akhir tahun melakukan cuci gudang untuk semua jenis sepatu, dari sepatu anak-anak sampai dewasa. Harga sepatu anak yang semula Rp20.000,00 turun menjadi Rp15.000,00. Akibat penurunan harga, jumlah permintaan sepatu anak-anak meningkat dari 1.000 menjadi 4.000. Jadi koefisien elastisitasnya bisa dihitung seperti berikut:
		\item  Di pasar tradisional, harga jeruk lokal mengalami kenaikan dari Rp6.000,00 menjadi Rp7.000,00 per kilogram. Kenaikan harga mengakibatkan permintaan jeruk lokal turun dari 700 kg menjadi 650 kg. perhitungan koefisien elastisitasnya yaitu:
		\item Harga sebuah Drone dari yang semula Rp700.000,00 turun menjadi Rp630.000,00, sehingga permintaan Drone naik menjadi 11.000 yang semula 10.000. Jadi perhitungan koefisien elastisitasnya adalah?
		\item Di pasar tradisional Kota Bandung mengalami perubahan harga setiap minggunya yaitu sekitar Rp4.000,00 sampai Rp6.000,00. Namun, permintaannya selalu sama yaitu berjumlah 1 ton setiap minggu. Perhitungan koefisien elastisitasnya adalah?
		\item Jika faktor produksi barang x naik dan faktor produksi barang y turun, Jelaskan  pengaruhnya terhadap permintaan dan penawaran barang x dan y!
	\end{enumerate}
\end{document}
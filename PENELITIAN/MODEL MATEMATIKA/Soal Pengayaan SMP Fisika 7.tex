\documentclass[12pt,a4paper,draft,final,oneside,twoside,openright,openany]{article}
\usepackage[latin1]{inputenc}
\usepackage{amsmath}
\usepackage{amsfonts}
\usepackage{multicol}
\usepackage{textcomp}
\usepackage{amssymb}
\usepackage{graphicx}
\usepackage[left=3.00cm, right=3.00cm, top=3.00cm, bottom=3.00cm]{geometry}
\author{Balya Rochmadi}		
\title{Soal Pengayaan Fisika SMP 7 HOTS\\Mapel : Induksi Elektromagnetik, Termodinamika \\Closed Book}
\begin{document}
	\maketitle
	\Large
		\noindent\makebox[\linewidth]{\rule{\paperwidth}{0.4pt}}
		
		\paragraph{Petunjuk: }
		\begin{enumerate}
			\item Muai Panjang, Luas, Volume:
					\begin{enumerate}
						\item  Koefisien mulai panjang ($\alpha$), koefisien muai luas ($2\alpha$),koefisien muai volume ($3\alpha$)
						\item Pemuaian Panjang $L_t=L_0(1+\alpha\times\Delta t)$
						\item Pemuaian Luas $A_t=A_0(1+2\alpha\times\Delta t)$
						\item Pemuaian Volume $V_t=V_0(1+3\alpha\times\Delta t)$
					\end{enumerate}
			\item Hukum Gay-Lussac (Pemuaian Gas)
					\begin{enumerate}
						\item Pemuaian Volume pada Tekanan Tetap (Isobarik) \\
						$\dfrac{V_1}{T_1}=\dfrac{V_2}{T_2}$
						\item  Pemuaian Tekanan Gas pada Volume Tetap (Isokhorik)\\
						$\dfrac{P_1}{T_1}=\dfrac{P_2}{T_2}$
						\item Pemuaian Volume Gas pada Suhu Tetap (Isotermis)\\
						$P_1V_1=P_2V_2$
						\item Semua parameter berubah\\
						$\dfrac{P_1V_1}{T_1}=\dfrac{P_2V_2}{T_2}$
					\end{enumerate}
			\item Perubahan suhu terhadap waktu : $Q=mc\Delta t$
			\item Perubahan lebur/uap			: $Q=mL$
			\item Asas Black ;	$Q_1m_1(T_0b-T)=Q_2m_2(T-T_0a) $
			\item Induksi Elektromagnetik Transformator:\\
				$\dfrac{N_p}{N_s}=\dfrac{V_p}{V_s}=\dfrac{I_p}{I_s} $
			\item Efesiensi Transformator : $\eta=\dfrac{P_s}{P_p}\times 100\%=\dfrac{V_sI_s}{V_pI_p}\times 100\%$
		\end{enumerate}
				\noindent\makebox[\linewidth]{\rule{\paperwidth}{0.4pt}}
	\begin{enumerate}
		\item Sebuah benda yang terbuat dari baja memiliki panjang 1000 cm. Berapakah pertambahan panjang baja itu, jika terjadi perubahan suhu sebesar 50°C? ($\text{Koefisien Muai Panjang} : 12\times 10^{-6} \text{C}^{-1}$)
		\item Sebuah bejana memiliki volume 1 liter pada suhu $25^oC$. Jika koefisien muai panjang bejana $2\times 10^{-5}/^{o}C$, maka tentukan volume bejana pada suhu $75^o$C!
		\item Suatu gas di dalam ruangan tertutup memiliki tekanan 1 atm, suhu 27\textdegree C, dan volume 2,4 L. Berapa volume gas tersebut pada suhu 127\textdegree C jika mengalami proses pemuaian pada tekanan tetap?
		\item Volume udara tertutup pada suhu 27\textdegree C adalah 60 liter. Berapa volume udara pada suhu 87\textdegree C jika tekanan tetap ?
		\item 
		Air panas yang memiliki massa 2 kg dicampur dengan air dingin dengan massa 1 kg dalam suatu wadah tertutup. Jika suhu air panas tersebut 80 derajat celcius dan suhu air yang dingin 30 derajat celcius. Berapakah suhu akhir atau suhu campuran dari kedua jenis air tersebut? (diketahui kalor jenis air 4200 J/kg\textdegree C).
		\item Sebuah air 500 gr bersuhu 20\textdegree C dicampurkan dengan air bermassa sama dengan suhu 70\textdegree C. Berapakah suhu akhir air?
		\item Perhatikan Gambar!
		\begin{center}
			\includegraphics[width=0.8\linewidth]{/home/dila/Desktop/99Pcture/Selection_043}
		\end{center}
		\item Jika kalor jenis es 2.100 J/kg\textdegree C, kalor lebur es 336.000 J/kg, dan kalor jenis air adalah 4.200 J/kg\textdegree C maka kalor yang dibutuhkan dalam proses dari P-Q-R?
		\item Perhatikan Gambar!
			\begin{center}
				\includegraphics[width=0.8\linewidth]{/home/dila/Desktop/99Pcture/Selection_044}
				\end{center}
		Bila 2 kg air dipanaskan, dan kalor uap air = $2,27\times 10^{-6} $J/kg, kalor jenis air = 4.200 J/kg \textdegree C dan tekanan udara 1 atmosfer, maka jumlah kalor yang diperlukan untuk proses dari B ke C adalah sebesar?
		\item Sebuah trafo digunakan untuk menaikkan tegangan AC dari 12 V menjadi 120 V. Hitunglah kuat arus primer, jika kuat arus sekunder 0,6 A dan hitunglah jumlah lilitan sekunder, jika jumlah lilitan primer 300.
		\item Sebuah trafo memiliki efisiensi 75$\%$. Tegangan inputnya 220 V dan tegangan outputnya 110 V. Jika kuat arus primer yang mengalir 2 A, berapakah kuat arus sekundernya?
	\end{enumerate}
\end{document}
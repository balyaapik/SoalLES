\documentclass[12pt,a4paper]{article}
\usepackage[latin1]{inputenc}
\usepackage{amsmath}
\usepackage{amsfonts}
\usepackage{amssymb}
\usepackage{graphicx}
\usepackage{cancel}
\usepackage[left=2.00cm, right=2.00cm, top=2.00cm, bottom=2.00cm]{geometry}
\author{H.O.W.K.E}
\title{Kunci Ujian Matematika SMA TAHAP 2\\ TRIGONOMETRI LANJUTAN}

\begin{document}
	\maketitle
	Kunci Jawaban Mulai Nomor 2
	\begin{enumerate}
	\item \textbf{SUDAH JELAS}
	\item $\cos 5x\cos3x-\sin3x\sin x =\cos 2x$\medskip \\
		  $\dfrac{1}{2}(\cos (5x+3x)+\cos(5x-3x))-\dfrac{1}{2}(\cos(3x-x)-\cos(3x+x))=\cos 2x$\medskip \\
		  $\dfrac{1}{2}(\cos (8x)+\cos(2x))-\dfrac{1}{2}(\cos(2x)-\cos(4x))=\cos 2x$\medskip \\
		  $\dfrac{1}{2}(\cos (8x)+\cos(2x)-\cos(2x)+\cos(4x))=\cos 2x$\medskip \\
		  $\dfrac{1}{2}(\cos (8x)+\cos(4x))=\cos 2x$\medskip \\
		  $\dfrac{1}{2}(2(\cos(\dfrac{8x+4x}{2})\cos(\dfrac{8x-4x}{2}))=\cos 2x$\medskip \\
		  $\cos6x\cos2x=\cos2x$\medskip \\
		  $\cos6x=1$\medskip \\
		  Sudah jelas bahwa $x=0$
		 
	\item $\cos 5x + \cos 3x + \sin 5x + \sin 3x = 2\cdot \cos (\dfrac{\pi}{4}-4x)$\medskip \\
		  $2(\cos(\dfrac{5x+3x}{2})\cos(\dfrac{5x-3x}{2})+2(\sin(\dfrac{5x+3x}{2})\sin(\dfrac{5x-3x}{2})=2\cdot \cos (\dfrac{\pi}{4}-4x)$\medskip \\
		  $2(\cos(\dfrac{8x}{2})\cos(\dfrac{8x}{2})+2(\sin(\dfrac{8x}{2})\sin(\dfrac{8x}{2})= 2\cdot \cos (\dfrac{\pi}{4}-4x)$\medskip \\
		  $2(\cos 4x \cos x)+2(\sin 4x \sin x)=2\cdot \cos (\dfrac{\pi}{4}-4x)$\medskip \\
		  $2(\cos 4x \cos x + \sin 4x \sin x)=2\cdot \cos (\dfrac{\pi}{4}-4x)$\medskip \\
		  $\cos 4x \cos x + \sin 4x \sin x=\cos (\dfrac{\pi}{4}-4x)$\medskip \\
		  $cos(4x-x)=\cos (\dfrac{\pi}{4}-4x)$\medskip \\
		  $cos(4x-x)=\cos (-(4x-\dfrac{\pi}{4}))$\medskip \\
		  Sudah jelas bahwa $x=\dfrac{\pi}{4}$
		   
		  \item $\sin x + \cos x -\sin x\cos x=-1$\medskip \\
			   $\sin x (1-\cos x)+ \cos x =-1 $\medskip \\
			   $\sin x(1-\cos x)=-1-\cos x $\medskip \\
			   $\sin x= \dfrac{-1-\cos x}{1-\cos x} $\medskip \\
			   $\sin x = -(\dfrac{1+\cos x}{1-\cos x})$\medskip \\
			   $\sin x = -(\dfrac{1+\cos x}{1-\cos x})\times \dfrac{1-\cos x}{1-\cos x}$\medskip \\
			   
			   $\sin x=-(\dfrac{1-\cos^2 x}{(1-\cos x)^2})$\medskip \\
			   $\sin x=-(\dfrac{\sin^2 x}{(1-\cos x)^2})$\medskip \\
			   $1 =-\dfrac{\sin x}{1-\cos x}$\medskip \\
			   $\dfrac{1-\cos x}{\sin x}=-1$\medskip \\
			   $\tan \frac{1}{2}x=-1$\medskip \\
			   
			   Karena $\tan \theta = -1$ hanya terdapat di $\pi>\theta>\frac{\pi}{2}$ dan $2\pi>\theta>\frac{3\pi}{2}$  maka $x=270^o$ dan $x=630^o$ atau faktor dari dua bilangan itu.
			   
			   
			\item $\sin 2x + \sqrt{3}\cos 2x=2$\medskip	\\
				  $\dfrac{1}{2}\sin 2x + \dfrac{\sqrt{3}}{2}\cos 2x=1$\medskip	\\
				  $\sin2x\cos(\frac{\pi}{3})+\sin(\frac{\pi}{3})\cos2x=1$\medskip \\
				  $\sin(\frac{\pi}{3}+2x)=\sin(\pi)$
				  
				  Jadi,\\
				   $\frac{\pi}{3}+2x=\pi$\medskip \\
					    $2x=\frac{2\pi}{3}$\medskip \\
					    $x=\frac{2\pi}{6}$
				   Atau faktor lain yang $x,k\in R$ dan $x=k\cdot\frac{2\pi}{6}$
			\item Tunjukkan bahwa :\medskip \\
				  $\cos\frac{\pi}{n}+\cos\frac{2\pi}{n}+...+\cos\frac{(n-1)\pi}{n}+\cos\frac{n\pi}{n}=-1 $\medskip \\
				  Langkah-langkah: 	   
				  \begin{enumerate}
				  	\item Sudah diketahui bahwa $\cos\frac{n\pi}{n}=\cos \pi=-1$ Jadi, $\cos\frac{\pi}{n}+\cos\frac{2\pi}{n}+...+\cos\frac{(n-1)\pi}{n}=0$ 
				  	\item Dalam kaidah sigma jika n adalah bilangan ganjil maka $n-1$ pasti bilangan genap sehingga tidak terdapat titik tengah tunggal alias setiap nilai maksimum dan minimum pasti bernilai nol. Pembuktian cukup melalui satu nilai minimum/maksimum saja.\medskip \\
				  	Contoh jika : 1+2+3+4 maka nilai 1+4 = 2+3, begitu pula berlaku di dalam operasi ini.\\
				  	
				  	$\cos \frac{\pi}{n}+\cos \frac{(n-1)\pi}{n}=0$\medskip \\
				  	$2\cos(\frac{\frac{\pi}{n}+\frac{(n-1)\pi}{n}}{2})\cos(\frac{\frac{\pi}{n}-\frac{(n-1)\pi}{n}}{2})=0$\medskip \\
				  	$2\cos \frac{\pi(1+n-1)}{2n}\cos \frac{\pi(1-n+1)}{2n}=0$\medskip \\
				  	$2\cos \frac{n\pi}{2n}\cos \frac{n\pi}{2n}=0$\medskip \\
				  	$2\cos \frac{\pi}{2}\cos \frac{\pi}{2}=0$\medskip \\
				  	$2(0)(0)=0$ \textbf{terbukti untuk n bilangan ganjil} \medskip \\ 
				  	\item Jika n adalah bilangan genap maka $n-1$ adalah bilangan ganjil, sehingga akan terdapat satu anggota urutan ke $\dfrac{1}{2}((n-1)+1)$ yang tidak memiliki anggota impasan ke nilai nol. Maka harus dibuktikan bahwa\medskip \\ $\cos \dfrac{\frac{1}{2}((n-1)+1)\pi}{n}=0$\medskip \\
				  	$\cos \dfrac{\frac{1}{2}(n)\pi}{n}=0$\medskip \\
				  	$\cos \dfrac{1}{2}\pi=0$ \textbf{terbukti untuk n bilangan genap} \medskip \\
				  	
				  	
				  \end{enumerate} 
				
				  
			   
			   
			   			   
			 
			   
			    
		
			    
		  	  
	\end{enumerate}
\end{document}

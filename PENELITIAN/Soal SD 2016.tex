\documentclass[12pt,a4paper,twocolumn]{article}
\usepackage[latin1]{inputenc}
\usepackage{amsmath}
\usepackage{amsfonts}
\usepackage{amssymb}
\usepackage{graphicx}
\usepackage[left=2.00cm, right=2.00cm, top=2.00cm, bottom=2.00cm]{geometry}
\author{H.O.W.K.E}
\title{SIMULASI MATEMATIKA US/M SD 2018\\ CLOSE BOOK\\ 150 Menit }

\begin{document}
	\maketitle
	\paragraph{Bagian I}
	\begin{enumerate}
		\item Hasil dari $1278+1223-1231=...$
		\item Hasil dari $225 \div 5 \times 10=...$
		\item Jika $n=\dfrac{100}{25}$, Berapakah $\dfrac{2n}{5}$?
		\item FEB dan KPK dari 15, 30, dan 55 adalah?
		\item Juminten, Sahroni, dan Rini berenang bersama-sama di telaga warna pada tanggal 31 Agustus 2018. Jika Juminten, Sahroni, dan Rini berenang masing-masing 8, 6, dan 12 hari sekali. Pada tanggal berapa mereka berenang bersama-sama lagi?
		\item Usia Tono dua kali usia Tini. Jika Jumlah usia mereka berdua adalah 42 Tahun, berapakah selisih usia mereka berdua?
		\item Perbandingan tiga kantung beras A,B,C adalah 2:3:7, Jika jumlah kantung beras B dan C adalah 30 kg, berapakah selisih kantung beras B dan A?
		\item $9721^2-9720^2=...$
		\item $\sqrt{\dfrac{256}{49}}=...$
		\item 42 adalah $20\%$ dari ?
		\item $1\dfrac{1}{4} \times 2\dfrac{2}{5}=.... $
		\item Tono bersepeda, berangkat dari rumah pukul 05.30 dan sampai ke sekolah pukul 07.00. Jika jarak sekolah Tono adalah 0,5 km dari rumah, berapakah kecepatan Tono?
		\item Yunda berangkat bekerja pada pukul 11.00, sedangkan Yuyun berangkat satu jam lebih lama dari Yunda. Jika kecepatan rata-rata Yunda dan Yuyun adalah 40km dan 50km berturut-turut, pada jam berapa Yunda menyusu Yuyun.
		\item Berapakah 10$km^2$ + 2$ ha$ + 5$hm^2=....$
		\item Urutkanlah bilangan berikut (Dari yang terkecil)!
			\begin{enumerate}
				\item 20\%, $\dfrac{1}{6}$, $\dfrac{4}{5}$, $\dfrac{1}{3}$, $33\dfrac{1}{3}\%$ 
				\item $\dfrac{12}{24}, \dfrac{1}{4}, \dfrac{4}{3}, 66\dfrac{2}{3}\%, 1\dfrac{3}{4}$
			\end{enumerate}
		\item Dalam peta, panjang sebuah wilayah adalah 5 cm dan lebarnya 2 cm. Jika dalam legenda skala dari peta tersebut adalah 1: 500 cm. Berapakah luasnya?
		\item Dalam sebuah pertandingan catur waktu \textit{player} bermain dibagi dalam 4 sesi, yaotu 0,5 jam, 30 menit, 340 detik, dan 500 detik. Berapakah total waktu dalam permainan tersebut?
		\item Zaenal menaiki puncak gunung  dengan jarak 12 km, kemudian dia berjalan turun mengitari gunung tersebut dengan jarak 5000 m. Berapa km-kah dia berjalan?
		\item Sebuah Bak berbentuk balok dengan panjang, lebar, dan tinggi masing-masing 10m, 8m, dan 5m, diisi air dengan kecepatan 100 Liter per detik. Berapakah waktu yang dibutuhkan untuk mengisi setengah dari bak tersebut?
		\item Berapak luas persegi dengan keliling 40 cm?
		\item Perhatikan gambar berikut!
		\begin{center}
			\includegraphics[scale=0.75]{../Desktop/g4156}
		\end{center}
		Jika luas lingkaran A adalah 154 $cm^2$. Berapakah luas persegi dalam gambar tersebut?
		\item Trapesium PQSR adalah trapesium sama kaki dengan panjang PQ=5 cm, PR= 5 cm, QS= 5cm, dan SR=10 cm, jika tinggi dari trapesium tersebut adalah 4 cm, berapakah luasnya?
		\item Perhatikan gambar berikut!
		\begin{center}
			\includegraphics[scale=0.80]{../Desktop/g4214}
		\end{center}
		Jika tinggi dabel dari tabung tersebut adalah 20 cm, berapakah valumenya? 
		\item Berapakah volume kubus dengan luas sisi 36 $cm^2$?
		\item Berapakah median dan rata-rata dari 12,12,24,12,45,41,40,18,19,19?
		
		
	\end{enumerate}
	\paragraph{Bagian II}
	\begin{enumerate}
		\item Buatlah Tabel frekuensi dari data dibawah ini!\\
		24,24,78,78,91,91,91,94,91,95,91,\\93,88,89,81,87,87,86,86,86,87,87,\\87,87,88,88,91,99, 98,96,96,99,88\\
		\item Buatlah Diagram batang dari soal nomor 1 (Dalam Persens).
		\item Data berikut adalah data hasil ternak ayam di \textit{Farm.Co}:
		\begin{enumerate}
			\item Ayam Cemani : 60 ekor
			\item Ayam Alas	  : 70 ekor
			\item Ayam Petelur : 120 ekor
			\item Ayam Pedaging : 100 ekor
			\item Ayam Kapas	: 10 ekor.
		\end{enumerate}
		Gambarkanlah diagram lingkarannya!
			\item Sebuah kelas dengan anak 30 buah memiliki rata-rata nilai matematika 70. Jika seorang anak masuk kedalam kelas tersebut kemudian rata-rata kelas dihitung kembali maka rata-ratanya menjadi 75. Berapakah nilai anak yang masuk tersebut?
			\item Berapakah luas balok dengan luas sisi-sisi 10$cm^2$, 30 $cm^2$, 20 $cm^2$?
		\end{enumerate}
		
			
\end{document}



\documentclass[12pt,a4paper]{article}
\usepackage[latin1]{inputenc}
\usepackage{amsmath}
\usepackage{amsfonts}
\usepackage{amssymb}
\usepackage{graphicx}
\usepackage[left=4.00cm, right=3.00cm, top=3.00cm, bottom=3.00cm]{geometry}
\author{H.O.W.K.E}
\title{KUNCI UJIAN MATEMATIKA 1 SMA, ZENIUS}
\begin{document}
	\maketitle
	\begin{enumerate}
		\item Sudah jelas, $\sin(180-A)=\sin A$
		\item $\sin^2 \theta = 1-\cos^2 \theta$\\
			  \\
			  $\sin \theta=\sqrt{1-cos^2 \theta}$\\
			  \\
			  $\sin \theta=\sqrt{1-(\dfrac{1}{a}^2)}$\\
			  \\
			  $\sin \theta=\sqrt{\dfrac{a^2-1}{a^2}}$\\
			  \\
			  $\sin \theta=\dfrac{\sqrt{a^2-1}}{a}$\\
			  
			  karena $\sin \theta$ ada di kuadran IV berarti $\sin$ bernilai negatif atau\\
			  \\
			  $\sin \theta=-\dfrac{\sqrt{a^2-1}}{a}$\\
			  \item memakai triple pythagoras kita dapatkan $\tan \theta = \dfrac{4}{3}$
			  \item $\cos^2 \dfrac{\pi}{6} - \sin^2 \dfrac{3\pi}{4} + 8\sin $
		
	\end{enumerate}
\end{document}

\documentclass[10pt,a4paper]{article}
\usepackage[utf8]{inputenc}
\usepackage[english]{babel}
\usepackage{amsmath}
\usepackage{amsfonts}
\usepackage{amssymb}
\usepackage{cancel}
\usepackage{graphicx}
\author{Balya Rochmadi}
\title{Kunci Soal TPM 2016 Matematika DIY C62}
\begin{document}
    \maketitle
	\begin{enumerate}
	
	\item $\mid 29-(-14) \mid=43$
	\item $2\dfrac{1}{5} : 1,20 - 25\% = \dfrac{11}{5} \times \dfrac{5}{6} - \dfrac{1}{4} = \dfrac{11}{6}-\dfrac{1}{4}=\dfrac{22-3}{12}=\dfrac{19}{12}=1\dfrac{7}{12} $
	\item Perbandingan jumlah kelereng Ali dan Budi 3:4, dan Budi dan Dedi adalah 5:2 berarti perbandingan ketiganya adalah:
		\begin{enumerate}
		\item Mencari KPK Budi di perbandingan Ali dan Budi dan Budi dan Dedi yaitu,\\
			  \begin{tabular}{|c||c|c|c|}
			  	\hline 
			  	\rule[-1ex]{0pt}{2.5ex} Ali : Budi & 3 & 4 &  \\ 
			  	\hline 
			  	\rule[-1ex]{0pt}{2.5ex} Budi : Dedi &  & 5 & 2 \\ 
			  	\hline 
			  	\rule[-1ex]{0pt}{2.5ex} Ali : Budi : Dedi & $20:4 \times 3=15$  & 20 & $20:5 \times 2=8$ \\ 
			  	\hline 
			  	\end{tabular} 	
			  	\\
			   \textbf{Jadi Ali : Budi : Dedi=15:20:8}
		 \item Selisih kelereng Dedi dan Ali jika jumlahnya 129 adalah:\\  $\dfrac{15-8}{15+20+8}\times 129=21$
		 
		\end{enumerate}
	\item $\dfrac{16}{2}p^{5-(-3)}q^{-6-(-3)}=8p^8q^{-3}=\dfrac{8p^8}{q^3}$
	\item $3\sqrt[]{4\times 5}-5\sqrt{5}+6\sqrt{15\times 5}: 2\sqrt{15}=$\\
		  $6\sqrt[]{5}-5\sqrt{5}+\cancelto{3}{6}\cancelto{1}{\sqrt{15}} \sqrt{5}: 
		  \cancelto{1}{2}\cancelto{1}{\sqrt{15}}=$\\
		  $\sqrt{5}+3\sqrt{5}=4\sqrt{5}$ 
	\item Memakai perbandingan biasa jika untung maka 100\% + keuntungan jadi 
		 \\
		 $\dfrac{\text{modal}}{\text{keuntungan}}\times \text{keuntungan}= \dfrac{100}{125}\times 500.000= 400.000$
	\item Diferensinya adalah $-3-n$ jadi -4,-5,-6,-7,-8 sehingga jawabannya adalah -1 dan -9
	\item $U_3=a+2b$ atau $11=a+2b$ dan $U_{11}=a+10b$ atau $43=a+10b$\\
	jadi, cara eliminasi dapat digunakan \\
	\begin{tabular}{cccc}
	  $11=$&$a+2b$&  &  \\ 
	  $43=$&$a+10b$& $-$  &  \\ 
	  \hline
      $-32=$&$-8b$	&  & \\
	  $  b=4$& & &
	\end{tabular} \\
	jadi a dapat dicari dengan substitusi ke\\ $11=a+2(4)$ dan $a=3$
	maka suku ke 50 adalah\\
	$U_{50}=3+49(4)=3+196=199$
	\item mencari \textit{r} dari deret geometri tersebut\\
	 Diketahui : $U_1=3$ dan $U_5=48$ ditanyakan $S_5$\\
		$3=ar^{1-1}=ar^0=a$\\
		\\
		$48=ar^{5-1}=(3)r^4$\\
		$48=3r^4$ \medskip \\ 
		$\dfrac{48}{3}=r^4$\medskip \\
		$16=r^4$\\
		$r=2$\\
		Jadi panjang tali mula-mula adalah Jumlah deret geometri tersebut,\\
		$3+6+..+48$\\
		atau\medskip \\
		$S_n=\dfrac{a(r^n-1)}{r-1} $ untuk $r>1$\bigskip \\
		$S_5=\dfrac a{2^5-1}{2-1}=3(31)=93$
	\item Bentuk lainnya adalah \\
	    $3x^2+2x-6x-4-2(x-1)(x+1)$\\
	    $3x^2-4x-4-(2x-2)(x+1)$\\
	    $3x^2-4x-4-2x^2+2x-2x+2 $\\
	    $x^2-4x-2$
	 \item nilai $y-5$ adalah...
		$15-6y=4y-25$\\
		$15+25=4y+6y$\\
		$40=10y$\\
		$y=4$\\
		jadi $y-5=-1$
	\item Himpunan dari $\dfrac{3x+4}{5}\leq \dfrac{2x+2}{3}$\\
		  $3(3x+4)\leq 5(2x+2)$\\
		  $9x+12\leq 10x+10$\\
		  $12-10\leq 10x-9x$\\
		  $2\leq x$ atau
		  $x\geq 2$\\
		  jadi himpunannya adalah \{2,3,4,5,6,....\}
	\item $K\cup L=$\{a,b,c,d,e,f,g \}
	\item Ingat! $A \cup B=\text{Semesta}-\text{Anggota Bukan Peristiwa}$\\
	dan\\
	$A\cup B=A+B-A\cap B$ \\
	jadi \\
	$A\cup B= 40-6=34$ \\
	maka\\
	$34=21+23-A\cap B$\\
	$A\cap B=10$
	\\
	Maka yang mengikuti PMR saja adalah $21-10=11$ orang
	\item Dapat kita ambil x,y nya adalh \{1,3\} dan \{2,5\}jadi,
	p dan q adalah\\
	$3=p+q$ dan $5=2p+q$ maka\\
	\begin{tabular}{ccc}
	$3=p+q$	&   \\ 
	$5=2p+q$& $-$ \\
	\hline
	$-2=-p$ & \\
	$p=2$ &
	\end{tabular}
	;Otomatis $q=1$\\
	
	Jadi $f(x)=2x+1$\\
	
	maka, $13=2a+1$ dan $a=6$\\
		  $b=2(11)+1=23$\\
    jadi $23+6=29$
    \item $f(x)=4x+3 dan f(x-2)-f(x)=...$\\
		   $4(x-2)+3-4x-3=4x-8+3-4x-3=-8$
	\item sumbu berpotongan saat y=0 maka x=5 dan pada saat x=0 maka y=4, karena gradiennya negatif jadi gambar garisnya miring ke kanan.
	\item $m=\dfrac{y_2-y_1}{x_2-x_1}=\dfrac{6-5}{2-4}=-\dfrac{1}{2}$\\ karena tegak lurus maka $m_1=-\dfrac{1}{m_2}$ jadi $m_1=-\dfrac{1}{-\dfrac{1}{2}}=2$
	\item $D=\sqrt{180^2+800^2}=820$
	\item $x+y=5.000$ dan $4x+5y=155.000$
	\item Persyaratan tiga buah sisi a,b,c dengan $c>a \text{ dan } c>b$ maka\\
	$a-b<c<a+b$
	\begin{enumerate}
		\item $28-17<45\nless 28+17$ (Salah)
		\item $28-9<31<9+28$ (Betul)
		\item $27-18<48\nless 27+18$ (Salah)
		\item $19-15<24<15+19$ (Betul)
	\end{enumerate}
	jawabannya b dan d
	\item Perhatikan gambar berikut!
	\begin{center}
\includegraphics[scale=0.5]{"TPM DIY IMAGE/Segitiga "}

\end{center}
	Luas daerah yang diarsis $A_\triangle BDF=A_\triangle ADB - A_\triangle BFA$\\
	Ingat bahwa $\overline{DF}=\overline{AF}=a$ jadi $\overline{AD}=2a$\\
	Luas $\triangle ADB$ dapat dihitung sebagai berikut\\
	$A_\triangle ADB=\dfrac{1}{2}\cdot 2 \cdot a \cdot t=at$\medskip \\
	$A_\triangle AFB=\dfrac{1}{2}at$
	
	$A_\triangle BDF=A_\triangle ADB - A_\triangle BFA=at-\dfrac{1}{2}at=\dfrac{1}{2}at$\\
	$A_\triangle BDF=\dfrac{1}{2}at$\\
	$16=\dfrac{1}{2}at$\\
	$32=at$\\
	Karena Luas yang tidak diarsir adalah $A_\triangle AFD \text{ dan } A_\triangle DBC$, perlu diingat bahwa $A_\triangle AFD=\dfrac{1}{2}at=\dfrac{1}{2}(32)=16$\\
	Ingat pula bahwa $A_\triangle ADB=A_\triangle BDF=at=32$
	maka luas bagian yang tidak diarsir,\\
	$A_\triangle AFD + A_\triangle DBC=15+32=48$
	\item $180=6x+5+a$ dan $a=x+7$ jadi,\\
		  $a-7=x$\\
		  masukkan kedalam persamaan lainnya,\\
		  $180-5=6(a-7)+a$\\
		  $175=6a-42+a$\\
		  $217=7a$\\
		  $a=31$
	\item $\angle BDE = \angle ADF$ dan sudut $\angle CDA= \dfrac{1}{2}\angle EDF$.\medskip \\
		  ingat pula bahwa $\angle CDA + \angle ADF = 90^o$\\
		  
		  jadi, $\dfrac{1}{2}(5x-13)+2x-7=90$\medskip \\
			    $\dfrac{5}{2}x-\dfrac{13}{2}+2x-7=90$ \medskip \\ 
			    $5x-13+4x-14=180$ ;\textit{kalikan semua ruas dengan dua}\\
			    $9x-27=180$\\
			    $9x=207$\\
			    $x=23$\\
		  jadi, $\angle BDE=2(23)-7=46-7=39^o$
	 \item $\angle OAC=\angle ACO= 48$ jadi $180=2(48)+ \angle AOC$\\
		   $\angle AOC=84$ sehingga $\angle ABC=\dfrac{1}{2} \angle AOC= 42$
	 \item $\dfrac{66-44}{66+44}\times 100 = 20 $cm 
	 \item $AB=\dfrac{12}{16}\times 24= 18$\\
		   $BE=\dfrac{4}{16}\times 24= 6$\\
		   $AD=\dfrac{12}{4}\times 3= 9$
		   Luas trapesium ABCD = $\dfrac{18+12}{2}\times 9=135cm^2 $
	\item Sudah jelas, C atau D.
	\item Sudut, sisi, Sudut
	\item 16 rusuk
	\item Sudah jelas, i dan iv
	\item $=(\dfrac{1}{2}\times \dfrac{4}{3}\times \dfrac{22}{7} \times 10,5^3)+(\dfrac{1}{3}\times \dfrac{22}{7}\times 10.5^2 \times 30)=3528cm^3$
	\item Luas bola $=\dfrac{1}{2}\times 4\dfrac{22}{7}(7^2)=308$ dikali biaya $308\times 125.000=38.500.000$
	\item $(12)(12)+(4)(12)(8)+4(0.5)(10)(12)=768$
	\item $V=25^2\times 3.14\times 80=157000cm^3=157L$ dan diisi ke kantong plastik masing masing 0,25L jadi $\dfrac{157}{\dfrac{1}{4}}=628$ kantong
	\item Median terdapat di data ke 20 dan 21 $(132+133):2=132.5 $
	\item Siswa Laki-laki=l\\ siswa perempuan=p\\
		  $35=p+l$ jadi dapat dikatakan bahwa $l=35-p$\medskip \\
		 $\dfrac{148l+141p}{l+p}=144$\medskip \\
		 $\dfrac{148(35-p)+141p}{35}=144$\medskip \\
		 $5180-148p+141p=5040$\\
		 $7p=140$\\
		 $p=20$\\
		 
		 Terdapat siswa putri sebanyak 20 orang.
	\item Sudah jelas B.
	\item $6\times 4=24$
	\item $\dfrac{3}{8}$ (AAG,GAA,AGA)
		 
	
		  
		  
	
	
	
	

	

	 
		
	
	
	
	
	\end{enumerate}

\end{document} 
\documentclass[14pt,a4paper]{article}
\usepackage[utf8]{inputenc}
\usepackage[english]{babel}
\usepackage{amsmath}
\usepackage{amsfonts}
\usepackage{amssymb}
\usepackage{makeidx}
\usepackage{graphicx}
\usepackage{lmodern}
\usepackage[left=4cm,right=3cm,top=4cm,bottom=3cm]{geometry}
\title{Bukti bahwa $2^n \geq n!$ Menggunakan Induksi Matematika}
\author{Balya Rochmadi}
\begin{document}
	\maketitle
	Pertama sebelum dilakukan pembuktian domain yang dimaksud adalah $\{n \geq 0\}$.
	Asumsikan bahwa $n=k$ maka $2^k \geq k!$. Sehingga jika $n=k+1$ maka $2^{k+1}\geq (k+1)!$ dengan $n=k\geq 0$. Jadi $2^{k+1}-(k+1)!\geq 0 $. Perlu diingat bahwa dengan adanya $2^k \geq k!$, berarti,\\
\begin{center}
$2^{k+1}-(k+1)!=2\cdot2^k-(k+1)! > 2\cdot k!-(k+1)!$\medskip \\
$2^{k+1}-(k+1)!\geq 2\cdot k!-(k+1)!$\medskip \\
$2^{k+1}-(k+1)!\geq 2\cdot k!-(k+1)k!$\medskip \\
$2^{k+1}-(k+1)!\geq 2\cdot (1-(k+1))k!$\medskip \\
$2^{k+1}-(k+1)!\geq 2\cdot (k)k!$\medskip \\
\end{center}
karena sisi kanan dari pertidaksamaan tersebut bernilai positif, ingat bahwa $k=n\geq  0$, maka terbukti bahwa $2^{k+1}-(k+1)!\geq 0 $ atau $2^k \geq k!$, dan persamaan ditunjukkan jika dan hanya jika $k=0$. 	  						
	  						
\end{document}